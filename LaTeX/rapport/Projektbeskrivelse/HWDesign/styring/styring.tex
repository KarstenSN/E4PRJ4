\subsection{Motor / Styring}


For at kunne styre bilens fremdrift og retning er bilen blevet udstyret med en H-bro og en RC servo motor\cite{lib:wiki-RC-Servo}.
H-broen styrer bilens motor ved hjælp af et PWM og 2 digitale signaler fra Pi kontrolleren. 
Hjertet i H-broen er L298N\cite{lib:L298N_datablad}, som er en Fuld H-Bro Motor Kontroller. 
Den bestemmer med de 2 digitale signaler om motoren skal få bilen til at kører frem eller tilbage. 
PWM signalet bestemmer hvor stærkt bilen kører. 
L298N giver også mulighed for at bremse bilen.\newline 
Vi har valgt selv at design et motorstyrnings  print ud fra design anbefalingerne for L298N  i MultiSim og Ultiboard. 
Det er designet med henblik på at reducere EMC støj mest muligt.\newline
På samme måde som bilens motor bliver bilens styretøj også styret af et PWM til servo motoren. 
Servoens udslag til siderne bliver kontrolleret ved at lave et PWM signal med en duty cycle mellem 0,5ms og 2,5ms. 
Det giver fuldt udslag til henholdsvis venstre og højre på ca. 40\si{\degree} til hver side.
Det var oprindeligt tiltænkt at begge PWM signaler skulle styres af det indbyggede PWM på Pi. 
Men det viste sig ikke muligt da det kun var muligt at sætte én PWM frekvens for begge PWM udgange. 
Derfor bliver motorens PWM, der er 40kHz, styret af hardwaren. 
Mens servoens PWM, der er ca. 50Hz, styret af en software PWM. 
Det er valgt for at belaste Pi'ens cpu mindst muligt.
For at de logiske signaler fra Pi'en til servo passer sammen er der designet et lille konverter print der hæver signalet fra 3,3V til 5V.
