\section{Projektgennemførelse} \label{ch:Projektgennemfoerelse}


Fra projektets opstart har det i gruppen været et fælles mål at   



- Roller
- Ren E-gruppe

- Scrum / V-model/ ASE-model 

Gruppen har i løbende projektfasen gjort brug af en kombination af udviklingsværktøjerne ASE-modellen, V-modellen samt Scrum. 
Kombinationen af disse er valgt på baggrund af ønsket om at kunne optimere hver enkelt fase i projektet.
Således er V-modellen brugt til at realisere ASE-modellen for at skabe den grundlæggende kravspecifikation og systemarkitektur, hvorefter Scrum-værktøjet tog mere over i design, implementering og testfasen. 
Værktøjerne er brugt i det omfang de gav mening for projektet, og i det omfang det har kunnet lade sig gøre rent tidsmæssigt. 
Da alle gruppemedlemmer også i de tidligere semesterprojekter har benyttet disse værktøjer, har der været rig mulighed for at gøre brug af hinandens erfaringer og opnå større effektivitet i projektstyringen. 
Se afsnit \ref{ch:Metoder} \nameref{ch:Metoder} på side \pageref{ch:Metoder} for nærmere beskrivelse. 
Projektet er gennemført med ugentlige planlagt møder både som arbejdsgruppe og møder med vejleder, samt et fælles ønske om en høj arbejdsindsats hvilket indebar brug af alle fritimer, samt anden fritid til at fuldføre projektet. Projektet er ført efter den overordnede tidsplan der blev udarbejdet i tidligt i projektet, \cite{lib:Tidsplan}


%TODO
OBS: Stjålet fra PRJ3

Gruppen, som er en videreførelse fra 2. semesterprojekt, har løftet opgaven med fornyet engagement og endnu større handlekraft end tidligere. Fra begyndelsen af projektperioden har arbejdet med projektet været relativt uden problemer. Det har hele tiden været gruppens mål, som på foregående semester, at holde sig foran tidsplanen\cite{lib:Tidsplan}, men samtidig have en fornuftig tilgang til arbejdet. Dette har for gruppen betydet en forøget arbejdsindsats i form af anvendelsen af alle fritimer, der var mulige at bruge.  
Forskellen på dette semesterprojekt og gruppens tidligere, er en meget mere klar opdeling i hardware og software. Opdelingen er kommet som en naturlig konsekvens af opdelingen i uddannelserne E/EP/IKT, og har samtidig betydet øget fokus på de relevante dele af projektet for de individuelle medlemmer. 

I projektet er anvendt en kombination af udviklingsmodellerne V-model, Scrum og ASE-modellen, som i en stor blanding, gav muligheden for udarbejdelsen af gruppens projektdokumentation og rapport. Se afsnit \ref{ch:Metoder} \nameref{ch:Metoder} på side \pageref{ch:Metoder} for nærmere beskrivelse. Modellerne er ikke nødvendigvis fulgt fuldstændigt, men gruppen har efterhånden udarbejdet sin egen fortolkning, som er meget velfungerende. Gennem projektperioden er hvert overemne blevet kørt som et sprint, men det er  først i forbindelse med design og implementering at der konkret kan tales om reelle sprint. 
Eftersom gruppen har valgt at fortsætte fra et tidligere semester, er mange af tingene som blev udarbejdet tidligere, fx samarbejdsaftale, mødeskabeloner og opgaver blevet genbrugt. Genanvendelsen af delelementer har gjort opstarten af projektet en del nemmere, end hvis der skulle startes fra bunden.

Rollerne der er blevet fordelt i projektet ser ud som følger:

\begin{itemize}
	\item Koordinator \newline
		Morten har haft det overordnede ansvar for administrative opgaver, som mødeindkaldelser, referater og 			logførelse.
	\item Ordstyrer \newline
		Philip har været ordstyrer igennem gruppens vejleder- samt arbejdsmøder.
	\item Dropbox Ansvarlig \newline
		Kristian T. har stået for orden og udlægning af deletjenesten.
	\item GitHub ansvarlig \newline
		David har været ansvarlig for kildekodedelingen over GitHub.
	\item SCRUM ansvarlig \newline
		David har været overordnet ansvarlig for anvendelsen af SCRUM.
	\item Lokale booking \newline
		Kristian S. har været ansvarlig for at booke lokaler når det var nødvendigt.
\end{itemize}

\clearpage