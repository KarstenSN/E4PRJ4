\section{Projektgennemførelse} \label{ch:Projektgennemfoerelse}


Fra projektets opstart har det i gruppen været et fælles mål at sætte kvalitet fremfor kvantitet samt at holde en god projektstyringsprofil.  
Som  det første semesterprojekt er denne gruppe dannet alene af Elektro-studerende, hvilket samlet set har givet et tættere arbejde HW/SW-grupperne imellem end på det forrige semester hvor der var et tydelige skel herimellem. 
Tilfældet denne gang var et tættere samarbejde grupperne imellem, samt at en markant større del af projektet var programmeringsorienteret i forhold til hardwareorienteret. 

Gruppen har løbende i projektfasen gjort brug af en kombination af udviklingsværktøjerne ASE-modellen, V-modellen samt Scrum. 
Kombinationen af disse er valgt på baggrund af ønsket om at kunne optimere hver enkelt fase i projektet.
Således er V-modellen brugt i forbindelse med ASE-modellen for at skabe den grundlæggende kravspecifikation og systemarkitektur, hvorefter Scrum-værktøjet tog mere over i design, implementering og testfasen. 
Denne disposition blev valgt på grundlaget af det mulige ugentlige antal arbejdestimer i starten af projektperioden kontra den sidste del af projektperioden hvor gruppen har haft mulighed for at bruge mere tid på projektet.
Værktøjerne er brugt i det omfang de gav mening for projektet, og i det omfang det har kunnet lade sig gøre rent tidsmæssigt. 
Da alle gruppemedlemmer også i de tidligere semesterprojekter har benyttet disse værktøjer, har der været rig mulighed for at gøre brug af hinandens erfaringer og sparring,  og hermed opnå større effektivitet i projektstyringen. 
Der henvises til afsnit \ref{ch:Metoder} \nameref{ch:Metoder} på side \pageref{ch:Metoder} for nærmere beskrivelse afudviklingsværktøjerne.

Projektet er gennemført med ugentligt planlagte møder, både som arbejdsgruppe og møder med vejleder, samt et fælles ønske om en høj arbejdsindsats hvilket indebar brug af alle fritimer og anden fritid til at bringe projektet op på det ønskede høje niveau. 
Projektet er ført efter den overordnede tidsplan der blev udarbejdet tidligt i projektet, \cite{lib:Tidsplan}, der var som udgangspunkt indlagt tidsbuffere til at opfange evt. uforudsete hændelser og samtidig var den planlagt så der var mulighed for også at arbejdet grundigt.

Ved projektopstart blev følgende roller fastlagt for at strømline projektstyringen, rollerne blev fordel således: 

\begin{itemize}
	\item Ordstyrer \newline
		Philip har været ordstyrer ved gruppens vejleder- samt arbejdsmøder.
	\item Referent \newline
		Lasse har været referent ved gruppens vejleder- samt arbejdsmøder.
	\item Dropbox og LaTeX ansvarlig\newline
		Kristian har stået for orden og udlægning af deletjenesten, samt været support på LaTeX-værktøjet.
	\item GitHub ansvarlig \newline
		Karsten har været ansvarlig for GitHub.
	\item SCRUM, Vejlederkommunikations ansvarlig\newline
		Kenn har været ansvarlig for anvendelsen af Scrum, kommunikation med vejleder, samt mødeindkaldelser og dagsorden.
	\item Lokalebooking og projektkalender\newline
		Henrik har været ansvarlig for lokalebooking og administration af projektkalenderen.
	\item Logfører \newline
		Jesper har været ansvarlig for at fører log.
\end{itemize}
\clearpage