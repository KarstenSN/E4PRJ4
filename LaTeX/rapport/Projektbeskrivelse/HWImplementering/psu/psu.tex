\subsection{Strømforsyning}

Under implementering af bilens strømforsyning har der været et par enkelte problemer, bl.a. med at skaffe dele men også i form af en mindre fejl i designet.
Første gang strømforsyningen blev koblet til forsyning, efter at alle komponenter var loddet på printet, var der problemer med at strømforsyningen konstant gav omkring 0.5 V på udgangen og ikke 5V, som der var designet efter.
Det viste sig at der i designet, som er beskrevet i afsnit \ref{sec:hwd_psu}, var lavet en mindre fejl ift. LM26003's egen VDD-udgang.

I databladet var det specificeret at VDD-udgangen maks måtte belastes med 1 mA, for at kunne holde bootstrap kondensatoren tilstrækkeligt opladet.
Måden LM26003 er implementeret på internt, er ved at den switch, som åbner op for forsyningsspænding til buck-converteren med en meget høj switchfrekvens, er en N-channel mosfet.
Det er gældende at gaten skal have en tilstrækkelig positiv spænding i forhold til drain for at transistoren kan lede strøm. 
Da drain i dette tilfælde er koblet til forsyningen på 7.2V, viser det sig at være besværligt at holde en konstant spænding der er højere end det.
Det er her at bootstrap-kondensatoren kommer ind i billedet.
Den er koblet imellem LM26003's BOOT-ben og udgangen på LM26003 (benene SW1-3), hvilket medfører at når transistoren er OFF, lades en spænding op over bootstrap-kondensatoren, hvor strømmen forsynes internt fra VDD.
Når transistoren herefter går ON hæves udgangen til forsyningsspænding og BOOT ligger nu på ca 2 gange forsyningsspændingen og holder derved transistoren åben.

Problematikken med designet af strømforsyningen lå i at kondensatoren, som skal sørge for at VDD kan oplade bootstrap-kondensatoren ikke var inkluderet i designet og at den lysdiode som skulle lyse når strømforsyningen fungerer, trak for stor en strøm (ca 5 mA).
Alt i alt blev VDD altså belastet hårdere end den er designet til og transistoren kunne derfor ikke åbne tilstrækkeligt for at udgangen opførte sig som forventet.
De to problemer blev udbedret ved at klippe formodstanden til LED'en over og ved at eftermontere en kondensator fra VDD til stel.

Efter disse udbedringer virkede strømforsyningen inden for tolerancen på forbrugerkredsløbene. 

