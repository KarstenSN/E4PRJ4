\subsection{Strømforsyning}

Under implementation af bilens strømforsyning har der været et par enkelte problemer, bl.a. med at skaffe dele til denne men også i form af en mindre fejl i designet af strømforsyningen.
Første gang strømforsyningen blev koblet til efter at alle komponenter var loddet på printet var der problemer med at strømforsyningen konstant gav omkring 0.5 V på udgangen og ikke de 5V, som det var ønsket.
Det viste sig at der i designet, som er beskrevet i afsnit \ref{sec:hwd_psu}, var lavet en mindre fejl ift. LM26003's egen VDD udgang.

I databladet var det specificeret at denne maks måtte belastes med 1 mA, for at kunne holde bootstrap kondensatoren tilstrækkeligt ladet.
Måden LM26003 er implementeret på internt er ved at den switch, som åbner op for forsyningsspænding til buck converteren med meget høj switchfrekvens, er en N-channel mosfet.
For disse er det gældende at gaten skal have en hvis positiv spænding i forhold til drain for at transistoren kan lede strøm. 
Da drain i dette tilfælde er koblet til forsyningen på 7.2V viser det sig at være besværligt at holde en konstant spænding der er højere end dette.
Det er her at bootstrap kondensatoren kommer ind i billedet.
Denne er koblet mellem LM26003's BOOT-ben og udgangen på LM26003 (benene SW1-3), hvilket medfører at når transistoren er OFF, lades en spænding op over bootstrap kondensatoren, hvor strømmen forsynes internt fra VDD.
Når transistoren herefter går ON hæves udgangen til forsyningsspænding og BOOT ligger nu spændingsmæssigt på ca 2 gange forsyning og holder derved transistoren åben.

Problematikken med designet af strømforsyningen lå i at den kondensator, som skal sørge for at VDD kan oplade bootstrap kondensatoren ikke var inkluderet i designet OG at den lysdiode, som skulle lyse når strømforsyningen fungerer som den skal, trak for stor en spænding (ca 5 mA).
Alt i alt blev VDD altså belastet meget hårdere end den er beregnet til og transistoren kunne derfor ikke åbne tilstrækkeligt for at udgangen opførte sig som forventet.
Disse to problemer blev udbedret ved at klippe formodstanden til LED'en over og ved at eftermontere en kondensator fra VDD til stel.

Efter disse udbedringer virkede strømforsyningen næsten helt som forventet, der var en smule offset på udgangsspændingerne ift. det ønskede, men det var indenfor de krav, som forbrugerkredsløb stillede.

%Nedenfor kan eventuelt bare strejes ved behov af plads
Under implementering af strømforsyningen blev der dannet erfaringer indenfor lodning af SMD-komponenter, design af og lodning på eget PCB board samt vikling af spoler/transformatorer. 
Ydermere er værd at notere at der blev opnået en højere forståelse af hvordan buck convertere kan drives med N-channel mosfets.
