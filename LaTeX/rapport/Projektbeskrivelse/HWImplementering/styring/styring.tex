\subsection{Motor / Styring} \label{sec:hwi_motor_styring}
Mens motorstyrings printet blev fremstillet igennem værkstedet blev software delen af motorstyringen testet med et af skolens print. Vi har i tidligere projekter arbejdet med H-broer så der var ikke de store problemer eller ændringer i forhold til designet motorstyrings printet.\newline
Det var derimod en større udfordring at få servo motoren til at virke korrekt.  
Data bladet til servo motoren Corona-CS238MG \cite{lib:Corona-CS238MG} er meget mangelfuldt. Og derfor svært at finde den rigtige frekvens af PWM signalet den kræver. 
Ved 50Hz viste den sig at være meget ustabil og ville ikke reagere rigtigt på PWM signalet. 
Efter nogle forsøg med en funktionsgenerator fandt vi frem til at den var mest stabil mellem 150 og 200Hz. 
Signal konverteren til servo motoren viste sig også at give nogle udfordringer og der blev lavet et nyt design af det.
Problemet var at alt så fint ud på oscilloskopet når servoen ikke var koblet til. 
Men når den var tilkoblet støjede den på PWM signalet så det ikke havde pæne lige flanker vi så. 
Det betød at servo ikke reagerede rigtigt og blev ustabil.
Da hele bilen med alle dele blev samlet så det ud til at servoen blev lidt ustabil igen. Det kan muligvis skyldes at PWM signalet, der styrer servo, er styret i ren software. 
Det giver nogle udfordringer i forhold til load af Pi'ens cpu. Da kamera softwaren \texttt{Motion} blev installeret på Pi'en blev det konstateret at den brugte op til 80\% af cpu kraften. Det kan have indvirkning på hvor nøjagtigt pulsbredde er. Og det vil bevirke at servoen opfatter det som om at den skal flytte sig og derved bliver ustabil