\section{Systemarkitektur} \label{ch:Systemarkitektur}

Systemarkitekturen fungerer som udgangspunkt for designfasen af hele systemet.
Den danner overblik over det samlede systemet, hvorved det deles op i mindre blokke. 
De mindre blokke beskriver efterfølgende systemet i nærmere detaljer, således man kan derefter kan gå direkte til design af systemet. For den detaljerede systemarkitektuk henvises til kapitel \ref{P-sec:sysark} \nameref{P-sec:sysark} på side \pageref{P-sec:sysark} i dokumentationen.

\begin{figure}[h]
\centering
\includegraphics[width=0.55\textwidth]{../fig/diagrammer/bdd_au2.pdf}
\caption{Overordnet BDD for AU2}
\label{fig:bdd_au2}
\end{figure}

På figur \ref{fig:bdd_au2} ses et overordnet BDD for AU2. 
Det ses at systemet AU2 består af en bil og en PC. 
Bilen er her det centrale element i systemet, hvilket kommunikerer med den software der kører på PC'en. 
Denne tilgang er valgt for at kunne fjernstyre bilen med XBox-360 controlleren igennem PC'en over Wi-Fi. På figur \ref{fig:bdd_bil} på side \pageref{fig:bdd_bil} ses et udvidet BDD for blokken Bil.  

På figur \ref{fig:bdd_bil} ses et overordnet BDD for bilen, den oprindelig plan var at tachometeret skulle være en del af Fremdrift, dette var for at lave én stor logisk blot omkring baghjulene på bilen.
Dog blev dette ændret i forbindelse med introduktionen af PSoC blokken, da denne sørger for kommunikation med samtlige af bilens sensorer og derved også erstatter den blok der tidligere var kaldet ''Sensorer''.

\clearpage

\begin{landscape}

\begin{figure}
\centering
\includegraphics[width=\linewidth]{../fig/diagrammer/bil/bdd_bil.pdf}
\caption{BDD for bil}
\label{fig:bdd_bil}
\end{figure}
\end{landscape}

\clearpage

Formålet med figur \ref{fig:bdd_bil} er at give et logisk overblik over de fysiske dele der er påmonteret bilen. 
Blokkene er opdelt med henblik på deres respektive ansvarsområde i forhold deres funktionalitet. Derfor ses blokkene med tilhørende parts, som herefter kan nedbrydes i moduler til design og implementering.

\begin{figure}[H]
\centering
\includegraphics[width=\textwidth]{../fig/diagrammer/bil/ibd_bil.pdf}
\caption{IBD for bil}
\label{fig:ibd_bil}
\end{figure} 

På figur \ref{fig:ibd_bil} ses IBD for blokken Bil. 
Her er beskrevet hvilke signaler der forbinder blokkene samt deres indehold. 
Det ses at blokken \emph{Fremdrift} er forbundet med \emph{motorctrl} som er styringssignalet til motoren.  
I flow-specifications udspecificeres signalernes indehold. 
Diagrammet skal læses i forbindelse med figur \ref{fig:bdd_bil} på side \pageref{fig:bdd_bil}. 
Det noteres at Strømforsyningen er forbundet til samtlige blokke i IBD Bil Forsyning, som findes på figur \ref{P-fig:ibd_bil_forsyning} på side \pageref{P-fig:ibd_bil_forsyning} i dokumentationen.

\clearpage