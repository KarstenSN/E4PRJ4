\subsubsection{Domain-klasse: Settings} \label{sec:settings}
Tanken med Settings-klassen var at bilen skulle kunne gemme indstillinger i en fil til næste gang bilen tændes.
Selve filfunktionaliteten blev ikke implementeret grundet fokus på andre aspekter af Pi-softwaren, men selve klassen bruges stadig af de øvrige dele af programmet.

Settings-klassen har til formål at håndtere de indstillinger brugeren sætter på PC softwaren. 
Disse indstillinger skrives til klassen fra PcCom-klassen og læses af Aks- og Steering-klassen. 
For at sikre at den indeholdte data ikke bliver korrupt, sørges der for at der ikke er flere tråde der kan læse og skrive til de samme data samtidigt. 
Dette gøres vha. \texttt{std::mutex} der er implementeret gennem C++11 standardbiblioteket \texttt{mutex}. 
Denne klasse er efter implementering testet med de andre klasser.