\subsubsection{Boundary-klasse: PcCom} \label{sec:pccom}
PcCom-klassens formål er at give mulighed for PC softwaren at skabe kontakt mellem Bil og PC. 
Den blev som udgangspunkt designet med en UDP protokol, men efter implementering af PC software blev dette skiftet til en TCP protokol, da softwaren var blevet implementeret således. 
PcCom-klassen er designet og implementeret som to tråde der hver især åbner en server med TCP sockets til at styre to forskellige former for datastreams mellem Bil og PC.
Trådene er implementeret ved brug af biblioteket \texttt{thread}, som giver anledning til konstruktion af et object af typen \texttt{std::thread}. 
Disse objekter er tråde der hver især er en sekvens af instruktioner, der kan udføres sammen med andre sådanne sekvenser i multithreading miljøer. 
Der er under design og implementering ligeledes draget meget nytte af ''Sockets Tutorial'' \cite{lib:socket_tutorial}, en instruktion i at implemntere TCP scokets på en linux maskine.