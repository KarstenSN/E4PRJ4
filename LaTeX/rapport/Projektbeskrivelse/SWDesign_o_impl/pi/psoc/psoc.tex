\subsubsection{Boundary-klasse: Psoc} \label{sec:psoc_klasse}

Oprindeligt var det tiltænkt at Pi skulle have en klasse til alle afstandssensorer og en klasse til tachometeret.
I forbindelse med at PSoC blokken blev introduceret, viste det sig mere effektivt at samle grænsefladen for de to blokke i én klasse.
Dette er gjort ud fra en optimering af læsehastighed fra de forskellige sensorer internt i Pi.

Psoc-klassen håndterer al \IIC-kommunikation imellem Pi og PSoC. Herunder, distancesensorerne og tachometeret. 
Klassen er implementeret med public-metoderne \texttt{getDistance()} og \texttt{getVelocity} hvor der returneres hhv. afstandensværdien fra den ønskede sensor i cm, samt hastigheden fra tachometeret i $km/t$. 
Internt i klassen kaldes \texttt{update()} og \texttt{psocRead()}, for at holde klassens attributter opdateret med det seneste datasæt til når Pi'en kalder en get-kommando. 
Derudover indeholder klassen en pointer til Log-klassen, her logges når klassen opretteres eller nedlægges, samt hvis der forekommer fejl under afviklingen. For flere detaljer om Psoc-klassen på Pi'en henvises til \myRef{P-sec:sw_design_psoc_pi} i dokumentationen.
Klassen er desuden udviklet således den er beskyttet imod, at flere forskellige tråde skriver eller læser fra atributterne samtidig. Dette er gjort ved hjælp af \texttt{std::mutexes}.