\subsubsection{Boundary-klasse: Steering} 

Steering-klassen skal håndtere styringen af bilens aktuatorer, dvs motor og servo. 
Til forskel fra PWM signalet til motoren er PWM signalet til servoen softwaregenereret, da frekvensen for signalet til servoen er væsentlig lavere end til motoren.

Det eneste interface til klassen er public metoden \texttt{userInput(UserInput)}. 
Metoden bliver kaldt hver gang der er nye data fra Brugeren. 
Klassen henter udover ny data fra brugeren også den aktuelle hastighed fra Data klassen, makshastighed fra Settings klassen og den kan anvende et Log objekt til at skrive i loggen.
Når klassens constructor bliver kaldt, opretter den en separat tråd til opdatering af motor PWM signalet. 
Det var oprindelig tiltænkt at bilens hastighed skulle reguleres af en PID regulering, men reguleringen er på nuværende tidspunkt implementeret som en P regulering.
Dette skyldes problemer under integrationsfasen, gruppen valgte at fjerne ID-leddene helt for at fokusere på at få den basale funktionalitet op at køre.
For at forenkle grænsefladen til Pi'ens pins, er der brugt et eksternt bibliotek, som hedder WiringPi \cite{lib:wiringpi} til at kontrollere PWM signaler både via HW og SW samt at kontrollere de digitale udgange. 