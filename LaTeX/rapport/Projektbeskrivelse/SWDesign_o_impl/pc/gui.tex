\subsection{GUI}
I arbejdet med GUI’en var det første der skulle findes ud af, hvilket sprog og udviklingsmiljø softwaren skulle skrives i. Henrik og Karsten gik begge i gang med at undersøge hver sit. Henrik undersøgte mulighederne i C\# og visual studio. Han fik hurtigt en GUI op og stå, hvori der kunne modtages videostream. Karsten gik i gang med C++ og Qt creator, her tog det dog længere tid at finde en løsning til at modtage videostream, men det var et kendt sprog og det var nemt at designe det grafiske. På et møde samlede begge deres erfaringer og det blev i gruppen besluttet at køre videre med Qt. Beslutningen blev argumenteret med at vi senere ville få faget ISU, hvor vi ville lære om tråde og deres kommunikation indbyrdes i C++. Det ville derfor være en fordel at kunne drage nytte af det i projektet. Henrik havde desuden fundet et bibliotek til at håndtere Xbox360-controlleren skrevet i C++. Dette ville ikke kunne inkluderes i C\#.
I selve arbejdet med GUI’en var det som før beskrevet en udfordring at finde en løsning til at modtage videostream i. I Qt creator er der en indbygget webbrowser som kan trækkes ind i den grafiske del af GUI’en. Denne startes nemt i constructoren af koden, hvor der kan gives en URL med ved start. Men desværre virkede det ikke med et videostream og det blev derfor undersøgt om det var muligt at tilføje en tilføjelse til browseren, som det kendes fra fx Chrome eller Firefox, osv. Dette var muligt, men blev hurtigt så kompleks at det ville være nemmere at bruge open source biblioteker fra VLC mediaplayer. Selvom at denne løsning umiddelbart virkede simplere, viste det sig dog at være ligeså tidskrævende. Der blev fundet flere guides på nettet, men disse var alle fyldt med fejl, da der enten var døde links eller henvisninger til biblioteker som ikke længere var tilgængelig. Det var derfor en sammensætning af dem alle som der gjorde det muligt at komme videre. I dokumentationens litteraturliste findes linket \cite{lib:vlc-using-qt} der er brugt som udgangspunkt. 
Da arbejdet med GUI’en startede før faget ISU, var der på det tidspunkt ikke nogen der viste noget om tråde. Det tog derfor et stykke tid at finde ud af hvordan disse skulle implementeres og hvordan der sendes signaler fra den ene tråd til den anden. Der blev fundet en bog på nettet om hvordan man programmerer i QT \cite{lib:qt-bog} s. 13-37 \& 381-396. fra denne er der fundet inspiration til at løse problemet med tråde samt \texttt{signals and slots}. Igennem arbejdet med GUI'en er der sideløbende udviklet en testserver, som skriver output fra GUI'en ud i en terminal, samt sender tilfældigt genereret data tilbage til GUI'en. På den måde kan størstedelen af GUI'en tested uden at der nødvendigvis behøver at være forbindelse til bilen. 
Da projektet var ved at nærme sig aflevering var der et problem som der manglede en løsning på. Problemet er at når der er oprettet forbindelse til bilen og controlleren er forbundet, ligger der to TCP-forbindelser i hver sin tråd at kører. Hvis bilen kommer uden for rækkevidde vil GUI’en miste forbindelsen til bilen og GUI’en crasher. Dette skyldes at når en TCP-socket kører i en tråd, kan den ikke sende signaler til hovedvinduet som ellers skal sørge for at der ikke længere sendes data når forbindelsen er tabt. Programmet crasher derfor når den skriver til en forbindelse som ikke længere eksisterer.Problemet er også beskrevet i dokumentationen.