\subsection{PSoC} \label{sec:swd_psoc_board}

Det blev valgt at overflytte distancesensorerne og tachometeret til at køre selvstændigt på PSoC'en i stedet for at have direkte bus-kontakt med Pi'en. 
Dette viste sig at være en fordel, da der opstod problemer med bla. direkte adgang til timere på Pi, samt et problematisk interface i forbindelse med \IIC kommunikation til distancesensorerne.
Et positivt resultat heraf blev, at de processor PSoC'en håndterer, og som kan ske meget ofte, nu kører selvstændigt på PSoC'en og dermed ikke forstyrrer Pi'en.
Pi'en ville ofte blive afbrudt, hvis den skulle afvikle PSoC'ens funktionalitet, hvilket vil påvirke hastigheden af processoren på Pi'en, der som resultat heraf sandsynligvis ville arbejde langsommere.
Problemerne undgås helt, når at Pi'en kun skal foretage én læsning fra PSoC'en og returnere det læste, hvorefter den uforstyrret kan fortsætte med sit program.
PSoC'en er programmeret til altid at have en buffer med det seneste datasæt klar til at Pi'en kan aflæse det. 

Det direkte problem i forbindelse med \IIC lå i, at når distancesensorerne skal programmeres til at foretage en scanning, kræver dette en kombineret skrivning/læsning til enheden. 
Dette viste sig efter lang tids arbejde med ''trial and error'', ikke at være muligt at implementere på Pi'en i den form som projektet krævede. 
Derfor blev det besluttet af overflytte sensorerne til PSoC'en, og herefter lade denne fungere som kombineret master/slave-enhed på \IIC bussen. 
Således fungerer den som Master for sensorerne, men som slave for Pi'en. For detaljer om den implementeringen af PSoC'en hensvises til afsnit \ref{P-sub:sw_impl_psoc_psoc} \nameref{P-sub:sw_impl_psoc_psoc} på side \pageref{P-sub:sw_impl_psoc_psoc} i dokumentationen.

PSoC'en er sat op med en timer, der kører kontinuert og således at tachometerets output skal tilsluttes til [P1.0], som trigger interrupts, når der detekteres en nedadgående flanke i spændingen på benet. 
Selve interrupt routinen gemmer den seneste timerværdi, aflæser en ny, finder differencen mellem disse og udregner hastigheden, som set i afsnit 
\ref{P-sec:tachometer_hw_impl} \nameref{P-sec:tachometer_hw_impl} på side \pageref{P-sec:tachometer_hw_impl} i dokumentationen. 
Når bilen kører med maksimal hastighed ca. $10km/t$ betyder dette at der vil være lige under 72 interrupts pr. sekund.

Ved implementering viste PSoC'en sig, som forventet, simplere og mere effektiv, da der allerede findes et API-set med \IIC kommandoer, som gør det muligt at kommunikere med sensorerne på det ønskede niveau. 
Det blev undersøgt på Pi'en hvorledes det var muligt at indføre et kernemodul, som ville tillade at aflæse processorens clockcycles eller få adgang til en timer, men det viste sig meget at være meget besværligt, hvorfor PSoC'en blev valgt til denne funktion. 
I første omgang stødte gruppen på et problem forbindelse med aflæsning af PSoC'en, da distancesensorerne som minimum tager 100ms om at foretage en valid scanning. I denne tid kan sensoren ikke afbrydes; hvis der i dette interval foretages en læsning fra Pi'en, forekommer korrupte værdier. 
Der blev pga. dette truffet et designvalg at samle alle sensordata i ét datasæt, som så kan afhentes af Pi'en vha. én læsning.
Dette gav markant mere stabile værdier ved aflæsning og test, hvorfor det også er denne løsning der er valgt til implementering i prototypen.