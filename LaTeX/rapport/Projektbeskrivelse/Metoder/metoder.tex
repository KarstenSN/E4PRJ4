\section{Metoder} \label{ch:Metoder}

Under Metoder vil de forskellige arbejdsmetoder, der er blevet brugt under dette projekt blive beskrevet. Disse metoder er hhv. V-model, SysML, Reviews og Versionsstyring.

\subsection{V-Model og ASE-Model} 
Under projektets forløb er V-Modellen fulgt, som en vejledning til udførelsen af projektet. Modellen er dog ikke fulgt fuldstændigt, da der ikke er blevet defineret flere testscenarier ud over Accepttesten. Dette skyldes, at gruppen har fundet det mere hensigtsmæssigt at lave løbende tests. Derved har det været svært at fastsætte mindre tests imellem de forskellige enheder i tidligere stadier. Det vil sige at tests såsom modultests blev beskrevet og bearbejdet sideløbende med design- og implementeringsfasen.

Ud over V-Modellen \cite{lib:T-006}, er ASE-Modellen \cite{lib:vejledning} taget i brug som en vejledning til udførelse af projektet. Der er hovedsageligt lagt fokus på at gøre det muligt for hardware og software at dele sig op under design og implementering. Dette har stillet større krav til systemarkitekturen, da kommunikation mellem hardware og softwaren grænsefladerne har skulle være veldefinerede for et effektivt arbejde.

\subsection{SysML}
Gruppen har anvendt SysML primært i systemarkitektur-fasen for at beskrive systemet bedst muligt ud fra logiske blokke.
Grænsefladerne mellem blokke har ikke været begrænset til elektriske signaler eller softwarekald, men også mekaniske elementer samt andre ikke-elektronikrelaterede elementer er beskrevet. 
Muligheden for derefter at beskrive overordnede blokke har været positiv for gruppen i og med at det har givet mulighed for forskellige abstraktionsniveauer i udvikling af projektet.
Veldefinerede grænseflader mellem blokke ved hjælp af SysML har også bidraget til en mere klart defineret designfase.

\clearpage

\subsection{Versionsstyring}
Versionerhistorik på dokumenter i projetdokumentationen er blevet opdateret løbende bla. i forbindelse med kommentarer fra vejleder og reviews. Væsentlige ændringer i fx design har givet anledning til versions-ændring, hvilket hjælper med at holde styr på hvilke ændringer projektet har gennemgået.
Udover den manuelle versionsstyring i hvert kapitel i dokumentationen er al dokumentation af projektet uploaded til Git, som har givet en meget detaljeret versionsstyring.

\subsection{Reviews}
Under projektforløbet er der i slutningen af kravspecifikations modtaget et review\cite{lib:Review1} med en stærkstrømsprojektgruppe, som ligeledes har fået foretaget et review af gruppen.
Reviewet var meget givtigt i det omfang at gruppen indså nogle væsentlige fejl ved det oprindelige udkast til projektets krav og tillod gruppen at få justeret disse tidligt i forløbet.
For målet med reviewet var for gruppen at modtage så objektiv kritik af projektet som muligt og i denne sammenhæng ikke modtage deciderede forslag til forbedringer, men i stedet få sat spørgsmålstegn ved formuleringer og specifikationer.
Efter et modtaget review har gruppen holdt et eget møde, hvori de modtagne kommentarer er blevet diskuteret og eventuelle ændringer er blevet lavet.

Udover reviews fra andre projektgrupper har gruppen ligeledes modtaget kommentarer fra vejleder AJU med samme formål som for det øvrige review.

\clearpage