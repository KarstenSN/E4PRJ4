\section{Henrik}

Dette semesterprojekt har været præget af en projektgruppe, som er en kombination af 3 gamle medlemmer, fra en gruppe jeg har arbejdet med de to sidste semestre, og 3 nye folk jeg ikke har arbejdet med før. Dette har givet nogle udfordringer i både hastighed og udarbejdelse af projektet. Jeg har gået fra en gruppe der allerede kendte hinanden og ikke var bange for langhårede diskussioner, hvor denne gruppe først har skulle lære hinanden at kende.

Semesterprojektet har været et med mange udfordringer, dette jeg har udviklet mig utrolig meget. Fagligt føler jeg at jeg har fået rigtig meget ud af forløbet. Jeg har primært siddet med programmering af Raspberry Pi'en i projektet, og jeg føler at jeg virkelig har rykket mig i embedded programmering. Ligeledes har jeg i fællesskab med Philip været en af personerne der har haft et overordnet overblik over softwaren på Pi'en. Hovedpunkterne jeg har udviklet mig inden for er datahåndtering, API, socket og tråd programmering, samt sikkerhed blandt disse, på et embedded system.
Ud over dette har jeg været til rådighed for de andre medlemmer hvis de har haft brug for en ekstra hånd. Dette har hovedsageligt været med henblik på kommunikationsveje mellem Pi og andre enheder.