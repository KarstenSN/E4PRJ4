\chapter{Opnåede Erfaringer}
\section{Fælles erfaringer}
Der er bred enighed om at kombinationen af fire medlemmer fra en gammel gruppe og tre ''nye'' medlemmer har skabt konflikter i semesterprojekt processen. Den større andel af gruppen har i nogle tilfælde virket dominerende overfor de andre medlemmer og presset deres gamle arbejdsmetoder ned over de ''nye'' medlemmer. Dette har resulteret i en primær fremgangsmåde som ikke alle har været enige i, men ikke har ville sige fra til. I manglen på kommunikation blandt medlemmer, er fremgangsmåden midtvejs blevet splittet, hvilket har resulteret i en rodet systemarkitektur, da folk begyndte på design og implementering før kommunikationsvejene, protokoller og funktionaliteter var blevet aftalt. Dette har gjort at medlemmer har haft meget svært ved at arbejde ud fra de indledende kapitler i dokumentationen, da der hele tiden blev foretaget ændringer, på baggrund af en rodet fremgangsmåde.

Denne gruppe består overordnet af engagerede medlemmer. Engagemangtet har til tider dog skabt miskommunikationer, som har givet nogle problematikker der har præget projektet, det har dog ikke stoppet gruppen men tvært imod styrket den.

Efter et møde blandt gruppens medlemmer blev disse miskommunikationer diskuteret hvor alle havde mulighed for at fortælle deres ''historie'' og synspunkt på problemerne. Dette har hjulpet gruppen til et nyt niveau, som klart ville kunne ses hvis denne gruppe nogensinde kommer til at arbejde sammen igen. Det har ligeledes været en øjenåbner for alle gruppens medlemmer, der helt sikkert kan tages med i fremtidige projekter med andre individer.

\section{Jesper} 

E4PRJ 4 har været meget udfordrende, lærerigt og til tider frustrerende. Jeg har brugt meget tid på at kunne crosscompile til Pi med WiringPi og tråde. Selvom tiden kunne været brugt bedre, end at få værktøjer til at virke, har det stadig været meget lærerigt. Jeg har også lavet en af de få hardware ting i projektet, som er en H-broen til styring af motoren. Vi har tidligere brugt en H-bro men på dette semester har vi lært om EMC og det er forsøgt implementeret i designet af H-broen.
\newline 
skriv skriv mere
%TODO skriv mere
\section{Henrik}
Semesterprojektet har været med mange udfordringer og herved har jeg udviklet mig utrolig meget, både socialt og fagligt.
Jeg har primært siddet med programmering af Raspberry Pi'en i projektet og jeg føler at jeg virkelig har rykket mig i embedded programmering.
Ligeledes har jeg i fællesskab med Philip været en af personerne der har haft et overordnet overblik over softwaren på Pi'en.
Hovedpunkterne som jeg har udviklet mig inden for er: Datahåndtering, API, socket og tråd programmering, samt sikkerhed blandt disse på et embedded system.

Ud over førnævnte har jeg været til rådighed for de andre medlemmer hvis de har haft brug for en ekstra hånd. Dette har hovedsageligt været med henblik på kommunikationsveje mellem Pi og andre enheder.
\section{Kenn}\label{sec:opnaaede_erfaringer_ke}

Arbejdet med projektet har for mit vedkomme bestået af softwarerelateret arbejde i forbindelse med design om implementering af distancesensorerne, samt projektstyring. 
Dette har været en afveksling fra de tidligere projekter hvor jeg primært har arbejdet på hardwaresiden. 
Men efter eget ønske var dette projekt rent software.
Arbejdet med softwaren har givet mig erfaring i programmering af \IIC både på Pi'en og PSoC'en. 
Disse 2 enheder håndterer bussen forskelligt, og dét at skabe et interface hvor disse to kan kommunikere involverede at opsætte PSoC'en som kombinerer Master og slave enhederne. Derudover gav programmering, debugging og crosskompilering på Pi'en nogle rigtig gode erfaringer i håndtering af Linux systemkald og filhåndtering.
I forbindelse med forløbet af projektet har det været interessant og lærerigt at indgå i en gruppe som, for 4 medlemmers vedkommende er videreført fra sidste semester. 
Dette har naturligt nok givet en omstillingsperiode, men også givet en indsigt i andre arbejdsmetoder og man har skulle omstille sig og tilpasse sig nye arbejdsgange. 
Dette har det været meget lærerigt i forbindelse med nye inputs og erfaringer. 
\subsection{Karsten} 
%TODO skal skrives
\section{Kristian}

Jeg har i dette projekt erfaret vigtigheden af at have en ikke-faglig tovholder, som bevarer overblik over tidsplan, møder etc. og selv blevet inspireret til måske at tage denne rolle i en kommende ingeniørgruppe.
Rent fagligt har jeg opnået erfaringer indenfor lodning af små SMD-komponenter, design af kredsløb (dette var for mit vedkommende det første semesterprojekt med ansvar for design og implementering af hardware), samt analyse af et fejlagtigt hardwaremodul i forbindelse med test og udbedring af førnævnte fejl.
Ydermere har jeg lært en masse omkring design af buck convertere og hvilke problemer der opstår ved anvendelse af HF-styresignaler på forsyningselektronik.
Jeg har indset at jeg knapt har rørt overfladen af hvordan man laver en rigtig god switch mode forsyning, men dette har givet inspiration til endnu mere læring indenfor området.
Softwaremæssigt har jeg lært at anvende GNU Make på fornuftig vis, samt at bruge trådprogrammering i en praktisk sammenhæng.
\subsection{Lasse}

AutoGreen har for mig været et spændende og særdeles udfordrende projekt. Selvom projektet virkede simpelt nok, er gruppen alligevel undervejs stødt på en lang række problemer. De problemer der var mulige at løse er via fokus og samarbejde blevet løst tilfredsstillende for det endelige system. Jeg er positivt overrasket over gruppens udvikling, hvilket har gjort sig meget bemærket i form af måden der samarbejdes på nu vs. måden vi samarbejde på i forrige semesterprojekt. Overordnet set er jeg meget tilfreds med forløbet.
\subsection{Philip} 
%TODO skal skrives

\clearpage