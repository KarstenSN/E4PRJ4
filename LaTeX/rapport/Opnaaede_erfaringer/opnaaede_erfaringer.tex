\chapter{Opnåede Erfaringer}
\section{Fælles erfaringer}
Der er bred enighed om at kombinationen af fire medlemmer fra en gammel gruppe og tre ''nye'' medlemmer har skabt konflikter i semesterprojekt processen. Den større andel af gruppen har i nogle tilfælde virket dominerende overfor de andre medlemmer og presset deres gamle arbejdsmetoder ned over de ''nye'' medlemmer. Dette har resulteret i en primær fremgangsmåde som ikke alle har været enige i, men ikke har ville sige fra til. I manglen på kommunikation blandt medlemmer, er fremgangsmåden midtvejs blevet splittet, hvilket har resulteret i en rodet systemarkitektur, da folk begyndte på design og implementering før kommunikationsvejene, protokoller og funktionaliteter var blevet aftalt. Dette har gjort at medlemmer har haft meget svært ved at arbejde ud fra de indledende kapitler i dokumentationen, da der hele tiden blev foretaget ændringer, på baggrund af en rodet fremgangsmåde.

Denne gruppe består overordnet af engagerede medlemmer. Engagemanget har til tider dog skabt miskommunikationer, som har givet nogle problematikker der har præget projektet, det har dog ikke stoppet gruppen men tvært imod styrket den.

Efter et møde blandt gruppens medlemmer blev disse miskommunikationer diskuteret hvor alle havde mulighed for at fortælle deres ''historie'' og synspunkt på problemerne. Dette har hjulpet gruppen til et nyt niveau, som klart ville kunne ses hvis denne gruppe nogensinde kommer til at arbejde sammen igen. Det har ligeledes været en øjenåbner for alle gruppens medlemmer, der helt sikkert kan tages med i fremtidige projekter med andre individer.

\section{Jesper} 

E4PRJ4 har været meget udfordrende og lærerigt. 
Jeg har brugt meget tid på at kunne crosscompile softwaren til Raspberry Pi med WiringPi og tråde. 
Selvom tiden kunne været brugt bedre, har det stadig været meget lærerigt. 
Jeg har udover software også lavet en af de få hardware ting i projektet, som er en H-broen til styring af motoren. 
Vi har tidligere brugt en H-bro, men på dette semester har vi lært om EMC, og det er implementeret i designet af H-broen.
At projektet samlingspunkt denne gang er et embedded system, som Raspberry Pi, har også været interessant og meget lærerigt. 
Erfaringerne jeg har fået kan helt sikker anvendes senere i uddannelsen.
\section{Henrik}

Dette semesterprojekt har været præget af en projektgruppe, som er en kombination af 3 gamle medlemmer, fra en gruppe jeg har arbejdet med de to sidste semestre, og 3 nye folk jeg ikke har arbejdet med før. Dette har givet nogle udfordringer i både hastighed og udarbejdelse af projektet. Jeg har gået fra en gruppe der allerede kendte hinanden og ikke var bange for langhårede diskussioner, hvor denne gruppe først har skulle lære hinanden at kende.

Semesterprojektet har været et med mange udfordringer, dette jeg har udviklet mig utrolig meget. Fagligt føler jeg at jeg har fået rigtig meget ud af forløbet. Jeg har primært siddet med programmering af Raspberry Pi'en i projektet, og jeg føler at jeg virkelig har rykket mig i embedded programmering. Ligeledes har jeg i fællesskab med Philip været en af personerne der har haft et overordnet overblik over softwaren på Pi'en. Hovedpunkterne jeg har udviklet mig inden for er datahåndtering, API, socket og tråd programmering, samt sikkerhed blandt disse, på et embedded system.
Ud over dette har jeg været til rådighed for de andre medlemmer hvis de har haft brug for en ekstra hånd. Dette har hovedsageligt været med henblik på kommunikationsveje mellem Pi og andre enheder.
\section{Kenn}\label{sec:opnaaede_erfaringer_ke}

Arbejdet med projektet har for mit vedkomme bestået af softwarerelateret arbejde i forbindelse med design og implementering af distancesensorerne, samt projektstyring. 
Dette har været en afveksling fra de tidligere projekter hvor jeg primært har arbejdet på hardwaresiden. 
Men efter eget ønske var dette projekt rent software.
Arbejdet med softwaren har givet mig erfaring i programmering af \IIC både på Pi'en og PSoC'en. 
Disse 2 enheder håndterer bussen forskelligt, og dét at skabe et interface hvor disse to kan kommunikere involverede at opsætte PSoC'en som kombinerer Master og Slave enhederne. Derudover gav programmering, debugging og crosskompilering på Pi'en nogle rigtig gode erfaringer i håndtering af Linux systemkald og filhåndtering.
I forbindelse med forløbet af projektet har det været interessant og lærerigt at indgå i en gruppe som, for 4 medlemmers vedkommende er videreført fra sidste semester. 
Dette har naturligt nok givet en omstillingsperiode, men også givet en indsigt i andre arbejdsmetoder og man har skulle omstille sig og tilpasse sig nye arbejdsgange. 
Dette har det været meget lærerigt i forbindelse med nye inputs og erfaringer. 
\section{Karsten}
Som tidligere beskrevet har gruppen været bestående af 4 personer som tidligere arbejdede sammen sidste semester og 3 personer som er gruppens ''nye'' medlemmer. At dette er vigtigt at prioriter skyldes at det har været mærkbart igennem hele samarbejdet. Den ''gamle'' gruppe havde en måde at samarbejde på, hvilket havde forløbet godt for dem sidste semester. Denne arbejdsmetode blev derfor trumfet igennem og de nye måtte derfor acceptere vilkårene. Desværre syntes jeg ikke om denne arbejdsmetode, da den var styret af en eller få personer. En af de store fejl i projektet var at man ikke ville gå i gang med implementeringen, før systemmet var fuldt dokumenteret i designfasen. Dette er selvfølgelig den rigtige metode i følge vores indlæring på 2. semester, men jeg vil betegne den som et solskinsscenarie som ikke kan lade sig gøre i virkeligheden. Da jeg var bekendt med denne fejl fra tidligere semestre, hvor man flere gange har vendt tilbage til designprocessen fra implementeringsprocessen for at rette fejl, var det for mit vedkommende et spørgsmål om at få et ansvarsområde hvor jeg kunne lave en teknologiundersøgelse, samtidig med at jeg kunne lave designet og implementeringen. Jeg gik i gang med at udvikle GUI'en i Qt-creator kort før efterårsferien, hvilket resulterede i at jeg kunne bruge store dele af min fritid på at lave en teknologiundersøgelse, samt implementering, som gjorde designprocessen langt mere præcis og hurtig. Dette ses også ved at jeg var stort set færdig, da størstedelen af gruppen først kom i gang med deres implementering få uger før aflevering. Jeg ser det som et bevis på at førnævnte arbejdsmetode har været en gennemgående fejl. Min erfaring fra dette projekt bliver i overensstemmelse med tidligere: At det er vigtigt at dele ansvarsområder ud til den enkelte, således der ikke kun er få personer som designer hele systemet, at det er vigtigt at lave en teknologiundersøgelse af sit ansvarsområde sideløbende med designprocessen og at det i det hele taget er vigtigt at gruppen samarbejder uden at der er en eller få der tager alle beslutninger.    
\section{Kristian}

Denne projektgruppe har været en delvis arv af min sidste semesterprojektgruppe hvortil der er kommet tre ''nye'' medlemmer.
Dette har givet en del udfordringer i starten af projektet ift. at få ensrettet måden gruppen arbejdede sammen på.
En af de større udfordringer fra mit synspunkt har været at gruppens torvholder fra sidste semester går på en anden retning og der har derfor været et behov for en der har holdt styr på de administrative opgaver i gruppen og presset gruppen videre i henhold til tidsplanen. 
Det jeg har opnået af erfaring i forbindelse med dette er at en projektgruppe meget stærkt har behov af en der kan bevare det ikke-faglige overblik over gruppen og sørge for at ting skrider fremad på fornuftig vis, hvilket ikke er en rolle jeg selv har siddet med tidligere.

Rent fagligt har jeg opnået erfaringer indenfor lodning af mindre SMD-komponenter, design af kredsløb (dette var for mit vedkommende det første semesterprojekt med ansvar for design og implementering af hardware), samt analyse af et fejlagtigt hardwaremodul i forbindelse med test og udbedring af førnævnte fejl.
Jeg har også lært at anvende GNU Make på fornuftig vis, samt at bruge trådprogrammering i en praktisk sammenhæng.

Jeg har, som i tidligere semesterprojekter, haft det faglige overblik over projektet og på et vist niveau fungeret som en form for faglig vejleder mht. nogle af de øvrige medlemmers problematikker under design- og implementeringsfasen. 
Dette har gjort at jeg har været nødt til at lægge fokus en smule væk fra de opgaver jeg personligt har været ansvarlig for, hvilket direkte afspejles i udførslen af de klasser jeg har skrevet på Pi'en.
\section{Lasse} 

E4PRJ4 har været et meget hårdt, men meget lærerigt projekt for mig. 
Jeg synes at jeg har været en del af en dynamisk gruppe, som har et meget højt ambitions niveau. 
Det er gået sent op for mig, hvor stort projektet i virkeligheden er, og jeg har lært en masse om mig selv igennem de sidste dele af projektet. Projektet har været 80 \% software, hvilket jeg tænker er atypisk for et 4. semesters Elektronik projekt, men jeg kan godt lide tanken om at have undersøgt muligheden for dette også. Resultatmæssigt er jeg fint tilfreds med det opnåede resultat, og jeg føler at vi har fungeret som det kan forventes af en nydannet gruppe. 
\subsection{Philip}

Projektforløbet har efter min vurdering båret præg af at denne gruppe er sammensat på ny efter at et rigtig godt samarbejde og ’’fælles sprog’’ var oparbejdet i min tidligere projektgruppe. Den praktiske koordinator fra den tidligere gruppe har manglet og jeg har derfor (primært til sidste i forløbet) forsøgt på gruppens vejene, at prioritere opgaverne da jeg indså at gruppen ikke kunne nå at blive færdig med alle de ønskede opgaver. Arbejdet har til tider været unødvendig kompliceret grundet misforståelser og mangel på kommunikation. Dette har dog også medført en del læring, både for mig men også for resten af gruppen.

Jeg har i samarbejde med Henrik udviklet det indledende design af al funktionaliteten på Pi, hvilket har været en omfattede opgave i og med at der er mange klasser og tråde der skal holdes styr på. Jeg har i forbindelse med dette fået erfaring med håndteringen af tråde i sekvensdiagrammer anvendelse af en Raspberry Pi og hvordan man bedst beskytter variable i et system hvor mange tråde benytter den samme data. Jeg har ligeledes været til hjælp for gruppen når der skulle implementeres tråde i de enkelte klasser eller når der har været problemer med enten GNU make eller generelt med Pi.


\clearpage