\section{Kristian}

Denne projektgruppe har været en delvis arv af min sidste semesterprojektgruppe hvortil der er kommet tre ''nye'' medlemmer.
Dette har givet en del udfordringer i starten af projektet ift. at få ensrettet måden gruppen arbejdede sammen på.
En af de større udfordringer fra mit synspunkt har været at gruppens torvholder fra sidste semester går på en anden retning og der har derfor været et behov for en der har holdt styr på de administrative opgaver i gruppen og presset gruppen videre i henhold til tidsplanen. 
Det jeg har opnået af erfaring i forbindelse med dette er at en projektgruppe meget stærkt har behov af en der kan bevare det ikke-faglige overblik over gruppen og sørge for at ting skrider fremad på fornuftig vis, hvilket ikke er en rolle jeg selv har siddet med tidligere.

Rent fagligt har jeg opnået erfaringer indenfor lodning af mindre SMD-komponenter, design af kredsløb (dette var for mit vedkommende det første semesterprojekt med ansvar for design og implementering af hardware), samt analyse af et fejlagtigt hardwaremodul i forbindelse med test og udbedring af førnævnte fejl.
Jeg har også lært at anvende GNU Make på fornuftig vis, samt at bruge trådprogrammering i en praktisk sammenhæng.

Jeg har, som i tidligere semesterprojekter, haft det faglige overblik over projektet og på et vist niveau fungeret som en form for faglig vejleder mht. nogle af de øvrige medlemmers problematikker under design- og implementeringsfasen. 
Dette har gjort at jeg har været nødt til at lægge fokus en smule væk fra de opgaver jeg personligt har været ansvarlig for, hvilket direkte afspejles i udførslen af de klasser jeg har skrevet på Pi'en.