\section{Kristian}

Jeg har i dette projekt erfaret vigtigheden af at have en ikke-faglig torvholder, som bevarer overblik over tidsplan, møder etc og selv blevet inspireret til måske at tage denne rolle i en kommende ingeniørgruppe.
Rent fagligt har jeg opnået erfaringer indenfor lodning af små SMD-komponenter, design af kredsløb (dette var for mit vedkommende det første semesterprojekt med ansvar for design og implementering af hardware), samt analyse af et fejlagtigt hardwaremodul i forbindelse med test og udbedring af førnævnte fejl.
Ydermere har jeg lært en masse omkring design af buck convertere og hvilke problemer der opstår ved anvendelse af HF-styresignaler på forsyningselektronik.
Jeg har indset at jeg knapt har rørt overfladen af hvordan man laver en rigtig god switch mode forsyning, men dette har givet inspiration til endnu mere læring indenfor området.
Softwaremæssigt har jeg lært at anvende GNU Make på fornuftig vis, samt at bruge trådprogrammering i en praktisk sammenhæng.