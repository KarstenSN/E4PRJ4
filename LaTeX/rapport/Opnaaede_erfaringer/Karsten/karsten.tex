\section{Karsten}
Som tidligere beskrevet har gruppen været bestående af 4 personer som tidligere arbejdede sammen sidste semester og 3 personer som er gruppens ''nye'' medlemmer. At dette er vigtigt at prioriter skyldes at det har været mærkbart igennem hele samarbejdet. Den ''gamle'' gruppe havde en måde at samarbejde på, hvilket havde forløbet godt for dem sidste semester. Denne arbejdsmetode blev derfor trumfet igennem og de nye måtte derfor acceptere vilkårene. Desværre syntes jeg ikke om denne arbejdsmetode, da den var styret af en eller få personer. En af de store fejl i projektet var at man ikke ville gå i gang med implementeringen, før systemmet var fuldt dokumenteret i designfasen. Dette er selvfølgelig den rigtige metode i følge vores indlæring på 2. semester, men jeg vil betegne den som et solskinsscenarie som ikke kan lade sig gøre i virkeligheden. Da jeg var bekendt med denne fejl fra tidligere semestre, hvor man flere gange har vendt tilbage til designprocessen fra implementeringsprocessen for at rette fejl, var det for mit vedkommende et spørgsmål om at få et ansvarsområde hvor jeg kunne lave en teknologiundersøgelse, samtidig med at jeg kunne lave designet og implementeringen. Jeg gik i gang med at udvikle GUI'en i Qt-creator kort før efterårsferien, hvilket resulterede i at jeg kunne bruge store dele af min fritid på at lave en teknologiundersøgelse, samt implementering, som gjorde designprocessen langt mere præcis og hurtig. Dette ses også ved at jeg var stort set færdig, da størstedelen af gruppen først kom i gang med deres implementering få uger før aflevering. Jeg ser det som et bevis på at førnævnte arbejdsmetode har været en gennemgående fejl. Min erfaring fra dette projekt bliver i overensstemmelse med tidligere: At det er vigtigt at dele ansvarsområder ud til den enkelte, således der ikke kun er få personer som designer hele systemet, at det er vigtigt at lave en teknologiundersøgelse af sit ansvarsområde sideløbende med designprocessen og at det i det hele taget er vigtigt at gruppen samarbejder uden at der er en eller få der tager alle beslutninger.    