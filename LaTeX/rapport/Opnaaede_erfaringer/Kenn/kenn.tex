\section{Kenn H Eskildsen}\label{sec:opnaaede_erfaringer_ke}

Arbejdet med AU2-projektet har for mit vedkomme, udover projektstyring, udelukkende bestået af Software-relateret arbejde i forbindelse med design om implementering af distancesensorerne. 
Dette har været en afveksling fra de tidligere projekter hvor jeg primært har arbejdet på hardware-siden. 
Men efter eget ønske var dette projekt rent software.
I mit arbejde med software i det omfang som projektet har krævet, har givet mig erfaring i brug og programmering af \IIC både på Pi'en og PSoC'en. 
Disse 2 enheder håndterer bussen forskelligt, og dét at skabe et interface hvor disse to kan kommunikere involverede at opsætte PSoC'en som kombineret Master og slave enhed. Derudover gav programmering, debugging og crosskompilering på Pi'en nogle rigtig gode erfaringer i håndtering af Linux systemkald og filhåndtering.
I forbindelse med forløbet af projektet har det været interessant og lærerigt at indgå i en gruppe som, for 3 medlemmers vedkomne er videreført fra sidste semester. 
Dette har givet indsigt i andre arbejdsmetoder og man har skulle omstille sig og tilpasse sig nye arbejdsgange. Derudover har det været meget givtigt med nye inputs og erfaringer. 