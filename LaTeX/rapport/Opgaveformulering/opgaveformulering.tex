\chapter{Opgaveformulering}
\label{ch:Opgaveformulering}

%TODO Er dette afsnit redundant? /KT

OBS: Stjålet fra PRJ3

Herunder er vist en prioritering af funktionaliteter i systemet efter MoSCoW metoden. 

Ambitionen for dette projekt var som absolut minimum, at realisere nedenstående punkter under \textit{"skal"}. 
Det forventedes desuden at punkterne under \textit{"bør"} skulle realiseres, men de har haft lavere prioritet.
Punkterne under \textit{"kan"} forventedes ikke realiseret, og punkterne under \textit{"vil ikke..."} realiseredes med sikkerhed ikke. 
Sidstnævnte punkter kan ses som udviklingsmuligheder i forhold til senere versioner af systemet. 

\begin{itemize}
	\item \textbf{Systemet skal:}
		\begin{itemize}
			\item Kunne monitorere temperaturen i drivhuset og regulere temperaturen i drivhuset vha. varmelegeme, åbning af vinduer og luftcirkulation.
			\item Give brugeren mulighed for at vælge varmelegeme og/eller luftcirkulation fra, hvis en mere økonomisk regulering af temperaturen ønskes. 
			\item Have et grafisk user interface.
		\end{itemize}
	\item \textbf{Systemet bør:}
		\begin{itemize}
			\item Måle jordfugtighed med op til seks sensorer i drivhuset og give brugeren besked på brugerfladen om, at det er tid til at vande. 
			\item Måle lysintensitet og luftfugtighed i drivhuset.
			\item Indeholde en log over alle målte parametre: Jordfugtighed, temperatur, luftfugtighed og lysintensitet. 
			Dataene præsenteres grafisk for brugeren.
			\item Indeholde en database over de mest almindelige drivhusplanter, så brugeren kan orientere sig om en plantes optimale forhold.
			\item Indeholde en systemlog, som noterer vigtige system hændelser.
		\end{itemize}
	\item \textbf{Systemet kan:}
		\begin{itemize}
			\item Sende besked til brugeren via e-mail, om at det er tid til at vande.
			\item Tilkobles et automatisk vandingssystem, som aktiveres ved behov for vanding. 
			\item Give brugeren mulighed for at tilføje planter i databasen.
			\item Give brugeren mulighed for at kommunikere trådløst med systemet fra brugerfladen, så denne kan placeres fx inde i brugerens bolig. 
		\end{itemize}
	\item \textbf{Systemet vil ikke i denne version:}	
		\begin{itemize}
			\item Indeholde et kamera, og tilhørende billedarkiv, som giver brugeren mulighed for at følge planternes udvikling fra dag til dag. 
			\item Give brugeren mulighed for at interagere med systemet via en app på dennes mobiltelefon. 
		\end{itemize}
\end{itemize}

For den fulde tekst se afsnit \ref{P-ch:projektformulering} \nameref{P-ch:projektformulering} på side \pageref{P-ch:projektformulering} i projektdokumentationen.

\clearpage