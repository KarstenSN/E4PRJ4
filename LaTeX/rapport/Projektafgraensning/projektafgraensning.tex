\chapter{Projektafgrænsning} \label{ch:Projektafgraensning}

Systemet AU2 er bygget på en skrabet fjernstyret bil, hvor kun chassiset, hjulene og DC-motoren er tilbage. 
På denne platform er der tilføjet en Raspberry Pi 2 B med kamara og Wi-Fi dongle, et PSoC 4 Pioneer kit, en strømforsyning med batteri, et tachometer-modul, fire afstandssensorer, et accelerometer, et motorstyrings-modul inkl. servomotor. 
Den anden side af projektet omfatter den software der er på PCen og den tilsluttede Xbox-360 controller. 
Selv PCen og Wi-Fi netværket er ikke en del af systemet, men skal være tilgængeligt for at systemet kan fungere. 
En mere detaljeret beskrivelse af bilens opdeling i blokke kan findes i afsnit \ref{P-sec:BDD_for_AU2} \nameref{P-sec:BDD_for_AU2} på side \pageref{P-sec:BDD_for_AU2} i dokumentationen.\\\\


%TODO
\textbf{OBS: Stjålet fra PRJ3}

AutoGreen består af en strømforsyning, en række controllere (tre stk. PSoC 4 Pioneer Kits og et stk. DevKit8000), et varmelegeme (en USB strømspareskinne og en 100W 230V AC glødepære), fire 12V DC ventilatorer og en 12V steppermotor - samt tilhørende mosfet drivere. 
AutoGreen uindeholder desuden sensorer til måling af lufttemperatur, jordfugt, lysintensitet og luftfugtighed.

Selve det fysiske drivhus er således ikke en del af systemet. 
Under udviklingen af denne prototype er anvendt en model af et drivhus på ca. 33 liter, se evt. Figur \ref{P-fig:dimensioner} på side \pageref{P-fig:dimensioner} i Projektdokumenationen. 
Såfremt prototypen skal monteres i et rigtigt drivhus, skal ventilatorer og varmelegeme dimensioneres derefter.

\mbox{}

I de indledende faser af projektarbejdet - Projektformulering, Kravspecifikation, Accepttestspecifikation og Systemarkitektur - omhandler projektdokumentationen det fulde system.
Grundet tidsnød og forskellige komplikationer, er der flere dele af det samlede system, som kun er delvist eller slet ikke implementeret.
Prioriteringen af hvad der skulle skæres væk undervejs har taget udgangspunkt i MoSCoW prioriteringen, se afsnit \ref{ch:Opgaveformulering} \nameref{ch:Opgaveformulering} på side \pageref{ch:Opgaveformulering}.

Alle punkter under "Systemet skal"\ er fuldt implementeret. 

Under "Systemet bør"\ er det første punkt, vedrørende jordfugtsensorer, fuldt implementeret; øvrige punkter er kun delvist eller slet ikke implementeret.

Punkter under "Systemet kan"\ og "Systemet vil ikke i denne version"\ er ikke implementeret. 

Det er med fuldt overlæg at denne løbende prioritering har fundet sted. 
Gruppen ønskede fra starten et system med mange muligheder for udvikling og udbygning; derfor blev der fra begyndelsen beskrevet et system, som var væsentligt mere omfattende, end hvad gruppen regnede med at kunne nå at realisere.

\mbox{}

Den gennemførte acceptest (af hele det beskrevne system) giver detaljeret information om hvad der er realiseret, hvad der er delvist realiseret og hvad der ikke er realiseret, se afsnit \ref{P-ch:Accepttest} \nameref{P-ch:Accepttest} på side \pageref{P-ch:Accepttest} i projektdokumentationen. 


\clearpage