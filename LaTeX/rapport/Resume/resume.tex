\chapter{Resume}
\label{ch:Resume}

Denne rapport omhandler udviklingen af en prototype til et system, der kan installeres i et drivhus. 
Systemet - kaldet AutoGreen - måler lufttemperatur og regulerer denne i drivhuset ved hjælp af åbning og lukning af et vindue, ventilation og et varmelegeme. 
Systemet kan desuden måle jordfugtighed med op til seks jordfugtmålere, og give brugeren besked om, at det er tid til at vande ved en given jordfugtmåler. 

Gennem de første faser af projektarbejdet er der desuden lagt op til måling af luftfugtighed og lysintensitet i drivhuset, logning og grafisk præsentaion af måledata, database med planteinformationer, e-mailnotifikationer med mere. 
Dette er dog ikke implementeret. 

Under udviklingen af prototypen er der anvendt en model af et drivhus på ca. 33 liter. 
Såfremt AutoGreen skulle anvendes i et rigtigt drivhus, skulle aktuatorer - dvs. ventilatorer, varmelegeme og vinduesmotor - skaleres derefter.

AutoGreen's brugerflade og controller er realiseret på et Embest DevKit8000 Evaluation board\cite{lib:DK8000}.
DevKit8000 kommunikerer vha. UART med en \IIC master, der er realiseret på et PSoC 4 Pioneer Kit\cite{lib:psoc4_guide}. 
Masterenheden kommnunikerer med flere \IIC slaver, der er koblet til hhv. aktuatorer og analoge sensorer. 
To af disse disse \IIC slaver er ligeledes realiseret på et PSoC 4 Pioneer Kit. 
Måling af temperatur i drivhuset sker vha. en sensor - LM75 - med \IIC interface\cite{lib:LM75}. 

Det realiserede system kan måle temperaturen i drivhuset med en præsision på +/- 0.5 $^{\circ}$C. 
Systemet kan - i området op til 10 $^{\circ}$C over den omgivende temperatur - regulere temperaturen med en præcision på +/- 1 grad. 
Måling af jordfugt fungerer ligeledes på det realiserede system. 
I tilfælde af manglende vand ved en sensor, gives der besked om dette på brugerfladen, og en port på en af systemets \IIC slaver går fra logisk lav til logisk høj.

Dette åbner op for muligheden for tilkobling af et vandingssystem, men der er som nævnt flere muligheder for videreudvikling og udbygning af systemet. Se afsnit \ref{P-ch:Projektformulering} \nameref{P-ch:Projektformulering} på side \pageref{P-ch:Projektformulering} i projektdokumentationen for yderligere information om systemet. 

Den største udfordring i det realiserede system ligger i UART kommunikationen mellem DevKit8000 og \IIC master. 
Der anvendes en simplificeret UART kommunikation - kun med Tx, Rx og reference - hvorpå der er en del fejlkommunikation. 
Dette opleves ikke, hvis man fx. kobler en UART terminal direkte på \IIC masteren. 
Se afsnit \ref{P-ch:Accepttest} \nameref{P-ch:Accepttest} på side \pageref{P-ch:Accepttest} i projektdokumentationen for yderligere information om resultatet.

\clearpage