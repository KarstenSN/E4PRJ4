\chapter{Resume}
\label{ch:Resume}

Denne rapport beskriver udviklingsprocessen for det fjerde semesterprojekt for elektro linjen på ingeniørhøjskolen i Århus. Projektet omhandler en intelligent bil kaldet AU2 som er en legetøjsbil der kan styres fra en almindelig computer med Windows installeret. Den intelligente del består af at bilen selv er i stand til at måle hvornår den er på vej mod en forhindring og derved selv undvige en kollision, på trods af brugerens styreinput. Brugeren er i stand til at se hvor bilen befinder sig via et kamera som er monteret på bilen, samt se bilens aktuelle hastighed mv. i et program installeret på brugerens computer. Igennem hele projektet har der været holdt mange gruppemøder hvor systemarkitekturen er blevet designet step for step. Det var meningen af systemarkitekturen skulle være færdige før designprocessen gik i gang. Op til den sidste halvdel af projektet blev det dog klart for gruppen at tiden var ved at løbe ud og der blev derfor arbejdet hårdt med implementeringen op mod aflevering. I hele projektet har der været stor fokus på at få alle med i gruppen, således alle havde overblik over de enkelte step. Dette viste sig at være svært at bibeholde i implementeringsprocessen hvilket endte i at ikke alle dage var lige produktive. I den sidste uge af projektet blev det klart at det ikke var alle krav til bilen som ville blive opfyldt. Dette skyldes at der var problemer med kommunikationen mellem bil og Pc samt softwaren på bilen havde taget længere tid at implementeret end forventet. Endvidere var der mindre problemer med hardwaren som ikke blev løst inden aflevering.  
\clearpage