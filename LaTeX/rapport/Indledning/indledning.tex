\chapter{Indledning}
\label{ch:Indledning}

Denne rapport indeholder beskrivelsen af udvikling og design for 4. semesterprojektet, AU2, samt opbygningen af en prototype. 
Rapporten er opsummeringen af projektdokumentationen, og dækker hovedsageligt over udviklingsmæssige overvejelser/beslutninger som gruppen har foretaget undervejs i projektperioden. 
Bilen gør det muligt for børn og voksne at underholdes uden risiko for at ødelægge bilen, eftersom den selv forhindrer en forestående kollision.

AU2 er en intelligent bil, som vha. et anti-kollisionssystem (AKS) kan undvige en detekteret forhindring, og desuden styres med en X-Box Controller over et trådløst Wi-Fi netværk. Bilen kan bremse, dreje, køre frem og tilbage.
I forbindelse med en nærtstående kollision, vil AKS overtage styringen fra brugeren, og dreje bilen i en anden mindre farlig retning. 
Projektet består af en PC med tilhørende X-Box Controller, en Raspberry Pi med tilhørende Wi-Fi Dongle og en PSoC. 

Brugeren kan se hvor bilen kører vha. det påmonterede kamera, hvorfor det er muligt at styre bilen fra et punkt hvor bilen ikke er synlig.
GUI på PC er lavet i QT og brugeren ser data om bilen i sammnhæng med videoen, der streames fra bilen.

\section{Læsevejledning}
Projektdokumentationen til denne rapport er skrevet kronologisk i forhold til de givne faser i ASE Modellen\cite{lib:vejledning}, pånær accepttesten, som er udarbejdet i forlængelse af kravspecifikationen, men selve testen er udført i slutningen af forløbet, deraf placeringen sidst i dokumentet.
Den samme rækkefølge er ført i denne rapport under kapitel \ref{ch:Projektbeskrivelse} \nameref{ch:Projektbeskrivelse}.

Da projektet er blevet delt op i enkelte ansvarsområder efter fuldendt systemarkitektur, indeholder rapporten samt dokumentationen både afsnit skrevet at den samlede gruppe samt individuelle afsnit.
Bemærk at der på side '\pageref{ch:arbejdsopgaver}' er vist en tabel med arbejdsopgaver, som beskriver hvad de enkelte gruppemedlemmers ansvarsområder har været i projektet. 
Dette betyder dog ikke at det nødvendigvis kun er markerede personer der har skrevet i de givne afsnit i rapporten, men at disse personer er ansvarlige for den del af projektet i sin helhed.
I starten af hvert kapitel i projektdokumentationen, er der tilknyttet en versionshistorik, som har påført initialer for hvem der har indført materialet, ikke nødvendigvis hvem der har lavet det.

\clearpage

\section{Ordforklaring}

\begin{table}[h]
\centering
\begin{tabularx}{\textwidth*9/10}{| l | Z |}
% header ------------------------
\hline
\textbf{Begreb} & \textbf{Forklaring} \\\hline
% header ------------------------
	At lave noget fornuftigt & 
	Sidde på en stol med en kop kaffe og stirre ud af vinduet. \\\hline
	Skænderi &
Når kaffekanden er tom \\\hline
	Systemlog &
Systemet er udstyret med en log over hvad systemet foretager sig. Dette kunne f.eks. være et indlæg når systemet foretager en måling, sender en e-mail og regulerer miljøet i drivhuset. \\\hline
	Virtuelt Drivhus &
Det virtuelle drivhus er systemets repræsentation af det fysiske drivhus. Brugeren kan tilføje planter fra plantedatabasen i det virtuelle drivhus, og på den måde give systemet indirekte oplysninger om ønskede parametre. Disse informationer lagres i systemets konfigurationsfil. \\\hline
	Fysisk Drivhus &
Ved det fysiske drivhus forstås det drivhus, hvori systemet er monteret. \\\hline
	Konfigurationsfil &
Dette er en klasse, der er placeret på DevKit8000, som indeholder brugerens konfigurationer om blandt andet notifikationer, e-mailadresser, antallet af fugtsensorer og deres unikke ID mm. \\\hline
	Notifikations e-mail &
Dette er en daglig e-mail, som brugeren kan vælge at få tilsendt. Den sendes klokken 12:00, og indeholder informationer om parametrene i det fysiske drivhus. \\\hline
	Advarsels e-mail &
Dette er en e-mail, som brugeren kan vælge at få tilsendt. Den sendes, hvis en parameter i det fysiske drivhus kommer uden for tolerancen af den ønskede værdi. \\\hline
\end{tabularx}
\end{table}
%TODO lave ordforklaring
\clearpage