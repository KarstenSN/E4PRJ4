\chapter{Indledning}
\label{ch:Indledning}
Rapporten er en opsummeringen af projektdokumentationen og dækker hovedsageligt over udviklingsmæssige overvejelser og beslutninger som gruppen har foretaget undervejs i projektarbejdet. 
Rapporten indeholder beskrivelse af udviklingen og design af prototypen til en fjernstyret bil, AU2.

AU2 kan via et antikollisionssystem (forkortet til AKS) undvige en mulig forhindring og derved sikre at brugeren ikke ødelægger hverken bil eller andre genstande. 
Bilen er tænkt til at være et stykke legetøj, som kan bruges af både voksne og børn. 
Bilen styres med en XBox360-controller forbundet til en PC, som er tilsluttet samme netværk som bilen, hvorved brugeren også kan modtage et video-stream fra bilen.


\section{Læsevejledning}
Projektdokumentationen til denne rapport er skrevet kronologisk i forhold til de givne faser i ASE Modellen\cite{lib:vejledning}, på nær accepttestspecifikationen, som er udarbejdet i forlængelse af kravspecifikationen. 
Selve accepttesten er dog udført i slutningen af forløbet, deraf placeringen sidst i dokumentet.
Den samme rækkefølge er ført i denne rapport under kapitel \ref{ch:Projektbeskrivelse} \nameref{ch:Projektbeskrivelse}.

Da projektet er blevet delt op i enkelte ansvarsområder efter fuldendt systemarkitektur, indeholder rapporten samt dokumentationen både afsnit skrevet at den samlede gruppe samt individuelle afsnit.
Bemærk at der på side '\pageref{ch:arbejdsopgaver}' er vist en tabel med arbejdsopgaver, som beskriver hvad de enkelte gruppemedlemmers ansvarsområder har været i projektet. 
Dette betyder dog ikke at det nødvendigvis kun er markerede personer der har skrevet i de givne afsnit i rapporten, men at disse personer er ansvarlige for den del af projektet i sin helhed.
I starten af hvert kapitel i projektdokumentationen, er der tilknyttet en versionshistorik, som har påført initialer for hvem der har indført materialet, ikke nødvendigvis hvem der har lavet det.

\clearpage

\section{Ordforklaring}

\begin{table}[h]
\centering
\begin{tabularx}{\textwidth*9/10}{| l | Z |}
% header ------------------------
\hline
\textbf{Begreb} & \textbf{Forklaring} \\\hline
% header ------------------------

	AU2 &
Navnet på hele det samlede system. Skal udtales ''auto''. \\\hline
	AKS & 
	En forkortelse af anti-kollisionssystem. Systemet overtager styringen fra brugeren hvis der er risiko for en kollision. \\\hline

	GUI &
En forkortelse for ''graphical user interface'' og kan oversættes til grafisk brugerflade. GUI'en i dette system er bestående af det program som brugeren kører på sin PC. \\\hline

	Pi &
En forkortelse af Raspberry Pi 2 B \cite{lib:rpi}. Dette er ''hjernen'' på bilen, altså den primære styreenhed/controller.  \\\hline

	PSoC &
En forkortelse for PSoC 4 Pioneer Kit \cite{lib:psoc4_guide}. Er anvendt som controller til bilens \IIC sensorer.\\\hline

	GPIO &
En forkortelse for ''General-purpose input/output''. Anvendes i forbindelse med tilslutningerne i form af harwin-pins der er at finde på Pi'en og PSoC'en.  \\\hline

	API &
En forkortelse for ''Application programming interface''. Dette begreb anvendes om grænsefladen til et stykke software.\\\hline

	Xbox-360 controller  & Dette er en ''controller'' som anvendes til spillekonsollen Xbox-360. Knapperne er benævnt med følgende forkortelser: 
\begin{packed_item}
	\item RT - Right trigger
	\item LT - Left trigger 
	\item LS - Left stick (Venstre styrepind)
	\item X  - X-knap
\end{packed_item}

Et billede med forklaringer kan ses på figur \ref{P-fig:xboxcontroller} i afsnit \ref{P-sec:ordforklaring} \nameref{P-sec:ordforklaring} på side \pageref{P-sec:ordforklaring} i dokumentationen.\\\hline

\end{tabularx}
\end{table}
%TODO Skal der tilføjes flere ting?
\clearpage