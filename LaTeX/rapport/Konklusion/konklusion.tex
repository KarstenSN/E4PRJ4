\chapter{Konklusion}
\label{ch:Konklusion}

Jf. kapitel \myRef{P-ch:Accepttest} i dokumentationen ses det, at der fortsat er en del at arbejde på før der kan udgives en færdig prototype. 
Dette betyder ikke at projektet i sidste ende har været en fiasko, men tværtimod at gruppen er kommet langt med de punkter der er realiseret, og uden tvivl ville kunne løse de manglende godkendelser ved videre arbejde. 
Projektet har været meget lærerigt og gruppen har oplevet en stor success med sammensætning af de forskellige dele af projektet, på en måde der viser at alles arbejde har været en stor fremdrift for projektet. 
Kommunikationsvejene fra bruger til sensor er implementeret, men giver sporadiske udfald. De væsentligste ting på projektet er oppe at køre, men mangler dog at blive finjusteret og fejlsøgt. 

Gennem samarbejde og et gennemgående engagement har gruppen formået at planlægge et realistisk projekt fra start og ende ud med en prototype. Selvom projektet har været plaget af forhindringer, som har ledt til mange ændringer undervejs. 
Flere af punkterne i acceptesten, som ikke er bestået, er ikke godkendt pga. manglende implementering. 
For de udeladte elementer i projektet, er der foretaget undersøgelser, samt lavet forberedelser til at kunne implementere på et senere tidspunkt. 
Blandt de ting som mangler at blive implementeret er bl.a. accelerometeret, drejemekanisme, optimal AKS funktionalitet og motorprint. 
Førnævnte er enten skåret fra projektet eller udeladt af hensyn til tidsplanen. Gruppen har fra start været enige om at det er bedre at skære noget fra projektet end at stå med for lidt at lave. 
Dog erkendes det at fraskæringerne af de forskellige dele er sket for sent, ift. hvad der havde været optimalt for projektet.

Blandt fremtidigt arbejde er de største udviklingspotentialer i drejemekanismen/motorprintet og AKS-klassen, således bilen får mulighed for at dreje og rent faktisk kan undvige en forhindring. 
Der kan læses mere om problemerne med disse i hhv. afsnit \myRef{P-sec:hwi_servo}, \myRef{P-sec:hwi_motor_driver} og \myRef{P-sec:aks_impl} i dokumentationen. 
Der er fuldtalligt enighed i gruppen om at det udelukkende skyldes et tidsmæssigt problem, som vil kunne løses ved længere tid til projektet.

Projektgruppen er resultatet af en kombinering af to forskellige grupper, som hver især har gjort tingene på hver deres måde. Det har været en udfordring at skulle tilpasse arbejdsmetoden bedst muligt, men alle har været villige til at give sig en smule. Dog har der, til trods for den nydannede gruppe, været et godt sammenhold og fint drive gennem hele projektperioden.

I starten af projektforløbet har gruppen været meget tilfredse med projektet og dets omfang, men i takt med forløbet, er det gået op for gruppen hvor omfattende hele projektet har været. 
Det har resulteret i en gruppe som nu er stærkt bundet sammen i et samarbejde, som har formået at løfte hele projektet og gjort det muligt at komme så langt. Gruppen har udviklet sig meget, både fagligt og socialt og alle har fået rykket deres kompetencer indenfor gruppearbejde endnu en gang.

Det realiserede system er så godt som gruppen har kunnet formå at gøre det, og med det store omfang og korte tid er gruppen enige om at det er gået bedre end forventet. 
Tidsplanen er blevet holdt nogenlunde fast, med undtagelse af implementerings og testperioden, som har skredet en del over planlagt tid. Til sidst kan gruppen konkludere at E4PRJ4 har været et særdeles udfordrende projekt, men samtidig lærerigt og underholdende på en helt speciel måde.

\clearpage