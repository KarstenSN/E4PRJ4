\chapter{Konklusion}
\label{ch:Konklusion}

Jf. kapitel \myRef{P-ch:accepttest} i dokumentationen ses det, at der fortsat er en del at arbejde på før der kan udgives en færdig prototype. 
Dette betyder ikke at projektet i sidste ende har været en fiasko, men tværtimod at gruppen er kommet langt med de punkter der er realiseret, og uden tvivl ville kunne løse de manglende godkendelser ved videre arbejde. 
Projektet har været meget lærerigt og gruppen har oplevet en stor success med sammensætning af de forskellige dele af projektet, på en måde der viser at alles arbejde har været en stor fremdrift for projektet. 
Kommunikationsvejene fra bruger til sensor er implementeret fuldt ud og virker som de skal, og de væsentligste ting på projektet er oppe at køre, men mangler dog at blive finjusteret og fejlsøgt. 

Gennem fantastisk samarbejde og et gennemgående engagement har gruppen formået at planlægge et realistisk projekt fra start og ende ud med en prototype, selvom projektet har været plaget af forhindringer, som har ledt til mange ændringer undervejs. 
Flere af punkterne i acceptesten, som ikke er bestået, er ikke godkendt pga. manglende implementering. 
For de udeladte elementer i projektet, er der foretaget undersøgelser, samt lavet forberedelser til at kunne implementere på et senere tidspunkt. 
Blandt de ting som mangler at blive implementeret er bl.a. accelerometeret, drejemekanisme, optimal AKS funktionalitet og motorprint. 
Førnævnte er enten skåret fra projektet eller udeladt af hensyn til tidsplaen. Gruppen har fra start været enige om at det er bedre at skære noget fra projektet end at stå med for lidt at lave. 
Dog erkendes det at fraskæringerne af de forskellige dele er sket for sent, ift. hvad der havde været optimalt for projektet.

Blandt fremtidigt arbejde er de største udviklingspotentialer i drejemekanismen/motorprintet \myRef{sec:hwi_motor_styring} og AKS-klassen, således bilen får mulighed for at dreje og rent faktisk kan undvige en forhindring.


%TODO
%Det implementerede systems største udviklingspotentiale ligger i UART kommunikationen mellem DevKit8000 og PSoC Mater, se \ref{ch:Resultater_og_diskussion} \nameref{ch:Resultater_og_diskussion} på side \pageref{ch:Resultater_og_diskussion} for nærmere diskussion af dette. 
%
%I de sidste faser af projektarbejdet er der i gruppen internt blevet talt en del om, at det nærmest er ærgerligt at forløbet er slut. 
%Der er mange funktionaliteter som kunne færdigimplementeres og/eller optimeres, hvis der havde været mere tid til rådighed. 
%
%\mbox{}
%
%Gruppen har undervejs været meget tilfreds med projektarbejdets forløb; de erfaringer gruppen gjorde sig på sidste semester har gavnet forløbet, og gruppen har formået at udvikle sig yderligere. 
%Der er ingen tvivl om at der er store fordele ved at arbejde sammen i en "gammel"\ gruppe, der ikke først skal til at lære hinanden at kende både socialt og fagligt. 
%
%Gruppen er meget tilfreds med det realiserede system, men der er bred enighed om at gruppens største styrke ligger i planlægning og koordinering af arbejdet. 
%Der blev - som på sidste semester - lagt en stram tidsplan fra start; der var lagt op til en periode på tre uger til skrivning af denne rapport. 
%Tidsplanen kom - som forventet - til at skride undervejs, men gruppen som helhed har undervejs formået at have overblik over arbejdet og rette i tidsplanen og kravene for projektet. 
%Derved har vi undgået at skulle lave makværk og lappeløsninger i slutningen af forløbet. 

\clearpage