\chapter{Konklusion}
\label{ch:Konklusion}

Jf. kapitel \myRef{P-ch:Accepttest} i dokumentationen ses det, at der fortsat er arbejde at udføre henimod udgivelse af en færdig prototype.
Projektet har været meget lærerigt og gruppen har oplevet en stor succes med sammensætning af de forskellige moduler af projektet. 
Kernen i denne arbejdsgang er at hver enkelt gruppemedlems arbejde har medvirket til en stor fremdrift for projektet. 
Kommunikationsvejene fra bruger til sensor er implementeret, men giver dog sporadiske udfald. 
De væsentligste ting på projektet er implementeret og fungerer som forventet, dog mangler der enkelte steder at blive finjusteret og fejlsøgt. 

Gennem samarbejde og et gennemgående engagement, har gruppen formået at planlægge et realistisk projekt fra start og ende ud med en fungerende prototype. 
Der er løbende i projektfasen opstået flere udfordringer som har ledt til  designændringer undervejs. 
Flere af de ikke-beståede punkter i accepttesten, er ikke godkendt pga. manglende implementering. 
For de udeladte elementer i projektet er der foretaget undersøgelser og forberedelser til at kunne implementere disse evt. som del af fremtidig arbejde. 
Blandt de elementer der mangler at blive implementeret er bla. accelerometeret, servomotor, fuld AKS-funktionalitet samt motorprint. 
Disse er enten skåret fra projektet, eller udeladt af hensyn til tidsplanen. 
Gruppen har fra projektstart været enige om at det er bedre at skære noget fra projektet i sidste ende, end at begrænse projektet for meget. 
Dog erkendes det at valget at skære de nævnte elementer fra er sket for sent, ift. hvad der havde været optimalt for projektet.

Blandt fremtidigt arbejde er de største udviklingspotentialer i servomootoren, motorprintet og AKS-klassen, således bilen får mulighed for at dreje og rent faktisk undvige en forhindring. 
Der kan læses mere om udfordringerne med disse moduler hhv. i afsnit \myRef{P-sec:hwi_servo}, \myRef{P-sec:hwi_motor_driver} og \myRef{P-sec:aks_impl} i dokumentationen. 
Der er i gruppen samlet enighed om at nedskæringerne udelukkende er sket på baggrund af tidsmæssige udfordringer.

Projektgruppen er kombineringer af to forskellige tidligere grupper der hver især har haft deres arbejdsmetoder.
en vigtig del af projektet har været at  tilpasse arbejdsmetoderne bedst muligt, og hermed opnå et optimalt samarbejde hvor der kunne trækkes på andre erfaringer. 
I forbindelse med dette har der været en naturlig indkøringsperiode for den nydannede gruppe.

Forløbet har resulteret i en gruppe som nu er stærkt bundet sammen i et samarbejde, som har formået at løfte hele projektet og gjort det muligt at komme så langt. 
Gruppen har været i konstant udvikling både fagligt og socialt og alle har fået rykket deres kompetencer indenfor gruppearbejde endnu en gang.

Det realiserede system er så godt som gruppen har kunnet formå at gøre det, og med det store omfang og korte tid er gruppen enige om at det er gået bedre end forventet. 
Tidsplanen er blevet holdt nogenlunde fast, med undtagelse af implementerings og testperioden, som gik lidt over planlagt tid. 
Til sidst kan gruppen konkludere at E4PRJ4 har været et udfordrende og lærerigt projekt, fagligt som socialt.

\clearpage