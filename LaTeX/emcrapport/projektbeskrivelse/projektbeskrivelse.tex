\chapter{Projektbeskrivelse}

Projektet omhandler design og implementering af en fjernstyret legetøjsbil. 
Bilen skal kunne fjernstyres via en Xbox 360 Controller koblet til et PC, som via WiFi kommunikerer med en central computer, i dette tilfælde en Raspberry Pi, på bilen.
Denne sørger for at styre bilen og udføre ''smarte'' funktioner, som indebærer et kamera, et anti-kollisionssystem samt mulighed for visning af data på PC'en omkring bilens fart, g-påvirkning mm.

Raspberry Pi'en styrer selve bilen ved hjælp af nogle sensorer, deriblandt et eget designet tachometer, en motor og en servomotor til at styre hjulene.
Alt dette er forsynet af bilens strømforsyning, som består af en buck converter, der leverer 3V og 5V forsyning samt en direkte adgang til bilens batteri.
I forbindelse med denne rapport er der lagt fokus på bilens strømforsyning, tachometer og motor, da disse arbejder med hhv. høje frekvenser, strømme og følsomt måling af bilens hastighed.