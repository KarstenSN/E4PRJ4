\chapter{Projektformulering} \label{ch:projektformulering}

%---------------------------------------------------------------------------------------
%											PROJEKTBESKRIVELSE
%---------------------------------------------------------------------------------------

\section{Beskrivelse} \label{sec:beskrivelse}
Som en vigtig del af børns opvækst indgår leg. Det er meget vigtigt for deres fremtid fordi det gør børnene sociale, robuste, kreative og ikke mindst nysgerrige. Med til at skabe rammerne for børns leg er legetøj, og i dag er det vigtigt at børn har mulighed for at anvende den teknologi der er til rådighed i dagens Danmark. Dette bekræfter Niklas Alexander Chimirri, der er forsker inden for områder som barndom, psykologi og teknologi, i en artikel der er udgivet på Roskilde Universitets hjemmeside i november 2014. 
%TODO Indsæt kildehenvisning: https://www.ruc.dk/om-universitetet/nyhedsportal/rubrik-forskningsmagasin/rubrik-nr-6/voksne-skal-forstaa-boerns-leg-med-teknik/

Projektgruppen vil gerne bidrage til eller i det mindste sætte fokus på, at det er vigtigt at børn har muligheden for at lege... og gerne med moderne teknologi. Derfor vil projektgruppen lave en fjernstyret bil. Det skal ikke være en almindelig fjernstyret bil - den skal være intelligent og får derfor navnet ’’AU2’’.

Den intelligente del består af sensorer samt en kommunikationsenheder, som gør det muligt at styre bilen over et trådløst netværk. Brugeren har hermed mulighed at navigere bilen ved at betragte en computerskærm, der viser et live-stream med video fra et kamera monteret på bilen. Er bilen inden for synsfeltet kan den selvfølgelig også styres ved at se direkte på den. For at undvige forhindringer på kørebanen, implementeres et anti-kollisionssystem bestående af afstandssensorer på bilen. Således kan bilen selv standse eller undvige, hvis bilen nærmer sig en forhindring hastigt. Anti-kollisionssystemet vil altså forhindre at den fjernstyrede bil går i stykker ved kollison eller at brugeren kommer til at skade de omgivelser omgivelserne med den.
\clearpage

%---------------------------------------------------------------------------------------
%										MOSCOW PRIORITERING	
%---------------------------------------------------------------------------------------

\section*{MoSCoW prioritering} \label{sec:moscow_prioritering}

Ambitionen for dette projekt er som absolut minimum at realisere nedenstående punkter under \textit{''skal''}. 
Det forventes desuden at punkterne under \textit{''bør''} realiseres, men de har lavere prioritet.
Punkterne under \textit{''kan''} forventes ikke realiseret, og punkterne under \textit{''vil ikke...''} realiseres med sikkerhed ikke. 
Sidstnævnte punkter kan ses som udviklingsmuligheder i forhold til senere versioner af systemet. 

\begin{itemize}
	\item \textbf{Skal:}
		\begin{itemize}
			\item Bilen skal være i stand til at køre frem og tilbage.
			\item Bilen skal være i stand til at dreje.
			\item Systemet skal monitorere hastighed og acceleration med sensorer.
			\item Systemet skal regulere hastigheden med feedback fra sensorer.
			\item Systemet skal regulere drejemekanismen med feedback fra sensorer.
			\item Systemet skal kommunikere internt mellem Bil og PC over Wi-Fi.
			\item Systemet skal undvige identificerede forhindringer. 
			\item Bilen skal styres vha. en XBOX-360 controller tilkoblet PC.
			\item Systemet skal sende video live-feed fra bilen til PC.
		\end{itemize}
	\item \textbf{Bør:}
		\begin{itemize}
			\item Systemet bør ikke have større forsinkelse på kommunikation end 50ms fra indtastning til reaktion.
			\item Systemet bør 
		\end{itemize}
	\item \textbf{Kan:}
		\begin{itemize}
			\item Systemet kan indeholde en batteriniveau indikator. %TODO NEJ! -Philip :)
		\end{itemize}
	\item \textbf{Vil ikke i denne version:}	
		\begin{itemize}
			\item Bum4 %TODO Skal vist laves færdig
		\end{itemize}
\end{itemize}


%TODO nedenstående inkorporeres i MOSCOW
På punkt opstilling ser funktionerne således ud:

\begin{itemize}\itemsep1pt \parskip0pt \parsep0pt
	\item Netværksbaseret fjernstyring (Wi-Fi) giver mulighed for kontrol over større afstande og fleksibilitet i form af Human Interface Devices (HID).
	\item Anti-kollisionssystemet består af automatiseret styringshjælp i form af
afstandsmåling, regulering af hastighed og retning.
	\item Kamera til live-feed video.
\end{itemize}

Som sensorer anvendes:

\begin{itemize}\itemsep1pt \parskip0pt \parsep0pt
	\item Fire ultralydssensorer til afstandsmåling.
	\item Et accelerometer til g-måling.
	\item Et tachometer til hastighedsmåling.
	\item Et kamera til live-feed.
	\item En XBOX-360 controller til styring af bilen.
\end{itemize}

Som aktuatorer anvendes:

\begin{itemize}\itemsep1pt \parskip0pt \parsep0pt
	\item En DC-motor til fremdrift.
	\item En servo-motor til styring af retning.
\end{itemize}

Målet med projektet vil være at implementere  kollisionssystemet, motorstyringen og -regulering, drejemekanismen samt den trådløse kommunikation til bilen.