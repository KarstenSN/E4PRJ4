%%%%%%%%%%%%%%%%%%%%%%%%%%%%%%%%%%%%%%%%%%%%%%%%%%%%%%%%%%%%%%%%%
%																%
%      ++++++++++++ Design af Afstandssensor ++++++++++++++		%
%																%
%%%%%%%%%%%%%%%%%%%%%%%%%%%%%%%%%%%%%%%%%%%%%%%%%%%%%%%%%%%%%%%%%

Afstandssensorene leveres formonteret på chip hvor benene fra IC'en er trukket til harwinpins som let kan tilgås. Der benyttes I2C-protokol til kommunikation med Pi'en. Ved kommunikation benyttes følgende pins: 

\begin{itemize}
	\item pin 1: Temporary Default
	\item pin 2: Address Announce / Status
	\item pin 3: Benyttes ikke
	\item pin 4: SDA: Data
	\item pin 5: SCL: Clock
	\item pin 6: GND: Reference
	\item pin 7: VCC: Forsyning
\end{itemize}

\noindent
\textit{SDA}-linjen kommunikerer data ud med reference til GND.
\noindent
\textit{SCL}-linjen sørger for at holde timing. 
\noindent
\textit{AA/Status} benyttes til at angive om sensoren er i gang med at foretage en range- reading. Status-pin'en holdes højt så længe sensoren scanner, og trækkes lavt når operationen er fuldført. Dette angiver at data er klar til levering. Når status-pin'en er høj ignoreres al I2C-kommunikation, således at sensoren kan arbejde uforstyrret. 
\noindent
\textit{Temp-adresse}-pin'en benyttes til at initiere sensoren med en ønsket adresse, denne pin sætte høj ved power-up, kan en brugerdefineret adresse sendes til sensoren, i dette tilfælde benyttes unikke 8-bit adresser til sensorerne.

Herefter skrives en klasse \texttt{distanceSensor} der kan håndtere de ønskede kald til de 4 afstandssensorer. 
Denne klasse skal indeholde funktion til at tilgå sensorerne. Klassen initieres med constructoren hvori der åbne for I2C-devicet, samt at der sætte den foromtalte adresse. Detudover skal klassen indeholde metoden: \texttt{getDistance()}, i denne funktion foretages et read-kald med den pågældende sensors adresse. 
Følgende kommander benyttes for at kommunikere med en med en given adresse: 











\clearpage