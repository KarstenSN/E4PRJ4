
\subsubsection{Boundary-klasse: Steering}

\begin{figure}[h]
\centering
\includegraphics[]{../fig/diagrammer/bil/cd_steering.pdf}
\caption{Klassebeskrivelse af boundary-klassen Sterering}
\label{fig:cd_Sterering}
\end{figure}

\textbf{Attributter}

\begin{table}[H]
\begin{tabularx}{\textwidth}{| l | l | Z |} \hline
Navn & Type & Beskrivelse \\\hline
\texttt{motorPWMOutValue\_} & \texttt{int} &Variabel der indeholder værdien den skal sendes til PWM hardware på Pi \\\hline
\texttt{direction\_} & \texttt{bool} &Variabel der indeholder værdien om bilen skal kører frem eller tilbage.\\\hline
\texttt{minServoPWM\_} & \texttt{int} &Minimum værdi for softwarePWM.\\\hline
\texttt{maxServoPWM\_} & \texttt{int} &Maximum værdi for softwarePWM.\\\hline
\texttt{dataClassPtr\_} & \texttt{Data*} &En pointer til det objekt af typen Data der ønskes læst og skrevet data til.\\\hline
\texttt{SettingsPtr\_} & \texttt{Settings*} &En pointer til det objekt af typen Settings der ønskes skrevet data til.\\\hline
\texttt{logPtr\_} & \texttt{Log*} &En pointer til et object af typen Log. Gør det muligt at kalde funktioner fra Log objektet.\\\hline
\texttt{dState\_} & \texttt{double } &Gemmer sidste motorPWMOutValue værdi til næste udregning .\\\hline
\texttt{iState\_} & \texttt{double} &Gemmer error værdier til udregning af integral ledet i PID regulering.\\\hline
\texttt{iMax\_} & \texttt{double } &Max værdi af integral ledet.\\\hline
\texttt{iMin\_} & \texttt{double } &Min værdi af integral ledet.\\\hline
\texttt{iGain\_} & \texttt{double } &Integral gain.\\\hline
\texttt{pGain\_} & \texttt{double } &Proportional gain.\\\hline
\texttt{dGain\_} & \texttt{double } &Derivative gain.\\\hline
\texttt{error\_} & \texttt{double } &Fejl værdien mellem ønsket hastighed og aktuel hastighed.\\\hline
\texttt{pTemp\_} & \texttt{double } &Værdi for udregnet proportional værdi.\\\hline
\texttt{dTemp\_} & \texttt{double } &Værdi for udregnet derivative værdi.\\\hline
\texttt{iTemp\_} & \texttt{double } &Værdi for udregnet integral værdi.\\\hline


\end{tabularx}
\caption{Attributter for klassen Steering}
\label{table:attr_steering}
\end{table}

\newpage
\textbf{Metoder} 

%TODO fix position

\begin{table}[H]
\begin{tabularx}{\textwidth}{| L{2.5 cm} | Z |} \hline
Prototype & \texttt{int userInput(unsigned char speedForward, unsigned char speedBackward, 
	char turn, char brake)} \\\hline
Parametre & \texttt{speedForward} \newline Den ønskede hastighed fremad. 0..255\newline
		\texttt{speedBackward} \newline Den ønskede hastighed bagud. 0..255\newline
		\texttt{turn} \newline Den ønskede drejning på forhjul. -128..127\newline
		\texttt{brake} \newline Der skal bremses. 0..1\newline 
 \\\hline
Returværdi &  \texttt{int} \newline 1 hvis alle operationer er gået okay ellers -1. \\\hline
Beskrivelse & Metoden sørger for at opdaterer retning og hastighed på bilen. \\\hline
\end{tabularx}
\caption{Metodebeskrivelse for \texttt{userInput}}
\label{table:met_userInput}
\end{table}



\begin{table}[H]
	\begin{tabularx}{\textwidth}{| L{2.5 cm} | Z |} \hline
		Prototype & \texttt{void PWMUpdate( )} \\\hline
		Parametre &   
		\\\hline
		Returværdi &  \texttt{int} \newline 1 hvis alle operationer er gået okay ellers -1. \\\hline
		Beskrivelse & Metoden sørger for at opdaterer PWM til hastighed på bilen. \\\hline
	\end{tabularx}
	\caption{Metodebeskrivelse for \texttt{PWMUpdate}}
	\label{table:met_PWMUpdate}
\end{table}

\begin{table}[H]
	\begin{tabularx}{\textwidth}{| L{2.5 cm} | Z |} \hline
		Prototype & \texttt{int brake( )} \\\hline
		Parametre & 
		\\\hline
		Returværdi &  \texttt{int} \newline 1 hvis alle operationer er gået okay ellers -1. \\\hline
		Beskrivelse & Metoden sørger for at opdaterer retning og hastighed på bilen. \\\hline
	\end{tabularx}
	\caption{Metodebeskrivelse for \texttt{brake}}
	\label{table:met_brake}
\end{table}

\begin{table}[H]
	\begin{tabularx}{\textwidth}{| L{2.5 cm} | Z |} \hline
		Prototype & \texttt{int softbrake( )} \\\hline
		Parametre & 
		\\\hline
		Returværdi &  \texttt{int} \newline 1 hvis alle operationer er gået okay ellers -1. \\\hline
		Beskrivelse & Metoden sørger for at opdaterer retning og hastighed på bilen. \\\hline
	\end{tabularx}
	\caption{Metodebeskrivelse for \texttt{softbrake}}
	\label{table:met_softbrake}
\end{table}

\begin{table}[H]
	\begin{tabularx}{\textwidth}{| L{2.5 cm} | Z |} \hline
		Prototype & \texttt{int turn(signed char value)} \\\hline
		Parametre & \texttt{value} \newline
		\\\hline
		Returværdi &  \texttt{int} \newline 1 hvis alle operationer er gået okay ellers -1. \\\hline
		Beskrivelse & Metoden sørger for at opdaterer retning og hastighed på bilen. \\\hline
	\end{tabularx}
	\caption{Metodebeskrivelse for \texttt{userInput}}
	\label{table:met_userInput}
\end{table}

\begin{table}[H]
	\begin{tabularx}{\textwidth}{| L{2.5 cm} | Z |} \hline
		Prototype & \texttt{int motorSetPWM(unsigned char speedForward, unsigned char speedBackward)} \\\hline
		Parametre & \texttt{speedForward} \newline Den ønskede hastighed fremad. 0..255\newline
		\texttt{speedBackward} \newline Den ønskede hastighed bagud. 0..255
		\\\hline
		Returværdi &  \texttt{int} \newline 1 hvis alle operationer er gået okay ellers -1. \\\hline
		Beskrivelse & Metoden sørger for at opdaterer retning og hastighed på bilen. \\\hline
	\end{tabularx}
	\caption{Metodebeskrivelse for \texttt{userInput}}
	\label{table:met_userInput}
\end{table}

