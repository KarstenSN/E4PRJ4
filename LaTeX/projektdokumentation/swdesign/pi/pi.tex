\section{Pi} \label{sec:swdesign_pi}

\subsection{Klassediagram}
I dette afsnit beskrives det overordnede design på den software der kommer til at ligge på Pi. På figur \ref{fig:cd_pi} ses et klassediagram der opdeler funktionaliteten i klasser.

\begin{figure}[h]
\centering
\includegraphics[width=\textwidth* 9/10]{../fig/diagrammer/bil/cd_pi.pdf}
\caption{Klassediagram over Pi}
\label{fig:cd_pi}
\end{figure}

\subsubsection{Controller-klasse: Pi}
Controller-klassen Pi indeholder main funktionen og har derfor ansvaret for at styre slagets gang. Klassen skal derfor iværksætte initialisering af alle de klasser som den har ejerskab over. En af klassen ansvarsområder er at indsamle data fra sensorerne, og dette gøres ved at starte en særskilt tråd til dette. Denne tråd skal også sørge for at iværksætte Aks-klassen hver gang nye data er indsamlet.

\subsubsection{Controller-klasse: Aks}
Controller-klassen analyserer indkomne sensordata og i tilfældet at bilen er ved at køre ind i en forhindring, blokeres brugerinput og Aks styrer udenom eller bremser.

\subsubsection{Domain-klasse: Data}
Denne klasse har til formål at indsamle alle sensordata i en datastruktur. Disse data gemmes i memory kan ikke overstige en defineret størrelse. Brugerinput gemmes ikke i denne klasse.

\subsubsection{Domain-klasse: Log}
Denne klasse har til formål at gemme samtlige systemhændelser i den fil, så kilden til eventuelle programcrash kan identificeres. Alle klasser på Pi har en reference til denne log, så de hver i sær kan skrive til den. På figur \ref{fig:cd_pi} er der undladt at lave pile fra alle klasser til denne, da dette vil gøre diagrammet uoverskueligt. 

\subsubsection{Domain-klasse: Settings}
Settings er datastruktur der indeholder indstillinger for maksimal hastighed, AKS status, og styretøjs kalibrering. Indstillingerne er gemt i en fil som kan tilgås af Pi-klassen og Steering-klassen.

\subsubsection{Boundary-klasse: PcCom}
Boundary-klassen PcCom håndterer kommunikationen imellem PC og Bil. Denne kommunikation sker vha. UDP via Wi-Fi.

\subsubsection{Boundary-klasse: Steering}
I denne klasse styres bilens aktuatorer. Dette er altså en driver til både motoren der skaber fremdrift og servo-motoren der styrer forhjulene. Klassen tager ligeledes højde for brugers indstillinger.

\subsubsection{Boundary-klasse: Psoc (Pi)}
Denne klasse håndterer kommunikationen til PSoC'en hvor afstandssensorer og tachometer er monteret, og konverterer sensordata til brugbar distance- og hastighedsmålinger. Klassen håndterer alle fire sensorer samt tachometeret. Desuden skulle den have håndteret kommunikation med MPU6050, hvis denne var blevet yderligere implementeret.

\clearpage

\subsection{Sekvensdiagrammer} \label{sec:swdesign_sekvensdiagrammer}
Herunder er udarbejdet sekvensdiagrammer for den funktionalitet som Pi blokken på bilen skal have. Der er tage udgangspunkt i de de tidligere fremstillede use cases. Sekvensdiagrammer for UC8 og UC12 er udeladt, da disse kun indeholder handlinger over interaktion imellem bruger og PC.

\begin{figure}[h]
\centering
\includegraphics[]{../fig/diagrammer/bil/sd_uc1.pdf}
\caption{Sekvensdiagram over  bilens funktionalitet i UC1: Aktiver system}
\label{fig:sd_uc1_bil}
\end{figure}

\begin{figure}[h]
\centering
\includegraphics[]{../fig/diagrammer/bil/sd_uc2.pdf}
\caption{Sekvensdiagram over  bilens funktionalitet i UC2: Stream video}
\label{fig:sd_uc2_bil}
\end{figure}

\begin{landscape}

\begin{figure}[h]
\centering
\includegraphics[]{../fig/diagrammer/bil/sd_uc3_1.pdf}
\caption{Sekvensdiagram over  bilens funktionalitet i UC3: Overvåg sensorer - Del 1}
\label{fig:sd_uc3_1_bil}
\end{figure}

\end{landscape}

\begin{figure}[h]
\centering
\includegraphics[]{../fig/diagrammer/bil/sd_uc3_2.pdf}
\caption{Sekvensdiagram over  bilens funktionalitet i UC3: Overvåg sensorer - Del 2}
\label{fig:sd_uc3_2_bil}
\end{figure}

\clearpage

\begin{figure}[h]
\centering
\includegraphics[]{../fig/diagrammer/bil/sd_uc4.pdf}
\caption{Sekvensdiagram over  bilens funktionalitet i UC4: Undvig forhindring}
\label{fig:sd_uc4_bil}
\end{figure}

\begin{landscape}

\begin{figure}[h]
\centering
\includegraphics[]{../fig/diagrammer/bil/sd_uc5.pdf}
\caption{Sekvensdiagram over  bilens funktionalitet i UC5: Kør bil frem/tilbage}
\label{fig:sd_uc5_bil}
\end{figure}

\end{landscape}

\begin{figure}[h]
\centering
\includegraphics[]{../fig/diagrammer/bil/sd_uc6.pdf}
\caption{Sekvensdiagram over  bilens funktionalitet i UC6: Drej til højre/venstre}
\label{fig:sd_uc6_bil}
\end{figure}

\begin{figure}[h]
\centering
\includegraphics[]{../fig/diagrammer/bil/sd_uc7.pdf}
\caption{Sekvensdiagram over  bilens funktionalitet i UC7: Brems bil}
\label{fig:sd_uc7_bil}
\end{figure}

\clearpage

\begin{figure}[h]
\centering
\includegraphics[]{../fig/diagrammer/bil/sd_uc9.pdf}
\caption{Sekvensdiagram over  bilens funktionalitet i UC9: Tænd/sluk AKS}
\label{fig:sd_uc9_bil}
\end{figure}

\begin{figure}[h]
\centering
\includegraphics[]{../fig/diagrammer/bil/sd_uc10.pdf}
\caption{Sekvensdiagram over  bilens funktionalitet i UC10: Indstil makshastighed}
\label{fig:sd_uc10_bil}
\end{figure}

\clearpage

\begin{figure}[h]
\centering
\includegraphics[]{../fig/diagrammer/bil/sd_uc11.pdf}
\caption{Sekvensdiagram over  bilens funktionalitet i UC11: Kalibrer styretøj}
\label{fig:sd_uc11_bil}
\end{figure}

\clearpage
\subsection{Klassebeskrivelser}
\subsection{Dataklassen}

Dataklassen håndterer alt den data der skal kommunikeres imellem brugeren og brugerinterfacet på PCen og ud til sensorer og aktautorer på bilen, så længe de er opbevaret på Pi. Klassen/datastrukturen er implementeret ganske simpelt med variable frem for en fil, da dette vil gøre klassen hurtigere. Igennem koden i denne klasse er der indtil flere \texttt{\#ifdef DEBUG}, hvilket giver mulighed for at slå debugging til ved at definere \texttt{DEBUG} i mainfilen. \\
Constructoren, der er vist i listing \ref{lst:data_con}, er for klassen er ganske simpel. Den initialiserer alle variable til ''0'' og gemmer en pointer til logfilen.

\lstinputlisting[linerange=Data::Data1-Data::Data2, label=lst:data_con, caption=Constructor for dataklassen.]{../../src/bil/data/data.cpp}

Klassen har en række variable \ref{write} metoder, som alle er beskyttet af variable

%TODO Skal skrivesfærdig
TODO skal skrives færdig.
\clearpage
\subsubsection{Domain-klasse: Log}

Loggen har til formål at kunne lokalisere og identificere fejl i systemet. Loggen skal oprettes mere eller mindre globalt og der skal efterfølgende medgives en pointer til samtlige klasser på Pi. Alle disse klasser skal således anvende loggen som debugging redskab. Der skal som udgangspunkt kun skrives i loggen hvis en fejl opstår, da loggen ellers bliver uoverskuelig. Når der skrives til loggen, anvender den pågældende tråd cpu-tid, hvilket ligeledes er en grund til at være opmærksom på hvornår det er smart at skrive til loggen. \\
For at forhindre at log-entries fra forskellige tråde sammenflettes, skal der i implementeringen anvendes \texttt{std::mutex} som lås når der skrives i loggen.

\begin{figure}[h]
\centering
\includegraphics[]{../fig/diagrammer/bil/cd_log.pdf}
\caption{Klassebeskrivelse for domain-klassen Log}
\label{fig:cd_log}
\end{figure}

\textbf{Attributter}

\begin{table}[h]
\begin{tabularx}{\textwidth}{| Z | Z | L{10cm} |} \hline
Navn & Type & Beskrivelse \\\hline

\texttt{currentTime} & \texttt{time\_t} &Denne attribut bruges til at gemme det nuværende tidspunkt, når \texttt{getTimestamp} kaldes. \\\hline

\texttt{localTime} & \texttt{struct tm*} & Anvendes til at holde tiden i et læseligt format. \\\hline

\texttt{mutex} & \texttt{std::mutex} & Anvendes som lås i \texttt{std::lock\_guard} der forhindrer flere tråde i at skrive i loggen på samme tid. \\\hline

\texttt{logFile} & \texttt{std::ofstream} & File descriptor til logfilen. \\\hline

\end{tabularx}
\caption{Attributter for klassen Log}
\label{table:attr_log}
\end{table}


\textbf{Metoder}
%------------------------------------- Log -------------------------------------
\begin{table}[h]
\begin{tabularx}{\textwidth}{| L{2.5 cm} | Z |} \hline
Prototype & \texttt{bool klassenavn(int velocity)} \\\hline
Parametre & \texttt{velocity} \newline Den hastighed der skal indlæses. \\\hline
Returværdi &  \texttt{bool} \newline Returnerer \texttt{true} hvis skrivningen er gået godt og \texttt{false} hvis skrivningen gik galt. \\\hline
Beskrivelse & Metoden indlæser den nyeste værdi af hastigheden i datastrukturen. \\\hline
\end{tabularx}
\caption{Metodebeskrivelse for \texttt{klassenavn}}
\label{table:met_klassenavn}
\end{table}


\clearpage
\subsubsection{Domain-klasse: Settings} \label{sec:settings}
Settings-klassen har til formål at håndtere de indstillinger Brugeren sætter på PC softwaren. 
Disse indstillinger skrives til klassen fra PcCom-klassen og læses af AKS- og Sterring-klassen. 
For at sikre at den indeholdte data ikke bliver korrupt, sørges der for at der ikke er flere tråde der kan læse og skrive til de samme data samtidigt. 
Dette gøres vha. \texttt{std::mutex} der er implementeret gennem biblioteket \texttt{mutex}. 
Denne klasse er efter implementering testet med de andre klasser.
%\clearpage dette er bevist
\subsection{Aks}\label{sec:aks_impl}

Aks klassen er det mellemled der behandler indsamlet information på bilen og vurderer om der er eventuelle farer og videreformidler input til bilens styreklasse, Steering.
På det vis styrer Aks alle "smarte" funktioner på bilen.

For lettere at holde styr på hvilke stadier eller "states" bilen kan være i er der implementeret en \texttt{enum} af stadier, som hjælper programmøren med at holde overblikket. Disse stadier afspejler fysiske tilstande, som bilen kan være i mht. sin fremdrift.
\texttt{Enum}'en kan ses i listing \ref{lst:Aks_states}.

\lstinputlisting[linerange=states0-states1, label=lst:Aks_states, caption=Enum til hjælp ved states i Aks]
{../../src/bil/inc/Aks.hpp}

Der er ydermere implementeret en hjælp i forhold til at holde styr på hvilket nummer forskellige sensorer har i systemet, når Aks f.eks. skal indlæse alle sensorer vha. en while-løkke. Se listing \ref{lst:Aks_sensorPos}.

\lstinputlisting[linerange=sensPos0-sensPos1, label=lst:Aks_sensorPos, caption=Enum til hjælp ved sensorpositioner i Aks]
{../../src/bil/inc/Aks.hpp}

Selve klassedeklarationen for Aks er meget ligetil i henhold til designet af Aks, se side \ref{sec:aks_design}, og er derfor ikke udpenslet yderligere her.
Derimod er selve implementering af \texttt{analyzeData()} værd at notere, se listing \ref{lst:Aks_analyzeData}.

Som udgangspunkt undersøger den hvilket stadie bilen er i, hvis den fx kører fremad undersøger den om de fremadrettede sensorers indlæste værdier udgører en fare for bilen. 
I dette tilfælde gælder det at ingen af de to fremadrettede sensorer må have en større forskel i afstand end \texttt{DELTA\_DISTANCE\_FOR\_ERROR}, som i udgangspunktet er sat til 10 (ud af 255 mulige).
Hvis dette udgør en fare returnerer metoden \texttt{true}, hvilket indebærer at aks skal reagere på faren.
Er det ikke tilfældet returnerer denne \texttt{false}, hvilket indebærer at der ikke skal reageres.

Metoden fungerer ligeledes for de øvrige stadier, blot at det drejer sig om andre sensorer.
Er bilen i stadiet \texttt{coasting} ved den ikke om den nødvendigvis bevæger sig fremad eller bagud og kontrollerer derfor alle afstandssensorer. 

\lstinputlisting[
linerange=analyze0-analyze1, 
label=lst:Aks_analyzeData, 
caption=Metoden analyzeData() i aks klassen
]{../../src/bil/Aks.cpp}

Kernen i Aks' funktionalitet ligger i metoden \texttt{activate()}.
Metoden fungerer overordnet set ved at være en uendelig while-løkke, som i starten henter input fra brugeren og tjekker om Aks funktionaliteten er slået til via brugerinterfacet på PC softwaren.
Hvis Aks funktionaliteten ikke er slået til, sættes bilens stadie til \texttt{still}, hvilket betyder at Aks aldrig analyserer på sensorer og sender brugerinput direkte videre til \texttt{Steering}.
Er denne funktion derimod slået til, kontrolleres den nuværende hastighed af bilen og hvis denne er tilstrækkelig afgøres det om bilen er i fremadgående eller bagudgående retning ud fra brugerinput.
Dette kan ses i listing \ref{lst:Aks_activate}.

\lstinputlisting[
linerange=activate0-activate1, 
label=lst:Aks_activate_1, 
caption=Indhentning af brugerinput og analyse af stadie i activate metoden i AKS
]{../../src/bil/Aks.cpp}

Herefter gemmes gamle sensorværdier i \texttt{old\_proxSensors}, så \texttt{analyzeData()} senere kan holde de nye sensorværdier op mod de gamle.
\texttt{activate} indsamler herefter nyt sensordata afhængig af hvilket stadie den er i.
Hvis bilen fx kører fremad analyserer udelukkende afstandssensorer der er fremadrettede.
Skulle bilen antage et ukendt stadige udskrives en fejl i loggen.
Se listing \ref{lst:Aks_activate_2}.

\lstinputlisting[
linerange=activate1-activate2, 
label=lst:Aks_activate_2, 
caption=Overførsel af gamle og indhentning af nye afstandssensordata.
]{../../src/bil/Aks.cpp}

Til slut analyseres disse data med \texttt{analyzeData()}, som returnerer \texttt{true} eller \texttt{false} afhængig af om der er behov for at undvige en forhindring.
Hvis der er behov, kaldes styringen i \texttt{Steering} klassen, med en bremsekommando i stedet for brugerens input.
Er der derimod ikke behov for undvigelse sendes brugerinput blot videre. Se listing \ref{lst:Aks_activate_3}.

\lstinputlisting[
linerange=activate2-activate3,
label=lst:Aks_activate_3, 
caption=Analysering og handling på baggrund af analyse.
]{../../src/bil/Aks.cpp}

Som det fremgår af ovenstående beskrivelser er Aks ikke færdigimplementeret i henhold til den funktionalitet der er stillet i Kravspecifikationen og den kode der er vedlagt er derfor i et meget råt stadie.
Der mangler at blive implementeret en funktionalitet til at bringe bilen helt til standsning ved identifikation af en forhindring og der mangler ligeledes funktionalitet til at dreje bilen, skulle en forhindring ikke være lige foran bilen.
\clearpage
\subsubsection{Boundary-klasse: PcCom} %TODO lav label

%TODO skal skrives
\clearpage
\subsection{Steeringklassen} \label{sec:steering_impl}
Steering klassen er den klasse der kontrollere PWM signalet til motorens fremdrift og til styretøj servoen. Den modtager nye input fra systemet om ændringer af fremdrift, retning på styretøj og brems. Derudover henter den, hver gang klassen skal opdaterer PWM signalet til motoren, den aktuelle hastighed på bilen fra Dataklassen. Klassen har kun en public metode den kan kaldes. I listing \ref{lst:steering_header} ses implementering af klassens headerfilen.\newline




\lstinputlisting[linerange=Steering::header1-Steering::header2, label=lst:steering_header, caption=\texttt{Header} for Steeringklassen.]{../../src/bil/steering/steering.hpp}

Constructoen sørger primært for at sætte WiringPi op. Der er en HW PWM og SW PWM del. Udover opsætning starter den en separat tråd der kører \texttt{Steering::PWMUpdate} i et loop indtil systemet lukkes ned 
\lstinputlisting[linerange=Steering::Steering1-Steering::Steering2, label=lst:steering_con, caption=\texttt{Constructor} for Steeringklassen.]{../../src/bil/steering/steering.cpp}


\clearpage
\lstinputlisting[linerange=Steering::~Steering1-Steering::~Steering2, label=lst:steering_decon, caption=\texttt{Deconstructor} for Steeringklassen.]{../../src/bil/steering/steering.cpp}


\lstinputlisting[linerange=Steering::userInput1-Steering::userInput2, label=lst:steering_userInput, caption=Metoden \texttt{userInput} Steeringklassen.]{../../src/bil/steering/steering.cpp} 


\lstinputlisting[linerange=Steering::brake1-Steering::brake2, label=lst:steering_brake, caption=Metoden \texttt{brake} Steeringklassen.]{../../src/bil/steering/steering.cpp}


\lstinputlisting[linerange=Steering::softbrake1-Steering::softbrake2, label=lst:steering_softbrake, caption=Metoden \texttt{softbrake} Steeringklassen.]{../../src/bil/steering/steering.cpp}


\lstinputlisting[linerange=Steering::turn1-Steering::turn2, label=lst:steering_turn, caption=Metoden \texttt{turn} Steeringklassen.]{../../src/bil/steering/steering.cpp}


\lstinputlisting[linerange=Steering::motorSetPWM1-Steering::motorSetPWM2, label=lst:steering_motorSetPWM, caption=Metoden \texttt{motorSetPWM} Steeringklassen.]{../../src/bil/steering/steering.cpp}


\lstinputlisting[linerange=Steering::PWMUpdate1-Steering::PWMUpdate2, label=lst:steering_PWMUpdate, caption=Metoden \texttt{PWMUpdate} Steeringklassen.]{../../src/bil/steering/steering.cpp}

\clearpage
\section{Psoc (PSoC)} \label{sub:sw_impl_psoc_psoc}
\subsection{Psoc (PSoC)}
PSoC'ens formål er at simplificere alt \IIC kommunikation, hvilket viste sig at være en nødvendighed, da implementeringen på Pi'en gav uforudsete problemer med kommunikationen med distancesensorerne.
Da der på Pi'en kører en Linux-distribution foregår \IIC kommunikationen som skrivning og læsning til og fra device-files der repræsenterer de respektive pins (SDA \& SCL), og da kommunikationen med distancesensorerne følger følgende protokollen fra Hardwaredesign afsnittet for distancesensoren.
Ydermere viste det sig at tachometeret, der vha. en schmittrigger trækkes til stel hver gang der detekteres en magnet, dette bliver detekteret som logisk lav på PSoC'en og  kalder den implementerede interrupt service rutine \texttt{ISR}. 
Det vil optage udnødvendig meget af Pi'ens processor og vil være meget tidskritisk i forhold til de andre opgaver som Pi-programmet varetager. Derfor beslutning om at ændre designretning.

Kravet til PSoC'en er, at koden der ligger herpå skal være så hurtig og effektiv som muligt, således at den kan aflæses når PI'en spørger på ny data. 
I listing \ref{lst:getDistance_FL2} ses implementeringen af denne kode.

\lstinputlisting
	[linerange=getDistance::FL-getDistance::FL1, caption=]
	{../../src/psoc/psoc_bil_1/psoc_bil.cydsn/main.c}

\lstinputlisting
	[linerange=getDistance::FL2-getDistance::FL3, label=lst:getDistance_FL2, caption=Front Left sensor aflæsningscyklus.]
	{../../src/psoc/psoc_bil_1/psoc_bil.cydsn/main.c}
	
aflæsningscyklus for de enkelte sensorer er identiske blot med ændret navn og index i \texttt{sendBuffer}

Tachometeret er simplere at aflæse, da alt dataen i forvejen er placeret på PSoC'en. I listing \ref{lst:sw_impl_psoc_getVelocity} ses interrupt service rutinen som køres hver gang der detekteres en magnet på hallswitchen.

\lstinputlisting
	[linerange=getVelocity::1-getVelocity::2, label=lst:sw_impl_psoc_getVelocity, caption=ISR til getVelocity.]
	{../../src/psoc/psoc_bil_1/psoc_bil.cydsn/main.c}

Til sidst kan implementeringen af programmets \texttt{main}-funktion ses i listing \ref{lst:sw_impl_psoc_main}

\lstinputlisting[linerange=main::1-main::2, label=lst:sw_impl_psoc_main, caption=Main program på PSoC.]{../../src/psoc/psoc_bil_1/psoc_bil.cydsn/main.c}

\clearpage

\subsection{Modultest for PSoC}

For at teste om kommunikationen imellem distancesensorene og PSoC'en fungerer korrekt, blev der foretaget en modultest, hvor følgende ønskedes opfyldt. 

\begin{enumerate}
  \item at der kan skrives korrekte kommander til alle 4 sensorer.
  \item at der kan læses korrekte 2-bytes værdier fra alle 4 sensorer.
\end{enumerate}

Der afvikles et main\_test program hvor der kontinuerligt skrives til de 4 sensorer én for én, og derefter læses den nuværende værdi retur i 2-bytes format. 

På figur \ref{fig:write_FL} til \ref{fig:write_RR} ses \texttt{write}-kommando sendt til alle 4 sensorer: 

\begin{figure}[h]
	\centering
	\includegraphics[scale=0.6]{../fig/billeder/psoc_distancesensor_modultest/I2C_write_0x70_FL.png}
	\caption{write til adresse 0x70 sensor FL}
	\label{fig:write_FL}
\end{figure}

\begin{figure}[h]
	\centering
	\includegraphics[scale=0.6]{../fig/billeder/psoc_distancesensor_modultest/I2C_write_0x71_FR.png}
	\caption{write til adresse 0x71 sensor FR}
	\label{fig:write_FR}
\end{figure}

\begin{figure}[h]
	\centering
	\includegraphics[scale=0.6]{../fig/billeder/psoc_distancesensor_modultest/I2C_write_0x73_RL.png}
	\caption{write til adresse 0x73 sensor RL}
	\label{fig:write_RL}
\end{figure}

\begin{figure}[h]
	\centering
	\includegraphics[scale=0.6]{../fig/billeder/psoc_distancesensor_modultest/I2C_write_0x76_RR.png}
	\caption{write til adresse 0x76 sensor RR}
	\label{fig:write_RR}
\end{figure}

\newpage

På figur \ref{fig:read_FL} til \ref{fig:read_RR} ses \texttt{read}-kommando sendt til alle 4 sensorer:

\begin{figure}[h]
	\centering
	\includegraphics[scale=0.6]{../fig/billeder/psoc_distancesensor_modultest/I2C_read_0x70_FL.png}
	\caption{read til adresse 0x70 sensor FL}
	\label{fig:read_FL}
\end{figure}

\begin{figure}[h]
	\centering
	\includegraphics[scale=0.6]{../fig/billeder/psoc_distancesensor_modultest/I2C_read_0x71_FR.png}
	\caption{read til adresse 0x71 sensor FR}
	\label{fig:read_FR}
\end{figure}

\begin{figure}[h]
	\centering
	\includegraphics[scale=0.6]{../fig/billeder/psoc_distancesensor_modultest/I2C_read_0x73_RL.png}
	\caption{read til adresse 0x73 sensor RL}
	\label{fig:read_RL}
\end{figure}

\begin{figure}[h]
	\centering
	\includegraphics[scale=0.6]{../fig/billeder/psoc_distancesensor_modultest/I2C_read_0x76_RR.png}
	\caption{read til adresse 0x76 sensor RR}
	\label{fig:read_RR}
\end{figure}


\textbf{NB!}
Til fordel for andre mere systemkritiske blokke, er accelerometer klassen blevet nedprioriteret herefter og er derfor ikke yderligere beskrevet eller implementeret, hvilket blev besluttet umiddelbart i starten af implementeringsfasen. Det er muligt at implementere sensoren, med den ændring at den således skal placeres umiddelbart i forlængelse af PSoC'en, i kontekst med afstandssensorerne og tachometeret.