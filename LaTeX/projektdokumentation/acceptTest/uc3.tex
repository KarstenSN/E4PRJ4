\begin{longtable}{| l | >{\raggedright}X | >{\raggedright}X | >{\raggedright}X | >{\raggedright\arraybackslash}p{2.3cm} |} \hline
	\multicolumn{2}{|l|}{\textbf{Use case under test}} & 
	\multicolumn{3}{l|}{UC3: Overvåg sensor} \\ \hline
	
	\multicolumn{2}{|l|}{\textbf{Scenarie}} & 
	\multicolumn{3}{l|}{Hovedscenarie} \\ \hline
	
	\multicolumn{2}{|l|}{\textbf{Forudsætning}} & 
	\multicolumn{3}{p{10.2cm}|}{UC1 frem til punkt 6 er fuldført \hfill} \\ \hline
	%\multicolumn{5}{|l|}{}\\ \hline
	\textbf{Step} & \textbf{Handling} & \textbf{Forventet Resultat} & \textbf{Resultat} & \textbf{Godkendt / Kommentar} \\ \hline

	3.1 & Bruger åbner en Linux terminal og indtaster ''ssh pi@'' efterfulgt af Pi'ens IP adresse. 
		& Visuel test:\\ Terminalen spørger om et password.
		& Terminalen spørger om password til brugeren pi.
		& OK\\ \hline

	3.2 & Bruger indtaster ''1''.
		& Visuel test:\\ Terminalen viser pi@raspberry \$.
		& Terminalen godtager passworded og står klar til brug.
		& OK\\ \hline
		
	3.3 & Bruger indtaster nano pi-log.txt
		& Visuel test:\\ Log filen viser Accelerometer initialisering.. Done. Tachometer initialisering.. Done. Distancesensors initialisering.. Done.
		& Logfilen viser initialisering af Log, Tachometer og Distancesensor.
		& IKKE OK - Accelerometeret er ikke implementeret.\\ \hline
		
	3.4 & Bruger kører en tur med bilen og observerer hovedmenu i softwaren på PC.
		& Visuel test:\\ Bruger observerer at data for bilens hastighed, afstand til forhindring og acceleration fremgår af brugerfladen.
		& Softwaren opdaterer bilens hastighed og afstand til forhindring men accelerationen opdateres ikke.
		& IKKE OK - Accelerometeret er ikke implementeret.\\ \hline

\caption{Accepttest for UC3: Overvåg sensor}\label{tbl:acceptuc3}
\end{longtable}