\chapter{Accepttest} \label{ch:Accepttest}
\section*{Version}
\begin{table}[h]
	\centering
	\begin{tabularx}{\textwidth - 2cm}{|l|l|l|X|}
	\hline
	Dato			& Version			& Initialer 		& Ændring.									\\ \hline
	29. september 	& 1 				& Alle				& Første udkast til accepttestspecifikation.\\ \hline
	26. oktober		& 2 				& PKP, KT og JEP	& Mindre rettelser efter review.			\\ \hline
	9. december		& 3 				& PKP, KT og HBN	& Accepttest udført og resultater indtastet.\\ \hline
	11. december	& 4 				& PKP 				& Tilføjet enkelte kommentarer til testet.	\\ \hline
	\end{tabularx}
\end{table}

\section*{Indledning}

Dette kapitel beskriver definitionen og resultatet af accepttesten.
Kapitlet bør læses i sammenhæng med kapitel \ref{ch:kravspecifikation} \nameref{ch:kravspecifikation} på side \pageref{ch:kravspecifikation}, om ikke andet med forståelse for krav og Use Cases.

\clearpage

\section{Funktionelle Krav}

Fremgangsmåden for test af funktionelle krav er generelt taget udgangspunkt i Use Cases. I tabel \ref{tbl:kravucmatrise} er vist en matrise der sammenholder Use Cases med funktionelle krav, der sikrer at alle krav bliver testet ved test af Use Cases. Der henvises til kravnumre i afsnit \ref{sec:funktionelle_krav} på side \pageref{sec:funktionelle_krav}.

\begin{table}[h]
\centering
\begin{tabularx}{\textwidth-5cm}{| L{1 cm} | X | X | X | X | X | X | X | X | X | X | X | X |}
\hline
Krav&1&2&3&4&5&6&7&8&9&10&11&12 \\ \hline
UC1 & & & & & &X& & & &X &  &   \\ \hline
UC2 & & & & & & & & & &X &  &   \\ \hline
UC3 & & & & & & &X& & &  &  &   \\ \hline
UC4 &X&X&X& & & &X&X&X&  &  &   \\ \hline
UC5 &X& &X& &X& & & & &  &  &   \\ \hline
UC6 & &X& & &X& & & & &  &  &   \\ \hline
UC7 & & &X& & & & & & &  &  &X  \\ \hline
UC8 & & & & & &X& & & &  &  &   \\ \hline
UC9 & & & & & & & & & &  &X &   \\ \hline
UC10& & &X&X& & & & & &  &  &   \\ \hline
UC11& &X& & & & & & & &  &  &   \\ \hline
UC12& & & & & & & & & &  &  &   \\ \hline
\end{tabularx}
\caption{Use Case-krav matrise}
\label{tbl:kravucmatrise}
\end{table}

%Her vælges bredde på hele tabellen samt hvilken FIL selve tabellen defineres i.
\subsubsection{Use Case 1: Aktiver system}
\LTXtable{\textwidth}{accepttest/uc1}
%\clearpage
\subsubsection{Use Case 2: Stream Video}
\LTXtable{\textwidth}{accepttest/uc2}
\clearpage
\subsubsection{Use Case 3: Overvåg sensorer}
\LTXtable{\textwidth}{accepttest/uc3} 
\clearpage
\subsubsection{Use Case 4: Undvig forhindring}
\LTXtable{\textwidth}{accepttest/uc4}
\clearpage
\subsubsection{Use Case 5: Kør bil frem/tilbage}
\LTXtable{\textwidth}{accepttest/uc5} 
\clearpage
\subsubsection{Use Case 6: Drej bil til højre/venstre}
\LTXtable{\textwidth}{accepttest/uc6}
\clearpage
\subsubsection{Use Case 7: Brems bil}
\LTXtable{\textwidth}{accepttest/uc7}
%\clearpage
\subsubsection{Use Case 8: Konfigurer IP-adresse}
\LTXtable{\textwidth}{accepttest/uc8}
\clearpage
\subsubsection{Use Case 9: Tænd/Sluk AKS}
\LTXtable{\textwidth}{accepttest/uc9}
\clearpage
\subsubsection{Use Case 10: Indstil maksimalhastighed}
\LTXtable{\textwidth}{accepttest/uc10}
\clearpage
\subsubsection{Use Case 11: Kalibrer styrtøj}
\LTXtable{\textwidth}{accepttest/uc11}
\clearpage
\subsubsection{Use Case 12: Afbryd system}
\LTXtable{\textwidth}{accepttest/uc12}

\clearpage

\section{Ikke-funktionelle krav}
\LTXtable{\textwidth}{Accepttest/ikke_funk}

\clearpage