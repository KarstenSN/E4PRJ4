\begin{longtable}{| l | >{\raggedright}X | >{\raggedright}X | >{\raggedright}X | >{\raggedright\arraybackslash}p{2.3cm} |} \hline
	\multicolumn{2}{|l|}{\textbf{Use case under test}} & \multicolumn{3}{l|}{UC8: Konfigurer IP-adresse} \\ \hline
	\multicolumn{2}{|l|}{\textbf{Scenarie}} & \multicolumn{3}{l|}{Hovedscenarie} \\ \hline
	\multicolumn{2}{|l|}{\textbf{Forudsætning}} & \multicolumn{3}{p{10.2cm}|}{UC1: Aktiver system er udført, bilen og PC er på samme netværk, at systemet viser ''Hovedvindue'' samt at systemet er operationelt.\hfill} \\ \hline
	%\multicolumn{5}{|l|}{}\\ \hline
	\textbf{Step} & \textbf{Handling} & \textbf{Forventet Resultat} & \textbf{Resultat} & \textbf{Godkendt / Kommentar} \\ \hline
	
	8.1 & Bruger trykker på ''Konfigurer IP''. 
		& Visuel test: \\ Konfigurations menuen for IP-adressen vises, og der er mulighed for at indtaste en IP-adresse. 
		&   
		&  \\ \hline		
	8.2 & Bruger indtaster bilens IP-adresse. Og trykker ''Gem''. 
		& Visuel test: \\ Systemet viser ''Hovedvindue''. 
		&  
		&  \\ \hline
	8.3 & Bruger trykker på ''Opret forbindelse''. 
		& Visuel test: \\ Hovedmenuen viser et videobillede samt opdater variablerne Hastighed, Afstand, 			Acceleration og Makshastighed. 
		&   
		&  \\ \hline
	
\caption{Accepttest for UC8: Konfigurer IP-adresse}\label{tbl:acceptUC8}
\end{longtable}