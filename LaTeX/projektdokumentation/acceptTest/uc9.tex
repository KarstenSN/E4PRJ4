\begin{longtable}{| l | >{\raggedright}X | >{\raggedright}X | >{\raggedright}X | >{\raggedright\arraybackslash}p{2.3cm} |} \hline
	\multicolumn{2}{|l|}{\textbf{Use case under test}} & \multicolumn{3}{l|}{UC9: Tænd/sluk AKS} \\ \hline
	\multicolumn{2}{|l|}{\textbf{Scenarie}} & \multicolumn{3}{l|}{Hovedscenarie} \\ \hline
	\multicolumn{2}{|l|}{\textbf{Forudsætning}} & \multicolumn{3}{p{10.2cm}|}{UC1: Aktiver system er udført, bilen og PC er på samme netværk, at systemet viser ”Hovedvindue” samt at systemet er operationelt.\hfill} \\ \hline
	%\multicolumn{5}{|l|}{}\\ \hline
	\textbf{Step} & \textbf{Handling} & \textbf{Forventet Resultat} & \textbf{Resultat} & \textbf{Godkendt / Kommentar} \\ \hline
	
	9.1 & Bruger trykker på ”AKS”. 
		& Visuel test: \\ System viser ”AKS-menu” med mulighed for at tænde/slukke for AKS og status for AKS. 
		&   
		&  \\ \hline
	9.2 & Bruger trykker på "Tænd AKS". 
		& Visuel test: \\ ”Hovedvindue” indikerer status af AKS for tændt. 
		&  
		&  \\ \hline
	9.3 & Bruger trykker på "AKS". 
		& Visuel test: \\ System viser ”AKS-menu” med mulighed for at tænde/slukke for AKS og status for AKS. 
		&   
		&  \\ \hline
	9.4 & Bruger trykker på "Sluk AKS". 
		& Visuel test: \\ ”Hovedvindue” indikerer status af AKS for slukket. 
		&  
		&   \\ \hline

\caption{Accepttest for UC9: Tænd/sluk AKS}\label{tbl:acceptuc9}
\end{longtable}