\begin{longtable}{| l | >{\raggedright}X | >{\raggedright}X | >{\raggedright}X | >{\raggedright\arraybackslash}p{2.3cm} |} \hline
	\multicolumn{2}{|l|}{\textbf{Use case under test}}  & \multicolumn{3}{l|}{UC10: Indstil makshastighed} \\ \hline
	\multicolumn{2}{|l|}{\textbf{Scenarie}} 			& \multicolumn{3}{l|}{Hovedscenarie} \\ \hline
	\multicolumn{2}{|l|}{\textbf{Forudsætning}} 		& \multicolumn{3}{p{10.2cm}|}{UC1: Aktiver system er udført, bilen og PC er på samme netværk, at systemet viser ''Hovedvindue'' samt at systemet er operationelt.\hfill} \\ \hline
	%\multicolumn{5}{|l|}{}\\ \hline
	\textbf{Step} 	& \textbf{Handling} & \textbf{Forventet Resultat} & \textbf{Resultat} & \textbf{Godkendt / Kommentar} \\ \hline
	
	10.1 & Bruger trykker på ''Indstil makshastighed''. 
		 & Visuel test: Menu med mulighed for indtastning af makshastighed fremkommer. 
		 & Brugeren har mulighed for indtastning.  
		 &  \\ \hline
	10.2 & Systemet præsenterer menu makshastighed med mulighed for indtastning af makshastighed fra 1-10 km/t. 
		 & Visuel test: Menuen fremkommer. 
		 & Systemet præsenterer menu makshastighed med mulighed for indtastning. 
		 & \\ \hline
	10.3 & Menuen indikerer bilens nuværende makshastighed. 
		 & den nuværende makshastighed vises.
		 & Systemet viser den nuværende makshastighed. 
		 & \\ \hline
	10.4 & Bruger indtaster bilens nye makshastighed. 
		 & Den ønskede makshastighed indtastes. 
		 & Bruger indtaster den ønskede makshastighed. 
		 & \\ \hline
	10.5 & Bruger trykker på ''Opdater''. 
		 & Systemet viser den nye makshastighed. 
		 & ''Hovedvindue'' viser den nye værdi som makshastighed. 
		 & \\ \hline
\caption{Accepttest for UC10: Indstil makshastighed }\label{tbl:acceptuc10}
\end{longtable}