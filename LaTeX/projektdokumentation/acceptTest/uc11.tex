\begin{longtable}{| l | >{\raggedright}X | >{\raggedright}X | >{\raggedright}X | >{\raggedright\arraybackslash}p{2.3cm} |} \hline
	\multicolumn{2}{|l|}{\textbf{Use case under test}} & 
	\multicolumn{3}{l|}{UC11: Kalibrer styretøj} \\ \hline
	
	\multicolumn{2}{|l|}{\textbf{Scenarie}} & 
	\multicolumn{3}{l|}{Hovedscenarie} \\ \hline
	
	\multicolumn{2}{|l|}{\textbf{Forudsætning}} & 
	\multicolumn{3}{p{10.2cm}|}{UC1: Aktiver system er udført, bilen og PC er på samme netværk, at systemet viser ''Hovedmenu'', at systemet er operationelt samt bilen holder stille\hfill} \\ \hline
	%\multicolumn{5}{|l|}{}\\ \hline
	\textbf{Step} & \textbf{Handling} & \textbf{Forventet Resultat} & \textbf{Resultat} & \textbf{Godkendt / Kommentar} \\ \hline
	
	11.1 & Bruger vælger ''Kalibrer styretøj'' 
		 & Visuel test: \\ Menu med mulighed for kalibrering fremkommer.
		 & Menu med mulighed for kalibrering vises
		 & OK\\ \hline
	11.2 & Bruger indtaster en værdi mellem 50 og -50 for kalibrering. 
		 & Den ønskede værdi vises.
		 & 
		 & \\ \hline
	11.3 & Bruger trykker på ''Ok''. 
		 & Forhjulene drejer en absolut værdi mod enten højre eller venstre: positiv værdi giver udslag til højre, og negativ værdi giver udslag venstre.
		 & Forhjulene bevæger sig ikke
		 & Ikke OK - servo til forhjul ikke installeret\\ \hline
	11.5 & Systemet returnerer til ''Hovedvindue''
		 & Visuel test: \\ ''Hovedvindue'' fremkommer 
		 & Hovedvinduet vises ved tryk på OK
		 & OK\\ \hline
		 
\caption{Accepttest for UC11: Kalibrer styretøj }\label{tbl:acceptuc11}
\end{longtable}