\subsection{Tachometer}

I forbindelse med måling af tachometerets design på side \pageref{sub:systemarkitektur_tachometer} blev der foretaget nogle målinger af strømmene som løber i kredsløbet. Det viser sig, når kredsløbet forholder sig passivt og der ikke er en magnet tilstede, at der løber ca. $520\mu A$ og at der under tilstedeværelse af magnet løber ca. $1.32mA$, hvilket gør strømforbruget for tachometeret væsentligt under forventet størrelse.

Forhjulet har en diameter på ca. 6.2 cm. Ved at placere i alt 5 magneter, en på hver akse, kan vi få en relativ høj opløsning af hastighed, selv når bilen bevæger sig langsomt. Der anvendes en PSoC, som aktiverer en timer når der detekteres en magnet, og aflæser en timerværdi, når den næste magnet passerer hallswitchen. Derved kan vi beregne bilens hastighed på følgende måde:

\begin{equation}
rpm = \dfrac{60s}{5\cdot \dfrac{T_{2} - T_{1}}{f_{clk}}}
\end{equation}

\begin{equation}
Km/t = \dfrac{O\cdot rpm}{60s}
\end{equation}