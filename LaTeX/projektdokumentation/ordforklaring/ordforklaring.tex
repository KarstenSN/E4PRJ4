%==================== Ordforklaring ====================

\section{Ordforklaring}

\subsubsection{System}
Det totale system, indeholdende både bil, software på PC og netværkskommunikation mellem Bil og PC. 

\subsubsection{Human Interface Device (HID)}
En måde at kommunikere med en computer fx et tastatur, eller for dette projekts tilfælde, en XBOX360 controller.

\begin{itemize}
	\item Right Trigger (RT) - ??
	\item Left Trigger (LT) - ??
	\item Flere knapper her.
	%TODO list alle knapper på XBOX controller
\end{itemize}

\subsubsection{Hovedmenu}
Hovedmenuen i software på PC, indeholder menupunkter, som brugeren kan navigere efter behov.

\subsubsection{Bil}
Køretøjet samt controller som udfører alle relevante opgaver for bilen. Alt kommunikation med PC, foregår gennem denne.

\subsubsection{Wi-Fi}
Trådløst netværk af specifikationerne ''IEEE 802.11'', som Bil og PC kommunikerer over.

\subsubsection{Anti-kollitionssystem (AKS)}
Et system på bilen bestående af fire ultralydssensorer, som er i stand til at forhindre en kollision ved at overtage styringen fra Bruger i tilfælde af en kollisionskurs. Der gøres forskel mellem "Aktiver AKS" og "Tænd/Sluk AKS".

\begin{itemize}
	\item Tænd/Sluk bruges i forbindelse med at slå AKS fra eller til, således bilen ikke vil undgå en kollision hvis slukket, men vil undgå en kollsion hvis tændt. 
	\item AKS aktiveres ifm. at bilen er på vej til at kollidere med en forhindring, hvorefter bilen vil forhindre en kollision (AKS tændt) eller ej (AKS slukket).
\end{itemize}

\subsubsection{Ultralydssensorer}
Ultralydssensorerne benævnes herefter som hhv. Front Left (FL), Front Right (FR), Rear Left (RL) og Rear Right (RR). 

\clearpage