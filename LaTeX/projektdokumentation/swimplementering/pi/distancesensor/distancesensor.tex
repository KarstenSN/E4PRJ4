//SW-implementering af Distancesensorklassen
\subsection{Distancesensorklassen}

Distancesensorklassen håndterer alt kommunikation mellem Pi og de 4 afstandssensorer. Som førnævnt er dette implementeret via I2C-kommunikation. Klassen er implementeret med en enkelt \texttt{getDistance(name)}-metode. på denne måde holdes et enkelt interface udadtil. Denne funktion kan kaldes fra dataklassen når der ønskes data hentet. Metoden tilgår internt I2C-bussen og henter data fra sensorerne. Herunder ses implementering af header for distanceSensorklassen.

\lstinputlisting[linerange=DS::DS1-DS::DS2, 
label=lst:Distancesensor_hpp, 
caption=headerfil for distanceSensorklassen.]{../../src/bil/distancesensor/distancesensor.hpp}

Klassen er primært designet omkring \texttt{getDistance()}-funktion, således at der der i her foretages den egentlige behandling af data, ligeledes er det her der skrives til de midlertidige variabler, herunder ses implementering af metoden: 

\lstinputlisting[linerange=DS::getD1-DS::getD2, 
label=lst:Distancesensor_hpp, 
caption=headerfil for distanceSensorklassen.]{../../src/bil/distancesensor/distancesensor.cpp}

Afstandssensorene leveres formonteret på print hvor benene fra IC'en er trukket til harwinpins som let kan tilgås. Til kommunikationen med sensoren benyttes følgende 4 linjer: 

\begin{itemize}
	\item pin 7: VCC: Forsyning
	\item pin 6: GND: Reference
	\item pin 5: SCL: Clock
	\item pin 4: SDA: Data
\end{itemize}

Data kommunikeres på SDA-linjen med reference til GND,  SCL styrer clocken.
Der benyttes de hardcodede default adresser til de 4 sensorer, disser er som følger: 

\begin{table}[h]\centering
	\begin{tabular}{| l | l |} \hline
		\textbf{Sensor} 	& \textbf{Adresse}  \\\hline
		FL 					& \texttt{0x70} 	\\\hline
		FR 					& \texttt{0x71} 	\\\hline
		RL 					& \texttt{0x73} 	\\\hline
		RR 					& \texttt{0x76} 	\\\hline
	\end{tabular}
	\caption{Adresser på afstandssensorer}
	\label{table:adr_afstandssensorer}
\end{table}

For at kunne benytte sensorerne skal der sendes en række kommandoer til hver enkelt sensor. Disse kommandoer følger fremgangsmåden herunder:

\begin{enumerate}
  \item \textit{write}-mode-Kommando sendes til den given sensoradresse.
  \item Der kan skrives til sensoradressen
  \item \textit{read}-mode-Kommando sendes til den given sensoradresse.
  \item Der kan skrives fra sensoradressen
\end{enumerate}

Alle \textit{write}- eller \textit{read}-kommandoer foregår altså som koblinger af 2 kommandoer, én hvor der sætte ''mode'' og til at foretage en given kommando. Derudover kommer databladets anbefalede 100ms til at foretage en Range-reading inden den aflæses.

Følgende kommandoer er benyttet til at skrive og læse fra sensorerne: 

\begin{table}[h]\centering
	\begin{tabular}{| l | l |} \hline
		\textbf{Kommando} & \textbf{Adresse}	\\\hline
		FLwrite 		  & 0xE0				\\\hline
		FRwrite 		  & 0xE2				\\\hline
		RLwrite 		  & 0xE6				\\\hline
		RRwrite 		  & 0xEC				\\\hline
		FLread  		  & 0xE1				\\\hline
		FRread  		  & 0xE4				\\\hline
		RLread  		  & 0xE7				\\\hline
		RRread  		  & 0xED				\\\hline
		StartReading	  & 0x51				\\\hline
	\end{tabular}
	\caption{\textit{read/write}-kommandoer til afstandssensorer}
	\label{table:adr_afstandssensorer_kommadoer}
\end{table}