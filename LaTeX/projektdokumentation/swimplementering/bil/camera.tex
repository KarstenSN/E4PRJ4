\subsubsection{Camera}
På Pi'en skal der installeres en virtuel driver til kameraet kaldet uv4l, før det vil virke med programmet motion. Cameraet der skal bruges er et \textbf{Raspberry Pi Camera Rev 1.3}. Kameraet forbindes med et fladkabel til connectoren på Pi'en som der er markeret med teksten ''Camera''. Når kameraet er forbundet, startes Pi'en op og der logges ind via en ssh-forbindelse. Efterfølgende indtastes: 
\begin{lstlisting}
$ sudo raspi-config
\end{lstlisting}
I denne menu skal cameraet aktiveres. Når kameraet er aktiveret trykkes exit og i terminalen skrives:
\begin{lstlisting}
$ curl http://www.linux-projects.org/listing/uv4l_repo/lrkey.asc | sudo apt-key add -
\end{lstlisting}
Herved åbned en fil som angiver hvilke url \textbf{apt-get} henter filer fra. Nederst i filen tilføjes:
\begin{lstlisting}
deb http://www.linux-projects.org/listing/uv4l_repo/raspbian/ wheezy main
\end{lstlisting}
Når filen er opdateret skrives:
\begin{lstlisting}
$ sudo apt-get update
$ sudo apt-get install uv4l uv4l-raspicam motion
\end{lstlisting}
Nu er driveren uv4l samt motion installeret. Før vi kan begynde at streame skal der ændres i motion's config file.  Filen åbnes ved at skrive:
\begin{lstlisting}
$ sudo nano /etc/motion/motion.conf
\end{lstlisting}
I filen skal følgende variabler stå til:
\begin{lstlisting}
daemon off
videodevice /dev/video0
width 640
height 480
framerate 25
quality 100
control_port 8080
webcam_port 8081
webcam_quality 100
webcam_localhost off
webcam_limit 0
threshold 99999999
\end{lstlisting}
Streamet kan nu startes ved følgende kommando:
\begin{lstlisting}
$ uv4l --driver raspicam --auto-video_nr
$ LD_PRELOAD=/usr/lib/uv4l/uv4lext/armv6l/libuv4lext.so motion -c ./motion.conf
\end{lstlisting}
Følgende findes også i scriptet startVidStrean.sh som ligger i bilagende. Scriptet skal kopieres over i mappen \textbf{/home/pi/Documents/}
For at streamet skal starte ved opstart, skal scriptet tilføjes til crontab. Dette gøres ved at åbne crontab og skrive stien til scriptet. Crontab åbnes:
\begin{lstlisting}
$ sudo crontab -e
\end{lstlisting}
I slutningen af filen tilføjes:
\begin{lstlisting}
$ /home/pi/Documents/startVidStream.sh
\end{lstlisting}


















