\subsection{Dataklassen}

Dataklassen håndterer alt den data der skal kommunikeres imellem brugeren og brugerinterfacet på PCen og ud til sensorer og aktautorer på bilen, så længe de er opbevaret på Pi. Klassen/datastrukturen er implementeret ganske simpelt med variable frem for en fil, da dette vil gøre klassen hurtigere. Igennem koden i denne klasse er der indtil flere \texttt{\#ifdef DEBUG}, hvilket giver mulighed for at slå debugging til ved at definere \texttt{DEBUG} i toppen af klassen. \\
Constructoren, der er vist i listing \ref{lst:data_con}, er ganske simpel. Den initialiserer alle variable til ''0'' og gemmer en pointer til logfilen.

\lstinputlisting[linerange=Data::Data1-Data::Data2, label=lst:data_con, caption=Constructor for dataklassen.]{../../src/bil/data/data.cpp}

Klassen har en række \texttt{write} metoder, som alle er beskyttet af en mutex, der forhindrer at der er flere forskellige tråde der kan redigere i variablen på samme tid. Hvis dette skete ville det kunne skabe invalide data. Metoderne \texttt{writeVelocity} og \texttt{writeAcceleration} er helt ens på nær den attribut om dataen gemmes i. I listing \ref{lst:data_wirteVelocity} er \texttt{writeVelocity} vist.

\lstinputlisting[linerange=Data::writeVelocity1-Data::writeVelocity2, label=lst:data_wirteVelocity, caption=Metoden \texttt{writeVelocity} i dataklassen.]{../../src/bil/data/data.cpp}

For at opbevare brugerinput, anvendes der \texttt{write}(listing \ref{lst:data_writeUserInput}) og \texttt{get}(listing \ref{lst:data_getUserInput}) metoder der bruger ''copy by reference'' til at overføre en struct med dataen. Denne struct er defineret i \texttt{utilities.hpp}. Copy by reference, er anvendt da det er vigtigt at overførslen af brugerinput går så hurtigt som muligt.

Data::writeUserInput1-Data::writeUserInput2
\lstinputlisting[linerange=Data::writeUserInput1-Data::writeUserInput2, label=lst:data_writeUserInput, caption=Metoden \texttt{writeUserInput} i dataklassen.]{../../src/bil/data/data.cpp}

\lstinputlisting[linerange=Data::getUserInput1-Data::getUserInput2, label=lst:data_getUserInput, caption=Metoden \texttt{geteUserInput} i dataklassen.]{../../src/bil/data/data.cpp}

\texttt{writeDistance} der anvendes til at indskrive en værdi fra én af distance-sensorerne i dataklassen er en smule anderledes end de ovenstående, da den skal have en parameter der angiver hvilken sensor der er tale om. Som det er beskrevet i ordlisten på side \pageref{sec:ordforklaring} omkring afstandssensorer, så har de hver især en betegnelse. For eksempel så er afstandssensoren forrest venstre angivet med ''FR''. Disse betegnelser er anvendt i implementeringen af metoden sammen med en mutex-beskyttelse, der tilsvarer den der også er anvendt i de øvrige \texttt{write} metoder. Metoden er vist i listing \ref{lst:data_wirteDistance}.

\lstinputlisting[linerange=Data::writeDistance1-Data::writeDistance2, label=lst:data_wirteDistance, caption=Metoden \texttt{writeDistance} i dataklassen..]{../../src/bil/data/data.cpp}

For alle de respektive \texttt{get} metoder, så er deres opsætning meget lig \texttt{write} metoderne, bortset fra at de returnerer den pågældende attribut. Det er derfor udeladt at eksemplificere dem alle ved kodeudsnit. Det er dog vigtigt at bide mærke i at \texttt{getLatestVelocity} anvender samme mutex som \texttt{writeVelocity} og vice versa. Listing \ref{lst:data_getLatestVelocity} viser at mutexen ved navn \texttt{sensorDataMut\_vel} anvendes her og i \texttt{writeVelocity}, vist i listing \ref{lst:data_wirteVelocity}.

\lstinputlisting[linerange=Data::getLatestVelocity1-Data::getLatestVelocity2, label=lst:data_getLatestVelocity, caption=Metoden \texttt{getLatestVelocity2} i dataklassen..]{../../src/bil/data/data.cpp}