\subsection{Dataklassen}

Dataklassen håndterer alt den data der skal kommunikeres imellem brugeren og brugerinterfacet på PCen og ud til sensorer og aktautorer på bilen, så længe de er opbevaret på Pi. Klassen/datastrukturen er implementeret ganske simpelt med variable frem for en fil, da dette vil gøre klassen hurtigere. Igennem koden i denne klasse er der indtil flere \texttt{\#ifdef DEBUG}, hvilket giver mulighed for at slå debugging til ved at definere \texttt{DEBUG} i toppen af klassen. \\
Constructoren, der er vist i listing \ref{lst:data_con}, er ganske simpel. Den initialiserer alle variable til ''0'' og gemmer en pointer til logfilen.

\lstinputlisting[linerange=Data::Data1-Data::Data2, label=lst:data_con, caption=Constructor for dataklassen.]{../../src/bil/data/data.cpp}

Klassen har en række \texttt{write} metoder, som alle er beskyttet af en mutex, der forhindrer at der er flere forskellige tråde der kan redigere i variablen på samme tid. Hvis dette skete ville det kunne skabe invalide data. Metoderne \texttt{writeVelocity} og \texttt{writeAcceleration} er helt ens på nær den attribut om dataen gemmes i. I listing \ref{lst:data_wirteVelocity} er \texttt{writeVelocity} vist.

\lstinputlisting[linerange=Data::writeVelocity1-Data::writeVelocity2, label=lst:data_wirteVelocity, caption=Metoden \texttt{writeVelocity} i dataklassen.]{../../src/bil/data/data.cpp}

For at opbevare brugerinput, anvendes der \texttt{write}(listing \ref{lst:data_writeUserInput}) og \texttt{get}(listing \ref{lst:data_getUserInput}) metoder der bruger ''copy by reference'' til at overføre en struct med dataen. Denne struct er defineret i \texttt{utilities.hpp}. Copy by reference, er anvendt da det er vigtigt at overførslen af brugerinput går så hurtigt som muligt.

Data::writeUserInput1-Data::writeUserInput2
\lstinputlisting[linerange=Data::writeUserInput1-Data::writeUserInput2, label=lst:data_writeUserInput, caption=Metoden \texttt{writeUserInput} i dataklassen.]{../../src/bil/data/data.cpp}

\lstinputlisting[linerange=Data::getUserInput1-Data::getUserInput2, label=lst:data_getUserInput, caption=Metoden \texttt{geteUserInput} i dataklassen.]{../../src/bil/data/data.cpp}

\texttt{writeDistance} der anvendes til at indskrive en værdi fra én af distance-sensorerne i dataklassen er en smule anderledes end de ovenstående, da den skal have en parameter der angiver hvilken sensor der er tale om. Som det er beskrevet i ordlisten på side \pageref{sec:ordforklaring} omkring afstandssensorer, så har de hver især en betegnelse. For eksempel så er afstandssensoren forrest venstre angivet med ''FR''. Disse betegnelser er anvendt i implementeringen af metoden sammen med en mutex-beskyttelse, der tilsvarer den der også er anvendt i de øvrige \texttt{write} metoder. Metoden er vist i listing \ref{lst:data_wirteDistance}.

\lstinputlisting[linerange=Data::writeDistance1-Data::writeDistance2, label=lst:data_wirteDistance, caption=Metoden \texttt{writeDistance} i dataklassen.]{../../src/bil/data/data.cpp}

For alle de respektive \texttt{get} metoder, så er deres opsætning meget lig \texttt{write} metoderne, bortset fra at de returnerer den pågældende attribut. Det er derfor udeladt at eksemplificere dem alle ved kodeudsnit. Det er dog vigtigt at bide mærke i at \texttt{getLatestVelocity} anvender samme mutex som \texttt{writeVelocity} og vice versa. Listing \ref{lst:data_getLatestVelocity} viser at mutexen ved navn \texttt{sensorDataMut\_vel} anvendes her og i \texttt{writeVelocity}, vist i listing \ref{lst:data_wirteVelocity}.

\lstinputlisting[linerange=Data::getLatestVelocity1-Data::getLatestVelocity2, label=lst:data_getLatestVelocity, caption=Metoden \texttt{getLatestVelocity2} i dataklassen.]{../../src/bil/data/data.cpp}

\clearpage

\subsubsection{Test}
Til at teste implementeringen af klassen løbende er det udarbejdet et testprogram \texttt{main.cpp}, som kalder de enkelte enkelte  \texttt{write} og \texttt{get} metoder. Programmet er udarbejdet så der er en række tråde der skriver til klassens værdier og at der (næsten) samtidig er en række tråde der læser fra dataklassen. Der er oprettet en tråd type der skriver i værdierne, kaldet \texttt{writers}, som genererer tilfældige tal og indskriver dem i dataklassen. Når der er indskrevet et tal, så skrives der i loggen hvilke værdier der er indskrevet. Koden for denne implementering er vist i listing \ref{lst:data_test_writers}.

\lstinputlisting[linerange=writers1-writers2, label=lst:data_test_writers, caption=\texttt{writer}-tråde i testprogram.]{../../src/bil/data/main.cpp}

Tilsvarende er der selvfølgelig udarbejdet en trådtype der læser fra dataklassen. Denne klasse skriver ligeledes de data som den læser ud i logfilen. Denne trådtype er navngivet \texttt{reader}, og er vist i listing \ref{lst:data_test_readers}

\lstinputlisting[linerange=reader1-reader2, label=lst:data_test_readers, caption=\texttt{reader}-tråde i testprogram.]{../../src/bil/data/main.cpp}

Ovenstående gør det muligt at se om der skulle være opstået nogle korrupte data iblandt de udlæste data. Korrupt data kunne fremkomme hvis mutex-låsene i dataklassen ikke fungerer efter hensigten og at der udlæses data samtidig med at der er ved at blive skrevet i dem. Dette vil medføre at der skrives en værdi i logfilen fra en \texttt{reader}-tråd som ikke tidligere er blevet indlæst i dataklassen af en \texttt{writer}-tråd. I listing \ref{lst:data_logfile} ses en logfil der er resultatet af at køre testprogrammet.

\lstinputlisting[language=,label=lst:data_logfile, caption=Den resulterede logfil når testprogrammet køres.]{swimplementering/bil/data/pi-log.txt}

I logfilen ses det at der ikke er sket nogle fejl af den art som er beskrevet. For at få et mere pålideligt testresultat, er det en fordel at køre programmet flere gange elle med flere tråde, hvilket selvfølgelig også er gjort. Testprogrammets \texttt{main}-funktion er vist i listing \ref{lst:data_test_main}.

\lstinputlisting[linerange=main1-main2, label=lst:data_test_main, caption=\texttt{main}-funktion i testprogram.]{../../src/bil/data/main.cpp}