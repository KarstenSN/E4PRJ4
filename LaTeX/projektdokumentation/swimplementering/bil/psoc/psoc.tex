%SW-implementering af Psoc-klassen (pi)
\subsection{Psocklassen (pi)}

\texttt{Psoc}-klassen håndterer alt kommunikation mellem Pi og PSoC'en på Pi-siden. Som førnævnt er dette implementeret via \IIC-kommunikation. Klassen er implementeret med to public metoder: \texttt{getDistance(name)} og \texttt{getVelocity()}. Disse metoder kaldes fra \texttt{Data}-klassen når afstandsdata eller hastighedsdata ønskes overført. Disse 2 metoder tilgår de midlertidige variabler i klassen. På denne måde holdes et enkelt interface udadtil. Derudover er klassen bygget op omkring metoderne \texttt{update()} og \texttt{psocRead()}. I listing \ref{lst:Psoc_hpp} ses implementering af klassens headerfilen.

\lstinputlisting[linerange=psoc::header1-psoc::header2, 
	label=lst:Psoc_hpp, 
	caption=headerfil for Psocklassen.]{../../src/bil/inc/Psoc.hpp}
	
	
\texttt{psocThread\_} benyttes her til at lade kassen kører i sin egen tråd parallelt med resten af pi-programmet. \texttt{psocMut} benyttes sammen \texttt{lock\_guard} til at beskyttet klassens attributter så de kun kan tilgås af én tråd af gangen, enten kan der skrives til dem, ellers kan læses fra dem, men aldrig på samme tid. \texttt{logPtr\_} benyttes til at skrive til Log-klassen i den globale Log for systemet.


Metoden \texttt{getDistance} sørger for først at låse attributter, FL,FR,RL,RR med \texttt{mutex} inden de tilgås. Herefter returnerer metoden dén sensorværdi hvis navn er sendt med som parameter. Slutteligt sørger \texttt{std::lock\_guard} selv for at frigive låsen igen når metoden går ud af scope. I listing \ref{lst:Psoc_getdistance} ses implementering af \texttt{getDistance()}-metoden. Yderligere ses implementeret debugging feature.

\lstinputlisting[linerange=psoc::getDistance1-psoc::getDistance2, 
	label=lst:Psoc_getdistance, 
	caption=Implementering af \texttt{getDistance()}]{../../src/bil/Psoc.cpp}
	

Metoden \texttt{getVelocity()} sørger for først at låse attributten \texttt{Tacho\_} inden denne returneres. Ophævelse af låsen sker automatisk når metoden går ud af scope. Implementering af \texttt{getVelocity} ses i listing \ref{lst:Psoc_getvelocity}:

\lstinputlisting[linerange=psoc::getVelocity1-psoc::getVelocity2, 
	label=lst:Psoc_getvelocity, 
	caption=Implementering af \texttt{getVelocity()}]{../../src/bil/Psoc.cpp}


Metoden \texttt{update()} åbner \IIC-fobindelsen til PSoC'en ved at åbne $/dev/i2c-1$ som O\_RDWR, så der både kan skrive til, og læses fra forbindelsen.herefter sættes slaveadressen til PSoC'ens værdi på \texttt{0x10}, og nu kan der hentes de seneste målinger fra \texttt{rdbuffer} på PSoC'en. Som det sidste nedlægges forbindelsen med \texttt{close(fd)}. I listing \ref{lst:Psoc_update} ses implementeringen af \texttt{update()}.

\lstinputlisting[linerange=psoc::update1-psoc::update2, 
	label=lst:Psoc_update, 
	caption=Implementering af \texttt{update()}]{../../src/bil/Psoc.cpp}

Metoden \texttt{psocRead()} sørger for at opdaterer klassens attributter med det seneste datasæt fra PSoC'en. 
Først kaldes \texttt{update()} så de seneste målinger lægger i \texttt{rdBuffer}. 
\texttt{rdBuffer} er det array der samler datasættet fra PSoC'en, således at data til \texttt{distanceFL\_} lægges i \texttt{rdBuffer[0] og rdBuffer[1]}. 
Ligeledes ligger data til \texttt{distanceFR\_} i \texttt{rdBuffer[2] og rdBuffer[3]} osv. Data til \texttt{Tacho\_} ligger i \texttt{rdBuffer[8]}. Herefter låses attributterne således kan der skrives til dem uden at det clasher med en samtidig læsning. 
Datasættet fra \texttt{rdbuffer} lægges nu over i attributterne som beskrevet ovenfor. tilsidst ventes 200ms, for at give sensorerne den anbefalede tid til at fortage en range-reading. 
Implementering af \texttt{psocRead()} ses i listing \ref{lst:Psoc_psocread}.

\lstinputlisting[linerange=psoc::read1-psoc::read2, 
	label=lst:Psoc_psocread, 
	caption=Implementering af \texttt{psocRead	()}]{../../src/bil/Psoc.cpp}

Constructoren for klassen sørger for at initiere pointer til Log-klassen, og der sendes en event-msg om at klassen er oprettet. Herefter startes den selvstændige tråd. implementering af Constructoren kan ses i listing \ref{lst:Psoc_ctor}:

\lstinputlisting[linerange=psoc::ctor1-psoc::ctor2, 
	label=lst:Psoc_ctor, 
	caption=Implementering af \texttt{constructor}]{../../src/bil/Psoc.cpp}


klassens destructor sørger for at sende en event-msg om at klassen er lukkes ned. Herefter joines den selvstændige tråd. dette gøres for at tråden afsluttes korrekt, ellers vil programmet tvinges til exit. Destructoren er implementeret som vist i listing \ref{lst:Psoc_dtor}:

\lstinputlisting[linerange=psoc::dtor1-psoc::dtor2, 
	label=lst:Psoc_dtor, 
	caption=Implementering af \texttt{destructor}]{../../src/bil/Psoc.cpp}