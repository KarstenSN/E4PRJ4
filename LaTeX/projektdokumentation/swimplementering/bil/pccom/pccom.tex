\subsubsection{PcCom} \label{sec:pccom_impl}

PcCom er en klasse, som styrer kommunikation mellem PC software og bil, fra bilens side. Klassens primære opgave er, at igangsætte to threads der hver især styrer en TCP kommunikation mellem PIen og softwaren på computeren. Den ene thread modtager og vidersender brugerinput fra brugeren til AKS og Steering objekter. Den anden thread modtager indstillinger og sætter disse i et Settings objekt, samt henter data fra et Data objekt og sender disse tilbage til softwaren på computeren.
Herunder vil koden blive forklaret i detaljer.


\paragraph{PcCom Constructor} er klassens constructor. Koden er vist i vist i Listing \ref{lst:pccom_con}. Ud over at initialisere, sker der en linkning til et Log, et Data og et Settings object (linje 15, 16 og 17), via association. Denne association sørger for at det er de samme objekter, der bliver bearbejdet af de andre klasser i systemet. På linje 24 og 25 igangsættes de førnævnte threads med hver deres funktionskald, hhv. dataStream og controllerStream. På linje 27 skriver den til Log objektet at den er færdig med initiaisering.

\lstinputlisting[linerange=PcCom::PcCom1-PcCom::PcCom2, label=lst:pccom_con, caption=Constructor for PcCom klassen.]{../../src/bil/pccom.cpp}

\paragraph{PcCom Destructor} er klassens destructor. Denne skal sørge for at vente med at lukke klassen ned til at de to threads er klar til dette. Koden er vist i vist i Listing \ref{lst:pccom_de}

\lstinputlisting[linerange=PcCom::PcCom3-PcCom::PcCom4, label=lst:pccom_de, caption=Destructor for PcCom klassen.]{../../src/bil/pccom.cpp}

\clearpage

\paragraph{controllerStream} er en funktion der igangsættes af tråden \texttt{controllerStreamTh}. Formålet med denne funktion er at oprette en TCP connection til streaming af brugerinput.\\
I Listing \ref{lst:pccom_controllerstream} vises initieringen af den ønskede socket. 

\lstinputlisting[linerange=PcCom::controllerStream1-PcCom::controllerStream2, label=lst:pccom_controllerstream, caption=Del 1 af funktionen \texttt{controllerStream}: Initieringen.]{../../src/bil/pccom.cpp}
%
%
%
%\paragraph{setAKS} er en funktion til at skrive til klassens attribut, der indeholder AKS status. Koden er vist i Listing \ref{lst:setaks}. Denne funktion er implementeret på samme måde som setMaxSpeed, beskrevet tidligere, og vil derfor ikke blive gennemgået i detaljer her.
%
%\begin{lstlisting}[caption={Funktionen setAKS},label=lst:setaks, language=c++]
%void Settings::setAKS(bool AKSStatus)
%{
%  std::lock_guard<std::mutex> lock(this->AKSStatusMut);
%  this->AKSStatus_ = AKSStatus;
%  return;
%}
%\end{lstlisting}
%
%
%\paragraph{calibrateSteering} er en funktion til at skrive til klassens attribut, der indeholder kalibrering af styrtøj. Koden er vist i Listing \ref{lst:calibratesteering}. Denne funktion er implementeret på samme måde som setMaxSpeed og setAKS, beskrevet tidligere, og vil derfor ikke blive gennemgået i detaljer her.
%
%\begin{lstlisting}[caption={Funktionen calibrateSteering},label=lst:calibratesteering, language=c++]
%void Settings::calibrateSteering(char calibration)
%{
%  std::lock_guard<std::mutex> lock(this->calibrationMut);
%  this->calibration_ = calibration;
%  return;
%}
%\end{lstlisting}
%
%
%\paragraph{getMaxSpeed}  er en funktion til at læse og returnere klassens attribut, der indeholder den indstillede maksimale hastighed. Koden er vist i Listing \ref{lst:getmaxspeed}. Ud over at læse attributen låses klassens attribut, så det ikke er muligt for andre funktionskald at skrive eller læse attributen, indtil at funktionen er færdig med at læse fra og returnere den.
%
%\begin{lstlisting}[caption={Funktionen getMaxSpeed},label=lst:getmaxspeed, language=c++]
%int Settings::getMaxSpeed()
%{
%  std::lock_guard<std::mutex> lock(this->maxVelocityMut);
%  return this->maxVelocity_;
%}
%\end{lstlisting}
%
%
%\paragraph{getAKS}  er en funktion til at læse og returnere klassens attribut, der indeholder AKS status. Koden er vist i Listing \ref{lst:getaks}. Denne funktion er implementeret på samme måde som getMaxSpeed, beskrevet tidligere, og vil derfor ikke blive gennemgået i detaljer her.
%
%\begin{lstlisting}[caption={Funktionen getAKS},label=lst:getaks, language=c++]
%bool Settings::getAKS()
%{
%  std::lock_guard<std::mutex> lock(this->AKSStatusMut);
%  return this->AKSStatus_;
%}
%\end{lstlisting}
%
%
%\paragraph{getCalibration}  er en funktion til at læse og returnere klassens attribut, der indeholder kalibrering af styrtøj. Koden er vist i Listing \ref{lst:getcalibration}. Denne funktion er implementeret på samme måde som getMaxSpeed og getAKS, beskrevet tidligere, og vil derfor ikke blive gennemgået i detaljer her.
%
%\begin{lstlisting}[caption={Funktionen getCalibration},label=lst:getcalibration, language=c++]
%char Settings::getCalibration()
%{
%  std::lock_guard<std::mutex> lock(this->calibrationMut);
%  return this->calibration_;
%}
%\end{lstlisting}