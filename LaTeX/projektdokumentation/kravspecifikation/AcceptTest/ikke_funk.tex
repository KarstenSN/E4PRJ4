\begin{longtable}{| l | L{4cm} | Z | Z | L{3 cm} |} \hline
Krav & Test & Forventet Resultat & Resultat & Godkendt/ kommentar \\ \hline
    1. & Start drivhuset med monitorering, noter hvornår værdier bliver tilføjet til systemloggen. Varighed 2 min. & Data loggen genererer 1 datapunkt hver 10. sekund. & Dataloggen viser kun et enkelt datapunkt. & Ikke godkendt. Ikke færdigimplementeret. \\ \hline
    2. & Det fysiske drivhus placeres i et rum ved 25 +/- 1 grader celcius, og opvarmes vha. et varmelegeme til minimum 30 grader celcius. I det virtuelle drivhus sættes en ønsket gennemsnitstemperatur på 25 grader celcius og ventilator er aktiveret. & Inden der er gået 30 min. aflæses temperaturen i drivhuset til 25 +/- 1 grader celcius. & Temperaturen er reguleret ned inden for temperaturtolerancen efter 4 min og 14 sekunder. & Godkendt. \\ \hline
	3. & En potte med tør muld indsættes i det fysiske drivhus. En fugtighedssensor places i mulden og vand hældes langsomt i. & Systemloggen skriver 11 dataværdier med stigende fugtighed og den 11. data værdi er ækvivalent med den 10. værdi. & Der vises ikke måleværdier. & Delvis godkendt. Systemet kan vise jordfugt i 10 trin, men det er ikke muligt at gennemføre testen som beskrevet, da mange forskellige faktorer påvirker dem. \\ \hline
	4. & Seks fugtighedsmålere tilsluttes systemet, hvorefter systemet startes. & Systemloggen indeholder måleværdier for 6 forskellige sensorer efter 1 min. & Der vises ikke måleværdier. & Delvist godkendt. Sensorerne giver korrekte værdier, men systemloggen er ikke færdigimplementeret. \\ \hline
    5. & 100 planter indsættes i plantedatabasen. & Databasen kontrolleres for alle 100 planters eksistens. & Plantedatabasen er ikke implementeret. & Ikke godkendt. Ikke implementeret. \\ \hline
    6. & Intervallet for datalogging sættes ned for at simulere et års data. & Historik kan ses med et års data. & Historik er ikke implementeret. & Ikke godkendt. Ikke implementeret. \\ \hline
       7. & Det fysiske drivhus placeres i et rum ved 25 +/- 1 grader celcius. Systemet sættes til at regulere temperaturen i det fysiske drivhus til 30 grader celcius. Et eksternt termometer med en usikkerhed på højst 0.1 grader celcius placeres ved siden af temperatursensoren. & Det eksterne termometer måler 30 +/- 1 grad celcius inden for 30 min. & Systemet opvarmer drivhuset til 30 grader efter 2 minutter og 30 sekunder, kommer ikke over 31 grader og tænder igen for varmelegemet efter 7 minutter og 2 sekunder. Derfor betragtes temperaturen som stabil. & Godkendt. Koden tager ikke højde for offset på temperatursensor, derfor anvendes ekstern termometer ikke i testen. \\ \hline
    8. & Der indtastets tre gyldige E-mail adresser via brugerfladens "E-mailmenu". Daglig E-mail notifikation aktiveres. Testpersonen kontrollerer de tre indtastede E-mailkontos indbakker næste gang klokken har passeret 12.00 & Testpersonen har modtaget en E-mail fra systemet på hver af de tre indtastede E-mailadresser.& E-mail er ikke implementeret. & Ikke godkendt. Ikke implementeret. \\ \hline
    9. & En potte med tør muld indsættes i det fysiske drivhus. En fugtighedssensor placeres i mulden og rapporting og Advarsels E-mail er aktiveret. En sensor med ønsket værdi for fugtighed på 10 placeres i mulden. & Før 10 min er forløbet, har brugeren modtaget en Advarsels E-mail. &  E-mail er ikke implementeret. & Ikke godkendt. Ikke implementeret. \\ \hline
\caption{Ikke funktionelle krav}
\label{tbl:ikkefunk}
\end{longtable}
