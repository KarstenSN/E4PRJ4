\chapter{Kravspecifikation} \label{ch:kravspecifikation}

%---------------------------------------------------------------------------------------
%											SYSTEMOVERSIGT
%---------------------------------------------------------------------------------------


\section{Systemoversigt} \label{sec:systemoversigt}
På figur \ref{fig:systemoversigt} ses den overordnede systemoversigt med kommunikationsveje og mekaniske forbindelser. Diagrammet skal give læseren et hurtigt overblik over det samlede system. I afsnittet %TODO henvisning
beskrives blokke og kommunikationsveje mere detaljeret. Under figur \ref{fig:systemoversigt} er blokkene kort beskrevet. 

\begin{figure}[h]
\centering
\includegraphics[scale=0.9]{../fig/diagrammer/systemoversigt.pdf}
\caption{Overordnet systemoversigt}
\label{fig:systemoversigt}
\end{figure}
\clearpage

\subsubsection{Pi}
Systemets kerne er et Raspberry Pi 2 board.
Pi'en står for at processere data fra afstandsensorene, og håndtere streaming af video. Derudover afvikles regulering til motor, samt styring af servo også fra Pi'en. 

\subsubsection{Servomotor}
Servomotor har til opgave at omsætte signal fra Pi'en til mekanisk styring af bilens forhjul. 

\subsubsection{Afstandssensor}
Bilens 2 fremadrettet og 2 bagudrettet afstandssensorer har til formål at indsamle data om eventuelle forhindringer i bilen kørebane. 

\subsubsection{Accelerometer}
Der er ligeledes påmonteret et accelerometer der anvendes til regulering af hastighed.

\subsubsection{Kamera}
Bilens kamera streamer video til PC'ens skærm så Bruger har mulighed for at navigere på baggrund af visuel feedback

\subsubsection{PC}
PC afvikler den software hvorigennem bilen kontrolleres, konfigureres og kalibreres. Det er ligeledes via computeren at Bruger får visuel feedback fra bilens kamera. 

\subsubsection{Xbox-360 Controller}
Til at kontrollere bilen, benyttes en Xbox-360 controller. vha. en række trykknapper og styrepinde kan bilens hastighed, såvel som retning bestemmes. 

\subsubsection{Motor}
Motoren omsætter data, herunder regulering fra Pi'en til mekanisk styring af bilens hastighed.

\subsubsection{Tachometer}
Motorens omdrejningshastighed kan via tachometeret aflæses og herefter benyttes til databehandling og regulering. 
\clearpage

%---------------------------------------------------------------------------------------
%											AKTØR-KONTEKSTDIAGRAM
%---------------------------------------------------------------------------------------

\section{Aktør-kontekstdiagram} \label{sec:aktor-kontekstdiagram}
På figur \ref{fig:aktor_kontekst} ses aktørkontekstdiagram over systemet. 

\begin{figure}[h]
\centering
\includegraphics[scale=1.1]{../fig/diagrammer/ac_au2.pdf}
\caption{Aktør kontekst diagram for AU2.}
\label{fig:aktor_kontekst}
\end{figure}

%---------------------------------------------------------------------------------------
%											AKTØRBESKRIVELSER
%---------------------------------------------------------------------------------------

\section{Aktørbeskrivelser} \label{sec:aktorbeskrivelser}
Som figur \ref{fig:aktor_kontekst} viser, er der 2 aktører til systemet.
\textit{Bruger} og \textit{Forhindring}.

\subsubsection{Bruger - Primær Aktør}
Bruger kan:
\begin{itemize}
	\item Starte og stoppe systemet 
	\item Styre bilen over et netværk.
	\item Konfigure og kalibrere system.
\end{itemize}

\subsubsection{Forhindring - Sekundær Aktør}
\textit{Forhindring} er objekter i det miljø bilen kører i, og som der dermed er risiko for at bilen kan kollidere med.  
\clearpage

%---------------------------------------------------------------------------------------
%									FUNKTIONELLE KRAV
%---------------------------------------------------------------------------------------

\section{Funktionelle krav} \label{sec:funktionelle_krav}
Systemet\ldots
\begin{enumerate}\itemsep1pt \parskip0pt \parsep0pt
	\item \ldots  \emph{Skal} kunne køre frem og tilbage.
	\item \ldots  \emph{Skal} kunne dreje.
	\item \ldots  \emph{Skal} kunne regulere hastigheden på bilen.
	\item \ldots  \emph{Skal} give Bruger mulighed for at begrænse maksimumshastighed.
	\item \ldots  \emph{Skal} give Bruger mulighed for manuel styring via Xbox-360 controller af hastighed og retning.
	\item \ldots  \emph{Skal} via Wi-Fi netværk kunne kommunikere mellem bil og PC.
	\item \ldots  \emph{Skal} kunne identificere forhindringer foran og bag bilen.
	\item \ldots  \emph{Skal} indeholde et anti-kollisionssystem baseret på afstandssensorer.
	\item \ldots  \emph{Skal} via. anti-kollisionssystem kunne undvige og/eller stoppe før kollision.
	\item \ldots  \emph{Skal} indeholde et kamera til at streame video.
	\item \ldots  \emph{Bør} give Bruger mulighed for at aktivere/deaktivere anti-kollisionssystemet på bilen.
	\item \ldots  \emph{Kan} indeholde en batteriniveau-indikator.
\end{enumerate}

%---------------------------------------------------------------------------------------
%											IKKE-FUNKTIONELLE KRAV
%---------------------------------------------------------------------------------------

\section{Ikke-funktionelle krav} \label{sec:ikke-funktionelle_krav}
\begin{enumerate}\itemsep1pt \parskip0pt \parsep0pt
	\item Bilens maksimumshastighed uden begrænsning er 10km/t $\pm$ 1km/t %TODO Passer denne?
	\item Bilens bremselængde ved maksimumshastighed uden begrænsning må ikke overstige 1 m. %TODO Passer denne?
	\item Bilen skal kunne accelerere fra 0 km/t til maksimumshastighed uden begrænsning på højest 6 s. %TODO Passer denne?
	\item Forsinkelse fra brugerinput til at bilen reagerer må ikke overstige 50ms. %TODO Passer denne?
	\item Afstandssensorerne skal kunne identificere en forhindring på mindst $(30cm \times 30cm)$ vinkelret mod bilen, på en maksimal afstand af 6 m. På afstande over 6 m er det ikke et krav at systemet kan detektere forhindringen. %TODO Passer afstand
	\item Mister bilen forbindelsen med PC i mere end 50ms, standser bilen automatisk. 
	\item Kameraet skal minimum have en opdateringshastighed på 15 billeder i sekundet. %TODO undersøg!
	\item Systemet skal vise video-stream med en opløsning på $640 \times 480$ pixels i hovedvinduet.
	\item PC skal som minimum sende kommandoer til bilen 60 gange i sekundet. 
	\item HID skal bestå af en Xbox-360 controller, tastatur og mus.
\end{enumerate}
\clearpage

%--------------------------------------------------------------------------------------
%												USE CASES
%--------------------------------------------------------------------------------------
\section{Use Cases}
På figur \ref{fig:UC_au2} ses use case diagram over de funktionelle krav. 

\begin{figure}[h]
\centering
\includegraphics[width=\textwidth - 1 cm]{../fig/diagrammer/uc_au2.pdf}
\caption{Use case diagram for AU2.}
\label{fig:UC_au2}
\end{figure}
\clearpage

%----------------------------------------------------------------------------------------
%								Use Case beskrivelser - Initiering og Formål
%----------------------------------------------------------------------------------------

\subsection{Use Case beskrivelser - Initiering og Formål} 
\subsubsection{UC1: Aktiver system}
Initieres af: Bruger

Denne UC giver Bruger mulighed for at aktivere systemet. Herunder konfigureres bilen, UC2 + UC3 initieres og PC'en viser hovedvinduet. 

\subsubsection{UC2: Stream Video}
Initieres af: UC1: Aktiver system

Denne UC initierer videostream fra kameraet, og forbindelsen over Wi-Fi netværket oprettes.

\subsubsection{UC3: Overvåg sensorer}
Initieres af: UC1: Aktiver system

Denne UC initierer overvågning af bilens sensorer, herunder, de 4 afstandssensorer, tachometer, samt accelerometer. Use casen kører kontinuerligt og henter løbende data fra sensorerne.

\subsubsection{UC4: Anvend AKS}
Initieres af: UC3: Overvåg sensorer

Denne UC har til formål at lade AKS overtage styring af bilen under kørsel hvis en forhindring detekteres enten foran eller bagved bilen.
Når forhindringen er undveget overgives styringen igen til Bruger.

\subsubsection{UC5: Kør bil frem/tilbage}
Initieres af: Bruger

Denne UC har til formål at give Bruger mulighed for at ændre hastighed på bilen via  de trykfølsomme ''LT'' og ''RT''-knapper på Xbox-360 controlleren. Bruger trykker på ''LT'' og bilen kører fremad, eller Bruger trykker på ''RT'' og bilen bakker.

\subsubsection{UC6: Drej bil til højre/venstre}
Initieres af: Bruger

Denne UC har til formål at lade Bruger ændre bilens retning. Bruger benytter venstre styrepind på Xbox-360 controlleren. Føres styrepinden til venstre, drejer bilen til venstre. Føres styrepinden til højre, drejer bilen ligeledes til højre.

\subsubsection{UC7: Brems bil}
Initieres af: Bruger

Denne UC har til formål at lade Bruger sænke bilens hastighed. Bruger trykker ''X'' på Xbox-360 controlleren, jo længere tid knappen holdes nede jo mere sænkes bilens hastighed.

\subsubsection{UC8: Konfigurer IP-adresse}
Initieres af: Bruger

Denne UC har til formål at lade Bruger konfigurere PC'ens IP-adresse således at der kan opnås forbindelse til bilen.  

\subsubsection{UC9: Tænd/sluk AKS}
Initieres af: Bruger

Denne UC har til formål at give Bruger mulighed for at vælge om AKS skal være tændt eller slukket. Bruger kan via ''Hovedvindue'' på PC'en vælge status for AKS. 

\subsubsection{UC10: Indstil makshastighed}
Initieres af: Bruger

Denne UC har til formål at give Bruger mulighed for at indstille en maksimumhastighed på bilen. Hastigheden indstilles via PC'ens ''Hovedvindue''.

\subsubsection{UC11: Kalibrer styretøj}
Initieres af: Bruger

Denne UC har til formål at give Bruger mulighed for at kalibrere bilens styretøj. Bruger indtaster via menuen ''Kalibrer styretøj'' en centerværdi i intervallet -50 til 50.  

\subsubsection{UC12: Afbryd system}
Initieres af: Bruger

Denne UC har til formål at lade Bruger afbryde hele systemet. Bruger afslutter software på PC, og sætte bilen ''ON/OFF''-knap til ''OFF'' for at afbryde forbindelse til batteriet. 
\clearpage

%----------------------------------------------------------------------------------------
%									Use Case fully dressed
%----------------------------------------------------------------------------------------

% UC1:  Aktiver system
\begin{longtable}{| l | >{\raggedright}X | >{\raggedright}X | >{\raggedright}X | >{\raggedright\arraybackslash}p{2.3cm} |} \hline
	\multicolumn{2}{|l|}{\textbf{Use case under test}} & 
	\multicolumn{3}{l|}{UC1: Aktiver system} \\ \hline
	
	\multicolumn{2}{|l|}{\textbf{Scenarie}} & 
	\multicolumn{3}{l|}{Hovedscenarie} \\ \hline
	
	\multicolumn{2}{|l|}{\textbf{Forudsætning}} & 
	\multicolumn{3}{p{10.2cm}|}{Netværksforbindelse er opsat og fungerende\hfill} \\ \hline
	%\multicolumn{5}{|l|}{}\\ \hline
	\textbf{Step} & \textbf{Handling} & \textbf{Forventet Resultat} & \textbf{Resultat} & \textbf{Godkendt / Kommentar} \\ \hline

	1.1 & Bruger sætter bilens ''ON/OFF''-switch til ''ON''. 
		& Vsuel test:\\ Lampe på strømforsyning lyser.
		& 
		& \\ \hline
		
	1.2 & Bruger starter software på PC.
		& Visuel test:\\ Hovedvinduet fremkommer på skærmen.
		& 
		& \\ \hline
		
	1.3 & Bruger trykker på ''Opret forbindelse''.
		& Visuel test:\\ Hovedvindue viser ''Forbindelse oprettet''.
		& 
		& \\ \hline
		
	1.4 & Bruger observerer hovedvinduet.
		& Visuel test:\\ Videostream fremkommer i hovedvinduet.
		& 
		& \\ \hline
		
	1.5 & Bruger observerer hovedvinduet.
		& Visuel test:\\ Bilens aktuelle hastighed fremkommer i hovedvinduet.
		& 
		& \\ \hline
		
	1.6 & Bruger observerer hovedvinduet.
		& Visuel test:\\ Bilens aktuelle g-påvirkning fremkommer i hovedvinduet.
		& 
		& \\ \hline
		
	1.7 & Bruger observerer hovedvinduet.
		& Visuel test:\\ Data fra bilens afstandssensorer fremkommer i hovedvinduet.
		& 
		& \\ \hline
		
\caption{Accepttest for UC1: Aktiver system}\label{tbl:acceptuc1}
\end{longtable}
\clearpage

% UC2: Stream Videofeed
\begin{longtable}{| l | >{\raggedright}X | >{\raggedright}X | >{\raggedright}X | >{\raggedright\arraybackslash}p{2.3cm} |} \hline
	\multicolumn{2}{|l|}{\textbf{Use case under test}} & 
	\multicolumn{3}{l|}{UC2: Stream Video} \\ \hline
	
	\multicolumn{2}{|l|}{\textbf{Scenarie}} & 
	\multicolumn{3}{l|}{Hovedscenarie} \\ \hline
	
	\multicolumn{2}{|l|}{\textbf{Forudsætning}} & 
	\multicolumn{3}{p{10.2cm}|}{UC1 frem til punkt 5 er fuldført \hfill} \\ \hline
	%\multicolumn{5}{|l|}{}\\ \hline
	\textbf{Step} & \textbf{Handling} & \textbf{Forventet Resultat} & \textbf{Resultat} & \textbf{Godkendt / Kommentar} \\ \hline

	2.1 & Bruger har Wireshark åbent på samme computer. Wireshark er opsat til at overvåge det pågældene netværk.
		& Visuel test:\\ I Wireshark observeres der for overføring af pakker fra bilens IP-adresse til computerens IP-adresse.
		& 
		& \\ \hline
		
\caption{Accepttest for UC2: Stream Video}\label{tbl:acceptuc2}
\end{longtable}
\clearpage

% UC3: Overvåg sensor
\begin{longtable}{| l | >{\raggedright}X | >{\raggedright}X | >{\raggedright}X | >{\raggedright\arraybackslash}p{2.3cm} |} \hline
	\multicolumn{2}{|l|}{\textbf{Use case under test}} & 
	\multicolumn{3}{l|}{UC3: Overvåg sensor} \\ \hline
	
	\multicolumn{2}{|l|}{\textbf{Scenarie}} & 
	\multicolumn{3}{l|}{Hovedscenarie} \\ \hline
	
	\multicolumn{2}{|l|}{\textbf{Forudsætning}} & 
	\multicolumn{3}{p{10.2cm}|}{UC1 frem til punkt 6 er fuldført \hfill} \\ \hline
	%\multicolumn{5}{|l|}{}\\ \hline
	\textbf{Step} & \textbf{Handling} & \textbf{Forventet Resultat} & \textbf{Resultat} & \textbf{Godkendt / Kommentar} \\ \hline

	3.1 & Bruger åbner programmet PUTTY.EXE og indtaster ssh -l pi IP\_ADRESS -p 22. 
		& Visuel test:\\ Terminalen spørger om et password''.
		& 
		& \\ \hline

	3.2 & Bruger indtaster 1234.
		& Visuel test:\\ Terminalen viser pi@raspberry \$.
		& 
		& \\ \hline
		
	3.3 & Bruger indtaster nano /etc/var/log/au2log %TODO opdater filnavnet til det rigtige
		& Visuel test:\\ Log filen viser Accelerometer initialisering.. Done. Tachometer initialisering.. Done. Distancesensors initialisering.. Done.
		& 
		& \\ \hline
		
	3.4 & Bruger kører en tur med bilen og observerer hovedmenu i softwaren på PC.
		& Visuel test:\\ Bruger observerer at data for bilens hastighed, afstand til forhindring og acceleration fremgår af brugerfladen.
		& 
		& \\ \hline

\caption{Accepttest for UC3: Overvåg sensor}\label{tbl:acceptuc3}
\end{longtable}
\clearpage

% UC4: Anvend AKS

\begin{longtable}{| l | >{\raggedright}X | >{\raggedright}X | >{\raggedright}X | >{\raggedright\arraybackslash}p{2.3cm} |} \hline
	\multicolumn{2}{|l|}{\textbf{Use case under test}} & \multicolumn{3}{l|}{UC4: Undvig forhindring} \\ \hline
	\multicolumn{2}{|l|}{\textbf{Scenarie}} & \multicolumn{3}{l|}{Hovedscenarie} \\ \hline
	\multicolumn{2}{|l|}{\textbf{Forudsætning}} & \multicolumn{3}{p{10.2cm}|}{UC1 er gennemført, UC3 er gennemført.\hfill} \\ \hline
	%\multicolumn{5}{|l|}{}\\ \hline
	\textbf{Step} & \textbf{Handling} & \textbf{Forventet Resultat} & \textbf{Resultat} & \textbf{Godkendt / Kommentar} \\ \hline
	
	4.1 & Bruger styrer bilen fremad mod en forhindring på $(30cm \times 30cm)$ vinkelret på bilens kørebane vha. Xbox-360 controlleren, således at bilen er umiddelbart til venstre for objektet. & Visuel test: \\ Bruger observerer at bilen ændrer kurs til højre på trods af brugerinput. & ~ & ~ \\ \hline
	
	4.2 & Bruger tester om det er muligt at styre bilen igen med Xbox-360 controlleren. & Visuel test: \\ Bruger observerer at bilen reagerer på brugerinput. & ~  & ~ \\ \hline
	
	4.3 & Bruger styrer bilen fremad mod en forhindring på $(30cm \times 30cm)$ vinkelret på bilens kørebane vha. Xbox-360 controlleren, således at bilen er umiddelbart til højre for objektet. & Visuel test: \\ Bruger observerer at bilen ændrer kurs til venstre på trods af brugerinput. & ~ & ~\\ \hline
	
	4.4 & Bruger styrer bilen fremad mod en forhindring på $(30cm \times 30cm)$ vinkelret på bilens kørebane vha. Xbox-360 controlleren, således at bilen har retning lige mod objektet. & Visuel test: \\ Bruger observerer at bilen standser på trods af brugerinput. & ~ & ~ \\\hline
	
	4.5 & Bruger bakker mod en forhindring på $(30cm \times 30cm)$ vinkelret på bilens kørebane vha. Xbox-360 controlleren, således at bilen er umiddelbart til venstre for objektet. & Visuel test: \\ Bruger observerer at bilen ændrer kurs til venstre på trods af brugerinput. & ~ & ~ \\ \hline
	
	4.6 & Bruger bakker bilen mod en forhindring på $(30cm \times 30cm)$ vinkelret på bilens kørebane vha. Xbox-360 controlleren, således at bilen er umiddelbart til højre for objektet. & Visuel test: \\ Bruger observerer at bilen ændrer kurs til højre på trods af brugerinput. & ~ & ~\\ \hline
	
	4.7 & Bruger bakker bilen mod en forhindring på $(30cm \times 30cm)$ vinkelret på bilens kørebane vha. Xbox-360 controlleren, således at bilen har retning lige mod objektet. & Visuel test: \\ Bruger observerer at bilen standser på trods af brugerinput. & ~ & ~\\\hline

\caption{Accepttest for UC4: Undvig forhindring}\label{tbl:acceptuc4}
\end{longtable}
\clearpage

% UC5: Kør bil frem/tilbage
\subsection{Use Case 5: Kør bil frem/tilbage}
\begin{table}[h]
\begin{tabularx}{\textwidth}{| L{3.3 cm} | Z |} \hline

\textbf{Navn:} 						& UC5: Kør bil frem/tilbage\\ \hline
\textbf{Mål:}						& At få bilen til at køre frem eller tilbage. \\ \hline
\textbf{Initiering:}				& Bruger \\ \hline
\textbf{Aktører:} 					& Bruger \\ \hline
\textbf{Reference:} 				& Ingen \\ \hline
\textbf{Antal samtidige forekomster:} & Én \\ \hline
\textbf{Forudsætning:} 				& UC1: Aktiver system er fuldført og systemet er operationelt. \\ \hline
\textbf{Resultat:}					& Bilens hastighed er ændret. \\ \hline
\textbf{Hovedscenarie:}				& 

\begin{packed_enum}
\item Bruger ændrer position af RT på HID controlleren.
	\begin{packed_item}\itemsep1pt \parskip0pt \parsep0pt
	\item {[}Ext 1.a: Bruger ændrer position af LT.{]}
	\end{packed_item}
\item Controllerens input streames til bilen.
\item Bilen ændrer fremadgående hastighed i henhold til brugerens input. Et hårdere tryk resulterer i en højere hastighed og et lettere tryk resulterer i en lavere hastighed.
\item UC5 afsluttes.
\end{packed_enum} \\ \hline
\textbf{Udvidelser:}				&  
\textbf{{[}Ext 1.a : Bruger ændrer position af LT.{]}}
	\begin{packed_enum}\itemsep1pt \parskip0pt \parsep0pt
		\item Controllerens input streames til bilen.
		\item Bilen ændrer bagudgående hastighed i henhold til brugerens input. Et hårdere tryk resulterer i en højere hastighed og et lettere tryk resulterer i en lavere hastighed.
		\item Systemet fortsætter fra punkt 4 i hovedscenariet.
	\end{packed_enum}
\\ \hline
\end{tabularx}
\caption{UC5: Kør bil frem/tilbage}
\label{tbl:UC5}
\end{table}
\clearpage

% UC6: Drej bil til højre/Venstre
\subsubsection{Use Case 6: Drej bil til højre/venstre}
\begin{table}[h]
\begin{tabularx}{\textwidth}{| L{3.3 cm} | Z |} 													   \hline

\textbf{Navn:} 						& UC6: Drej til højre/venstre									\\ \hline
\textbf{Mål:}						& At få bilen til at dreje mod højre eller venstre				\\ \hline
\textbf{Initiering:}				& Bruger 														\\ \hline
\textbf{Aktører:} 					& Bruger	 													\\ \hline
\textbf{Reference:} 				& UC3 															\\ \hline
\textbf{Antal samtidige forekomster:} & Én 															\\ \hline
\textbf{Forudsætning:} 				& UC1: Aktiver system er fuldført og systemet er operationelt   \\ \hline
\textbf{Resultat:}					& Retningen på bilens forhjul er ændret 						\\ \hline
\textbf{Hovedscenarie:}				& 

\begin{packed_enum}
\item Bruger ændrer position på den venstre styrepind på xbox-360 controlleren.
	\begin{packed_item}\itemsep1pt \parskip0pt \parsep0pt
		\item {[}Ext 1.a: AKS bliver anvendt.{]}
	\end{packed_item}
	\item Controllerens input streames til bilen.
	\item Bilen behandler input fra Bruger, hvis styrepinden føres til venstre drejes forhjulene til venstre, hvis styrepinden føres til højre drejes forhjulene ligeledes til højre.
	\item UC6 afsluttes.
\end{packed_enum} \\ \hline
\textbf{Udvidelser:}				&  
\textbf{{[}Ext 1.a : AKS bliver anvendt.{]}}
	\begin{packed_enum}\itemsep1pt \parskip0pt \parsep0pt
		\item Bilen analyserer input fra UC3.
		\item Bilen drejer til højre, hvis sensor FL registrerer en forhindrer, ditto venstre og FR.
		\item Bilen undviger forhindringen. %TODO Skal eventuelt lige finpudses
		\item Systemet fortsætter fra punkt 3 i hovedscenariet.
	\end{packed_enum}																				\\ \hline
\end{tabularx}
\caption{UC6: Drej til højre/venstre}
\label{tbl:UC6}
\end{table}
\clearpage

% UC7: Brems bil
\begin{longtable}{| l | >{\raggedright}X | >{\raggedright}X | >{\raggedright}X | >{\raggedright\arraybackslash}p{2.3cm} |} \hline
	\multicolumn{2}{|l|}{\textbf{Use case under test}}  & \multicolumn{3}{l|}{UC7: Brems Bil} \\ \hline
	\multicolumn{2}{|l|}{\textbf{Scenarie}} 			& \multicolumn{3}{l|}{Hovedscenarie} \\ \hline
	\multicolumn{2}{|l|}{\textbf{Forudsætning}} 		& \multicolumn{3}{p{10.2cm}|}{UC1: Aktiver system er fuldført og systemet er operationelt.\hfill} \\ \hline
	%\multicolumn{5}{|l|}{}\\ \hline
	\textbf{Step} 	& \textbf{Handling} & \textbf{Forventet Resultat} & \textbf{Resultat} & \textbf{Godkendt / Kommentar} \\ \hline
	
	7.1 & Bruger trykker på ''X'' knappen på Xbox-360 controlleren. 
		& Visuel test: \\ Bruger observerer at bilens hastighed sænkes hvis i fart, ellers tændes bilens bremselys blot. 
		&   
		&  \\ \hline
	
\caption{Accepttest for UC7: Brems Bil }\label{tbl:acceptuc7}
\end{longtable}
\clearpage

% UC8: Konfigurer IP-adresse
\subsection{Use Case 8: Konfigurer system}
%-------------------- UC8 --------------------
\begin{table}[h]
\begin{tabularx}{\textwidth}{| >{\raggedright\arraybackslash}p{3.3 cm} | >{\raggedright\arraybackslash}X |} \hline

\textbf{Navn:} 						& UC8: Konfigurer IP-adresse\\ \hline
\textbf{Mål:}						& At konfigurere bilens IP-adresse til PC'en\\ \hline
\textbf{Initering:}					& Bruger \\ \hline
\textbf{Aktører:} 					& Bruger (primær) \\ \hline
\textbf{Reference:} 					& Ingen\\ \hline
\textbf{Antal samtidige forekomster:} & Én \\ \hline
\textbf{Forudsætning:} 				& UC1: Aktiver system er udført til punkt 3, bilen og PC er på samme netværk, at systemet viser ''Hovedmenu'' samt at systemet er operationelt\\ \hline
\textbf{Resultat:}					& IP adressen på bilen er indstillet\\ \hline
\textbf{Hovedscenarie:}				& 

\begin{packed_enum}
\item Bruger trykker på ''Konfigurer IP''.
\item Konfigurationssmenuen for IP-adressen kommer frem og der er mulighed for at indstaste en IP-adresse
\item Bruger indtaster bilens IP-adresse
\item Bruger trykker på ''Opret forbindelse'' 
\item Menuen indikerer at der er forbindelse til bilen
	\begin{packed_item}\itemsep1pt \parskip0pt \parsep0pt
	\item {[}Ext 5.a Menuen indikerer at der ikke er forbindelse til bilen{]}
	\end{packed_item}
\item Bruger trykker på "tilbage"
\item Systemet viser "Hovedmenu"
\end{packed_enum} \\ \hline
\textbf{Udvidelser:}				&  
\textbf{[}Ext 5.a Menuen indikerer at der ikke er forbindelse til bilen{]}
	\begin{packed_enum}\itemsep1pt \parskip0pt \parsep0pt
	\item Bruger gentager fra punkt 3. 
	\end{packed_enum}
\\ \hline
\end{tabularx}
\caption{UC8: Konfigurer IP-adresse}
\label{tbl:UC8}
\end{table}
\clearpage

% UC9: Tænd/sluk AKS
%-------------------- UC1 --------------------
\begin{table}[h]
\begin{tabularx}{\textwidth}{| >{\raggedright\arraybackslash}p{3.3 cm} | >{\raggedright\arraybackslash}X |} \hline

\textbf{Navn:} 						& UC9: Kalibrer system\\ \hline
\textbf{Mål:}						& At kalibrere systemet så bilen kører lige ved en drejevinkel på 0\% \\ \hline
\textbf{Initering:}					& Bruger \\ \hline
\textbf{Aktører:} 					& Bruger (primær) \\ \hline
\textbf{Reference:} 			    & \\ \hline
\textbf{Antal samtidige forekomster:} & Én \\ \hline
\textbf{Forudsætning:} 				& Systemet er startet op og bilen holder stille\\ \hline
\textbf{Resultat:}					& Bilens styretøj er kalibreret \\ \hline
\textbf{Hovedscenarie:}				& 

\begin{packed_enum}
\item Bruger vælger "Kalibrer system" i "Hovedmenuen" 
\item Menuen skifter til "Kalibreringsmenu" hvor det er muligt at indtaste et nummer mellem -50 og 50 
\item Bruger indtaster et nummer 
\item Bruger trykker på "Save"
\item Systemet gemmer værdien 
\item "Kalibreringsmenu" retunerer til "Hovedmenu"
\end{packed_enum} \\ \hline
\textbf{Udvidelser:}				&  
\textbf{}
\\ \hline
\end{tabularx}
\caption{UC1: Start}
\label{tbl:UC1}
\end{table}
\clearpage

% UC10: Indstil maks. hastighed
\begin{longtable}{| l | >{\raggedright}X | >{\raggedright}X | >{\raggedright}X | >{\raggedright\arraybackslash}p{2.3cm} |} \hline
	\multicolumn{2}{|l|}{\textbf{Use case under test}}  & \multicolumn{3}{l|}{UC10: Indstil makshastighed} \\ \hline
	\multicolumn{2}{|l|}{\textbf{Scenarie}} 			& \multicolumn{3}{l|}{Hovedscenarie} \\ \hline
	\multicolumn{2}{|l|}{\textbf{Forudsætning}} 		& \multicolumn{3}{p{10.2cm}|}{UC1: Aktiver system er udført, bilen og PC er på samme netværk, at systemet viser ''Hovedvindue'' samt at systemet er operationelt.\hfill} \\ \hline
	%\multicolumn{5}{|l|}{}\\ \hline
	\textbf{Step} 	& \textbf{Handling} & \textbf{Forventet Resultat} & \textbf{Resultat} & \textbf{Godkendt / Kommentar} \\ \hline
	10.1 			& Bruger trykker på ''Indstil makshastighed''. & Visuel test: Menu med mulighed for indtastning af makshastighed fremkommer. & Brugeren har mulighed for indtastning  &  \\ \hline
	10.2			& Systemet præsenterer menu makshastighed med mulighed for indtastning af makshastighed fra 1-10 km/t. & Visuel test: Menuen fremkommer & Systemet præsenterer menu makshastighed med mulighed for indtastning & \\ \hline
	10.3			& Menuen indikerer bilens nuværende makshastighed. & den nuværende makshastighed vises & Systemet viser den nuværende makshastighed & \\ \hline
	10.4			& Bruger indtaster bilens nye makshastighed & Den ønskede makshastighed indtastes & Bruger indtaster den ønskede makshastighed & \\ \hline
	10.5			& Bruger trykker på ''Opdater'' & Systemet viser den nye makshastighed & ''Hovedvindue'' viser den nye værdi som makshastighed. & \\ \hline
\caption{Accepttest for UC10: Indstil makshastighed }\label{tbl:acceptuc10}
\end{longtable}
\clearpage

% UC11: Kalibrer styretøj
\subsubsection{Use Case 11: Kalibrer styretøj}
%-------------------- UC11 --------------------
\begin{table}[h]
\begin{tabularx}{\textwidth}{| >{\raggedright\arraybackslash}p{3.3 cm} | >{\raggedright\arraybackslash}X |} \hline

\textbf{Navn:} 						& UC11: Kalibrer styretøj												\\ \hline
\textbf{Mål:}						& At kalibrere systemet så bilen kører ligeud 
									  når brugeren slipper styrepinden på Xbox-360 controlleren 			\\ \hline
\textbf{Initering:}					& Bruger 																\\ \hline
\textbf{Aktører:} 					& Bruger																\\ \hline
\textbf{Reference:} 			    & Ingen																	\\ \hline
\textbf{Antal samtidige forekomster:} & Én 																	\\ \hline
\textbf{Forudsætning:} 				& UC1: Aktiver system er udført, bilen og PC er på samme netværk, 
									  at systemet viser ''Hovedmenu'', at systemet er operationelt 
									  samt bilen holder stille												\\ \hline
\textbf{Resultat:}					& Bilens styretøj er kalibreret 										\\ \hline
\textbf{Hovedscenarie:}				& 

\begin{packed_enum}
	\item Bruger vælger ''Kalibrer styretøj''.
	\item Systemet viser menu for Kalibrering med mulighed for indtastning af værdi mellem -50 og 50, hvor -50 svarer til fuldt udslag til venstre og 50 vil fuldt udslag til højre.
	\item Bruger indtaster værdi. 
	\item Bruger trykker på ''Gem''.
	\item Forhjulene drejer en absolut værdi mod enten, højre eller venstre: positiv værdi oversætte til højre, og negativ værdi oversættes venstre.
	\item Systemet returnerer til ''Hovedvindue''
\end{packed_enum} 																							\\ \hline
\textbf{Udvidelser:}					&  
~																				\\ \hline
\end{tabularx}
\caption{UC11: Kalibrer styretøj}
\label{tbl:UC11}
\end{table}
\clearpage

% UC12: Afbryd system
\begin{longtable}{| l | >{\raggedright}X | >{\raggedright}X | >{\raggedright}X | >{\raggedright\arraybackslash}p{2.3cm} |} \hline
	\multicolumn{2}{|l|}{\textbf{Use case under test}} & 
	\multicolumn{3}{l|}{UC12: Afbryd system} \\ \hline
	
	\multicolumn{2}{|l|}{\textbf{Scenarie}} & 
	\multicolumn{3}{l|}{Hovedscenarie} \\ \hline
	
	\multicolumn{2}{|l|}{\textbf{Forudsætning}} & 
	\multicolumn{3}{p{10.2cm}|}{UC1: Aktiver system er fuldført, bilen holder stille og systemet er operationelt\hfill} \\ \hline
	%\multicolumn{5}{|l|}{}\\ \hline
	\textbf{Step} & \textbf{Handling} & \textbf{Forventet Resultat} & \textbf{Resultat} & \textbf{Godkendt / Kommentar} \\ \hline

	12.1 & Bruger trykker på ''Luk ned''. 
		 & Visuel test:\\ Hovedvinduet forsvinder fra skærmen.
		 & En fejl opstår og programmet lukkes ned efter noget tid.
		 & Ikke OK\\ \hline
		
	12.2 & Bruger venter på at lampen på Piens grønne lys stopper med at lyse
		 & Visuel test:\\ Lampe på Pi slukker.
	 	 & Lampen på Pi er fortsat tændt
		 & Ikke OK\\ \hline
		 
	12.3 & Bruger skubber kontakten ''ON/OFF'' på undersiden af bilen til position ''OFF''
		 & Visuel test:\\ Lampe på strømforsyning slukker.
	 	 & Lampen på strømforsyningen lyser konstant
		 & Ikke OK\\ \hline
		
\caption{Accepttest for UC12: Afbryd system}\label{tbl:acceptuc12}
\end{longtable}
\clearpage

%---------------------------------------------------------------------------------------
%											ACCEPTTESTS
%---------------------------------------------------------------------------------------

%TODO accepttests