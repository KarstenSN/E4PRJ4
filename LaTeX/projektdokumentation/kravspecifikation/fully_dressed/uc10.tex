\subsubsection{Use Case 10: Indstil makshastighed}
%-------------------- UC10 --------------------
\begin{table}[h]
\begin{tabularx}{\textwidth}{| >{\raggedright\arraybackslash}p{3.3 cm} | >{\raggedright\arraybackslash}X |} 			   \hline

\textbf{Navn:} 						& UC10: Indstil makshastighed														\\ \hline
\textbf{Mål:}						& At konfigurere bilens makshastighed												\\ \hline
\textbf{Initering:}					& Bruger 																			\\ \hline
\textbf{Aktører:} 					& Bruger																			\\ \hline
\textbf{Reference:} 				& UC8: Konfigurer IP-adresse														\\ \hline
\textbf{Antal samtidige forekomster:} & Én 																				\\ \hline
\textbf{Forudsætning:} 				& UC1: Aktiver system er udført, bilen og PC er på samme netværk, 
								      at systemet viser ''Hovedvindue'' samt at systemet er operationelt 				\\ \hline
\textbf{Resultat:}					&  																					\\ \hline
\textbf{Hovedscenarie:}				& 

\begin{packed_enum}
	\item Bruger trykker på ''Indstil makshastighed''.
	\item Systemet præsenterer menu makshastighed med mulighed for indtastning af makshastighed fra 1-10 km/t.
	\item Menuen indikerer bilens nuværende makshastighed.
	\item Bruger indtaster bilens nye makshastighed.
	\item Bruger trykker på ''Opdater''. 
	\item ''Hovedvindue'' viser den nye værdi som makshastighed.
	\begin{packed_item}\itemsep1pt \parskip0pt \parsep0pt
		\item {[}Ext 1. ''Hovedvindue'' viser ikke den nye makshastighed{]}
	\end{packed_item}
\end{packed_enum} 																										\\ \hline
\textbf{Udvidelser:}				&  
\textbf{[}Ext 1. Menuen indikerer ikke den nye makshastighed{]}
	\begin{packed_enum}\itemsep1pt \parskip0pt \parsep0pt
		\item Bruger går til UC8
	\end{packed_enum}																									\\ \hline
\end{tabularx}
\caption{UC10: Indstil makshastighed}
\label{tbl:UC10}
\end{table}