%-------------------- UC1 --------------------
\begin{table}[h]
\begin{tabularx}{\textwidth}{| >{\raggedright\arraybackslash}p{3.3 cm} | >{\raggedright\arraybackslash}X |} \hline

\textbf{Navn:} 						& UC1: Start\\ \hline
\textbf{Mål:}						& At starte systemet helt eller delvist. \\ \hline
\textbf{Initering:}					& Bruger \\ \hline
\textbf{Aktører:} 					& Bruger (primær) \\ \hline
\textbf{Reference:} 					& UC10: Monitorering, UC11: Regulering \\ \hline
\textbf{Antal samtidige forekomster:} & Én \\ \hline
\textbf{Forudsætning:} 				& Systemet er stoppet helt, er operationelt og viser hovedmenuen.\\ \hline
\textbf{Resultat:}					& UC10: Monitorering og evt. UC11: Regulering er startet, systemet viser Hovedmenuen. \\ \hline
\textbf{Hovedscenarie:}				& 

\begin{packed_enum}
\item Bruger trykker på "Monitorering". 
\item System aktiverer UC10: Monitorering. 
\item Bruger trykker på "Regulering". 
	\begin{packed_item}\itemsep1pt \parskip0pt \parsep0pt
	\item {[}Ext 3.a : Bruger trykker ikke "Regulering".{]}
	\end{packed_item}
\item Systemet aktiverer UC11: Regulering.
\item UC1 afsluttes.
\end{packed_enum} \\ \hline
\textbf{Udvidelser:}				&  
\textbf{{[}Ext 3.a : Bruger vælger kun monitorering.{]}}
	\begin{packed_enum}\itemsep1pt \parskip0pt \parsep0pt
	\item Systemet fortsætter ved pkt. 5 i hovedscenariet.
	\end{packed_enum}
\\ \hline
\end{tabularx}
\caption{UC1: Start}
\label{tbl:UC1}
\end{table}