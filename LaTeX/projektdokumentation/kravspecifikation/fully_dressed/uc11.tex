\subsubsection{Use Case 11: Kalibrer styretøj}
%-------------------- UC11 --------------------
\begin{table}[h]
\begin{tabularx}{\textwidth}{| >{\raggedright\arraybackslash}p{3.3 cm} | >{\raggedright\arraybackslash}X |} \hline

\textbf{Navn:} 						& UC11: Kalibrer styretøj												\\ \hline
\textbf{Mål:}						& At kalibrere systemet så bilen kører ligeud 
									  når brugeren slipper styrepinden på Xbox-360 controlleren 			\\ \hline
\textbf{Initering:}					& Bruger 																\\ \hline
\textbf{Aktører:} 					& Bruger																\\ \hline
\textbf{Reference:} 			    & Ingen																	\\ \hline
\textbf{Antal samtidige forekomster:} & Én 																	\\ \hline
\textbf{Forudsætning:} 				& UC1: Aktiver system er udført, bilen og PC er på samme netværk, 
									  at systemet viser ''Hovedmenu'', at systemet er operationelt 
									  samt bilen holder stille												\\ \hline
\textbf{Resultat:}					& Bilens styretøj er kalibreret 										\\ \hline
\textbf{Hovedscenarie:}				& 

\begin{packed_enum}
	\item Bruger vælger ''Kalibrer styretøj''.
	\item Systemet viser menu for Kalibrering med mulighed for indtastning af værdi mellem -50 og 50, hvor -50 svarer til fuldt udslag til venstre og 50 vil fuldt udslag til højre.
	\item Bruger indtaster værdi. 
	\item Bruger trykker på ''Gem''.
	\item Forhjulene drejer en absolut værdi mod enten, højre eller venstre: positiv værdi oversætte til højre, og negativ værdi oversættes venstre.
	\item Systemet returnerer til ''Hovedvindue''
\end{packed_enum} 																							\\ \hline
\textbf{Udvidelser:}					&  
~																				\\ \hline
\end{tabularx}
\caption{UC11: Kalibrer styretøj}
\label{tbl:UC11}
\end{table}