\subsection{Use Case 9: Tænd/sluk AKS}
%-------------------- UC9 --------------------
\begin{table}[h]
\begin{tabularx}{\textwidth}{| L{3.3 cm} | Z |} 																			   \hline

\textbf{Navn:} 						& UC9: Tænd/sluk AKS 																	\\ \hline
\textbf{Mål:}						& At tænde eller slukke for AKS på bilen 												\\ \hline
\textbf{Initering:}					& Bruger 																				\\ \hline
\textbf{Aktører:} 					& Bruger 																				\\ \hline  
\textbf{Reference:} 				& UC11: Kalibrer styretøj																\\ \hline
\textbf{Antal samtidige forekomster:} & Én 																					\\ \hline
\textbf{Forudsætning:} 				& UC1: Aktiver system er udført, bilen og PC er på samme netværk, 
									   at systemet viser ''Hovedvindue'' samt at systemet er operationelt					\\ \hline
\textbf{Resultat:}					&  																						\\ \hline
\textbf{Hovedscenarie:}				& 

\begin{packed_enum}
\item Bruger trykker på ''AKS''.
\item System viser ''AKS-menu''.
\item AKS-menu giver Bruger mulighed for at tænde/slukke for AKS .
\item AKS-menu viser status for AKS.
\item Bruger trykker tænd/sluk efter ønske.
\item Bilen tænder/slukker for AKS systemet efter brugerens ønske.
\item ''Hovedvindue'' indikerer nuværende status af AKS.
\end{packed_enum} 																											\\ \hline
\textbf{Udvidelser:}				&  
~
																															\\ \hline
\end{tabularx}
\caption{UC9: Tænd/sluk AKS}
\label{tbl:UC9}
\end{table}