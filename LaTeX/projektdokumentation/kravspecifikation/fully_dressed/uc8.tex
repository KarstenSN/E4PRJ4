\subsection{Use Case 8: Konfigurer system}
%-------------------- UC8 --------------------
\begin{table}[h]
\begin{tabularx}{\textwidth}{| >{\raggedright\arraybackslash}p{3.3 cm} | >{\raggedright\arraybackslash}X |} \hline

\textbf{Navn:} 						& UC8: Konfigurer IP-adresse\\ \hline
\textbf{Mål:}						& At konfigurere bilens IP-adresse til PC'en\\ \hline
\textbf{Initering:}					& Bruger \\ \hline
\textbf{Aktører:} 					& Bruger (primær) \\ \hline
\textbf{Reference:} 					& Ingen\\ \hline
\textbf{Antal samtidige forekomster:} & Én \\ \hline
\textbf{Forudsætning:} 				& UC1: Aktiver system er udført til punkt 3, bilen og PC er på samme netværk, at systemet viser ''Hovedmenu'' samt at systemet er operationelt\\ \hline
\textbf{Resultat:}					& IP adressen på bilen er indstillet\\ \hline
\textbf{Hovedscenarie:}				& 

\begin{packed_enum}
\item Bruger trykker på ''Konfigurer IP''.
\item Konfigurationssmenuen for IP-adressen kommer frem og der er mulighed for at indstaste en IP-adresse
\item Bruger indtaster bilens IP-adresse
\item Bruger trykker på ''Opret forbindelse'' 
\item Menuen indikerer at der er forbindelse til bilen
	\begin{packed_item}\itemsep1pt \parskip0pt \parsep0pt
	\item {[}Ext 5.a Menuen indikerer at der ikke er forbindelse til bilen{]}
	\end{packed_item}
\item Bruger trykker på "tilbage"
\item Systemet viser "Hovedmenu"
\end{packed_enum} \\ \hline
\textbf{Udvidelser:}				&  
\textbf{[}Ext 5.a Menuen indikerer at der ikke er forbindelse til bilen{]}
	\begin{packed_enum}\itemsep1pt \parskip0pt \parsep0pt
	\item Bruger gentager fra punkt 3. 
	\end{packed_enum}
\\ \hline
\end{tabularx}
\caption{UC8: Konfigurer IP-adresse}
\label{tbl:UC8}
\end{table}