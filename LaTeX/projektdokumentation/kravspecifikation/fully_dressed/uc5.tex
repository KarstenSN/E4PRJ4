\subsection{Use Case 5: Kør bil frem/tilbage}
\begin{table}[h]
\begin{tabularx}{\textwidth}{| L{3.3 cm} | Z |} \hline

\textbf{Navn:} 						& UC5: Kør bil frem/tilbage\\ \hline
\textbf{Mål:}						& At få bilen til at køre frem eller tilbage. \\ \hline
\textbf{Initiering:}				& Bruger \\ \hline
\textbf{Aktører:} 					& Bruger \\ \hline
\textbf{Reference:} 				& Ingen \\ \hline
\textbf{Antal samtidige forekomster:} & Én \\ \hline
\textbf{Forudsætning:} 				& UC1: Aktiver system er fuldført og systemet er operationelt. \\ \hline
\textbf{Resultat:}					& Bilens hastighed er ændret. \\ \hline
\textbf{Hovedscenarie:}				& 

\begin{packed_enum}
\item Bruger ændrer position af RT på HID controlleren.
	\begin{packed_item}\itemsep1pt \parskip0pt \parsep0pt
	\item {[}Ext 1.a: Bruger ændrer position af LT.{]}
	\end{packed_item}
\item Controllerens input streames til bilen.
\item Bilen ændrer fremadgående hastighed i henhold til brugerens input. Et hårdere tryk resulterer i en højere hastighed og et lettere tryk resulterer i en lavere hastighed.
\item UC5 afsluttes.
\end{packed_enum} \\ \hline
\textbf{Udvidelser:}				&  
\textbf{{[}Ext 1.a : Bruger ændrer position af LT.{]}}
	\begin{packed_enum}\itemsep1pt \parskip0pt \parsep0pt
		\item Controllerens input streames til bilen.
		\item Bilen ændrer bagudgående hastighed i henhold til brugerens input. Et hårdere tryk resulterer i en højere hastighed og et lettere tryk resulterer i en lavere hastighed.
		\item Systemet fortsætter fra punkt 4 i hovedscenariet.
	\end{packed_enum}
\\ \hline
\end{tabularx}
\caption{UC5: Kør bil frem/tilbage}
\label{tbl:UC5}
\end{table}