\subsection{Use Case 4: Undvig forhindring}
\begin{table}[h]
\begin{tabularx}{\textwidth}{| L{3.3 cm} | Z |} \hline

\textbf{Navn:} 						& UC4: Undvig forhindring											\\ \hline
\textbf{Mål:}						& At bilen undviger en evt. kollision med en forhindring.	\\ \hline
\textbf{Initering:}					& UC3: Overvåg sensor 										\\ \hline
\textbf{Aktører:} 					& Forhindring 												\\ \hline
\textbf{Reference:} 				& UC3, UC6: Drej bil til højre/venstre, UC7: Brems bil 		\\ \hline
\textbf{Antal samtidige forekomster:} & Én 														\\ \hline
\textbf{Forudsætning:} 				& UC1 er gennemført, UC3 er gennemført, 
									  bilen er på vej mod en forhindring. 						\\ \hline
\textbf{Resultat:}					& UC5, UC6 og/eller UC7 gennemføres og UC3 fortsætter. 		\\ \hline
\textbf{Hovedscenarie:}				& 

\begin{packed_enum}
	\item Bilen analyserer indsamlet data fra afstandssensorer, kører den fremad analyseres de forreste sensorer ditto bagud.
	\item AKS overtager styring fra Bruger midlertidigt.
	\item 
		\begin{packed_item}\itemsep1pt \parskip0pt \parsep0pt
			\item {[}ALT a: UC6: Drej bil til højre/venstre aktiveres, hvis en enkelt sensor registrerer en forhindring{]}
			\item {[}ALT b: UC7: Brems bil aktiveres hvis begge sensorer registrer en forhindring{]}
		\end{packed_item}
	\item Bilen giver igen styring tilbage til brugeren.
	\item UC4 afsluttes.
\end{packed_enum} 																				\\ \hline
	\textbf{Udvidelser:}				&   													\\ \hline
\end{tabularx}
\caption{UC4: Undvig forhindring}
\label{tbl:UC4}
\end{table}