\section{Protokolbeskrivelse}

\subsection{Kamera}

Kameraet er påmonteret Pi'en og der er installeret en virtuel driver kaldet uv4l. uv4l er en overbygning på den generelle driver således programmet motion kan finde kamaret i /dev/video0. Motion har den fordel at det kan sætte et IP videostream op hvorved Pi’en kan streame videodata den vej igennem. På PC'en modtages videoen igennem GUI'en ved hjælp af opensource biblioteker fra VLC media player. Herved laves der en socket til at modtage videoen i. Adressen hvor i video stream modtages er Pi'ens Ip-adresse port 8081.

\begin{table}[h]
\begin{tabularx}{\textwidth}{| Z | Z | Z | Z |} \hline

\textbf{Kommando} 						&
\textbf{Svar}							&
\textbf{Beskrivelse}					&
\textbf{Bitmønstre}						\\ \hline

openPlayer								&
Kontinuert stream fra motion			&
Motion streamer video data til PC'en	&
UDP socket forbindelse					\\ \hline

\end{tabularx}
\caption{Kamera Protokol}
\label{tbl:prt_cam}
\end{table}

Motion streamer video ligeså snart der er tændt for bilen. Der er derfor ikke nogen direkte kommunikation mellem PC'en og Pi'en igennem Kameraprotokollen.


\subsection{GUI}

PC'en og bilen kommunikerer igennem to TCP forbindelser. Xbox360 Controlleren har en TCP forbindelse hvor controller data bliver sendt kontinuert. GUI'en har en anden TCP forbindelse hvor data også udveksles kontinuert. Data består af makshastighed, afstand til forhindring, AKS-status,  styretøj, acceleration og hastighed på bilen. TCP forbindelse til GUI'en sker på Pi'ens Ip-adresse port 1234 og controlleren har port 1235.

\begin{table}[h]
\begin{tabularx}{\textwidth}{| Z | Z | Z | Z |} \hline

\textbf{Kommando} 						&
\textbf{Svar}							&
\textbf{Beskrivelse}					&
\textbf{Bitmønstre}						\\ \hline

getData &

sendStatus &

GUI'en udveksler dens data med bilens &

Ved getData sendes data som char array i rækkefølgen: makshastighed(km/h), hastighed((km/h)$\times$10), afstand(m$\times$10), acceleration(g$\times$10), AKS-status(1/0) og styretøj(-50..50). \\ \hline



shutDown &

shuttingDown &

GUI'en sender besked til bilen om at den skal lukke dens software ned &

shutDown sendes som en string ''dwnnow'' \
bilen sender ''dwnnow'' tilbage som ACK. Alt andet tolkes som NACK \\ \hline



controllerFrem &

\textit{Intet svar} &

Bruger har trykket på RT på controlleren &

Char fra 0-255 afhængig af hvor hårdt der trykkes på knappen \\ \hline



controllerTilbage &

\textit{Intet svar} &

Bruger har trykket på LT på controlleren &

Char fra 0-255 afhængig af hvor hårdt der trykkes på knappen \\ \hline



controllerVenstre &

\textit{Intet svar} &

Bruger har ændret positionen af venstre styrepind, til venstre, på controlleren &

Char ''V'' \\ \hline



controllerHøjre &

\textit{Intet svar} &

Bruger har ændret positionen af venstre styrepind, til højre, på controlleren &

Char ''H'' \\ \hline

\end{tabularx}
\caption{GUI Protokol}
\label{tbl:prt_gui}
\end{table}